\documentclass[10pt]{sigplanconf}

\usepackage{amssymb}
\usepackage{amsmath}
\usepackage{amsthm}
\usepackage{stmaryrd}
\usepackage{color}
\usepackage{graphics}

\definecolor{darkgreen}{rgb}{0.0,0.5,0.0}
\definecolor{darkpurple}{rgb}{0.6,0.0,0.6}
\definecolor{orange}{rgb}{0.8,0.4,0.0}
\definecolor{darkorange}{rgb}{0.5,0.2,0.0}
\definecolor{marco}{rgb}{0.0,0.3,0.5}
\definecolor{gray}{rgb}{0.2,0.2,0.2}

\newcommand{\dnote}[1]{\textcolor{darkpurple}{Dom: #1}}

\newcounter{block}

\newtheorem{theorem}[block]{Theorem}
\newtheorem{lemma}[block]{Lemma}
\newtheorem{proposition}[block]{Proposition}
%\newtheorem{definition}[block]{Definition}

\theoremstyle{definition}
\newtheorem{remark}[block]{Remark}
\newtheorem{example}[block]{Example}
\newtheorem{definition}[block]{Definition}

% Writing macros
\newcommand{\ie}{\emph{i.e..}}
\newcommand{\eg}{\emph{e.g.}}

% Semantics related
\newcommand{\interp}[1]{\llbracket{#1}\rrbracket}

\title{Abstract Shape Specifications for Stencil Verification}
\authorinfo{}{}

\begin{document}
\maketitle

\begin{abstract}
\end{abstract}

\bibliographystyle{plain}

\section{Introduction}


\section{Specification language}

\begin{align*}
spec ::= & \; S  \; \texttt{,} \; spec \, \mid \, \epsilon \\ \\
%
S ::= &  \; \texttt{forward} \; [\textit{ixs}] \; [\textit{depth}] \\
& \; \mid \texttt{backward} \; [\textit{ixs}] \; [\textit{depth}] \\
& \; \mid \texttt{centered} \; [\textit{ixs}] \; [\textit{depth}] \\
& \; \mid \texttt{unspecified} \; [\textit{ixs}]
\end{align*}

\section{Static analysis}

\begin{definition}[Induction variable]

\end{definition}

\begin{definition}[Constant translations]
An indexing expression $e$ is a \emph{constant translation} if,
for an \emph{induction variable} $i$, then $e \equiv i + a$ or $e \equiv i - a$ 
where $a$ is a constant. The relation $\equiv$ is used which
represents term equality up-to commutativity of $+$ and the inverse
relation of $+$ and $-$ (\eg{}, $(-b) + i \equiv i - b$). 
In the following, we classify constant translation expressions 
using the predicate $\textsf{trans}$.
\end{definition}

\paragraph{Step 1: Data-access analysis}

\dnote{Might want to do 'stencil' (gather) and 'mould' (scatter)
  patterns separately}
\begin{align*}
\begin{array}{lll}
\interp{e_1 = e_2}    & = \interp{e_2} \\
\interp{e_1 e_2}      & = \interp{e_1} \cup \interp{e_2} \\
\interp{a(\tilde{e})} & = \{e | e \in \tilde{e}, \textsf{trans}(e)\} & \textit{where each $e \in
                                          \tilde{e}$ is a constant or constant translation}
\end{array}
\end{align*}

Index analysis is a \emph{coeffect analysis}.

\paragraph{Step 2: Coalesce contiguous indices into regions}

% fairly mechanical

\section{Evaluation}

\section{Discussion}


\bibliography{references}

\end{document}