\documentclass[9pt]{sigplanconf}

\usepackage{amssymb}
\usepackage{amsmath}
\usepackage{amsthm}
\usepackage{stmaryrd}
\usepackage{color}
\usepackage{graphics}
\usepackage{fancyvrb}
\usepackage{subfigure}
\usepackage{amsthm}

\CustomVerbatimEnvironment{SpecVerbatim}{Verbatim}{fontsize=\footnotesize,xleftmargin=0.65cm,
xrightmargin=0.2cm,commandchars=\\\{\},baselinestretch=0.97}
\CustomVerbatimEnvironment{ExmVerbatim}{Verbatim}{fontsize=\footnotesize,xleftmargin=0.65cm,
xrightmargin=0.2cm,baselinestretch=0.97,numbers=left}
\CustomVerbatimEnvironment{IVerbatim}{Verbatim}{fontsize=\relsize{-1},xleftmargin=0.65cm,
xrightmargin=0.2cm,commandchars=\\\{\},baselinestretch=0.97}


\definecolor{darkgreen}{rgb}{0.0,0.5,0.0}
\definecolor{darkpurple}{rgb}{0.6,0.0,0.6}
\definecolor{orange}{rgb}{0.8,0.4,0.0}
\definecolor{darkorange}{rgb}{0.5,0.2,0.0}
\definecolor{marco}{rgb}{0.0,0.3,0.5}
\definecolor{gray}{rgb}{0.2,0.2,0.2}

\newcommand{\bn}{\mathbb{N}}

\newcommand{\dnote}[1]{\textcolor{darkpurple}{Dom: #1}}
\newcommand{\mnote}[1]{\textcolor{darkgreen}{Mistral: #1}}

\newcounter{block}

\newtheorem{lemma}[block]{Lemma}
\newtheorem{proposition}[block]{Proposition}
%\newtheorem{definition}[block]{Definisssstion}

\theoremstyle{definition}

\newtheorem{theorem}[block]{Theorem}
\newtheorem{remark}[block]{Remark}
\newtheorem{example}[block]{Example}
\newtheorem{definition}[block]{Definition}

% Writing macros
\newcommand{\ie}{\emph{i.e..}}
\newcommand{\eg}{\emph{e.g.}}

\newcommand{\dimId}{\texttt{dim}}

% Semantics related
\newcommand{\interp}[1]{\llbracket{#1}\rrbracket}

% Syntax macros
\newcommand{\stenFwd}[2]{\texttt{forward}, \texttt{depth}=#1,
  \texttt{dim}=#2}
\newcommand{\stenBwd}[2]{\texttt{backward}, \texttt{depth}=#1,
  \texttt{dim}=#2}
\newcommand{\stenCen}[2]{\texttt{backward}, \texttt{depth}=#1,
  \texttt{dim}=#2}

\title{Time and space specifications for stencils}
\authorinfo{}{}

\begin{document}
\maketitle

\begin{abstract}
\end{abstract}

\bibliographystyle{plain}

\section{Introduction}

\emph{Stencils} are a ubiquitous programming pattern in many numerical
applications. Informally, a stencil computation computes
an array, where the value at each index $i$ of this array is
calculated from a set of neighbouring values to $i$ in some input
array(s), \eg{}, the Game of Life, convolutions in image processing,
approximations to differential equations. For example, the following
psuedo-code iteratively applies the discrete Laplace
operator (an approximation to a derivative):
%
\begin{ExmVerbatim}
do iter = 0, itermax
   do i = 1, (n-1)
      b(i) = a(i-1) - 2*a(i) + a(i+1)
   a = b
\end{ExmVerbatim}
%
Line 3 is the core of the stencil computation, calculating
the value at \texttt{b[i]} from a neighbourhood of elements about
\texttt{a[i]}. Line 4 swaps
\texttt{a} and \texttt{b} between iterations, where \texttt{b} becomes the
input for the next iteration.

%This stencil computation exhibits statically decideable
%spatial and temporal relationships between $\texttt{a}$ and
%$\texttt{b}$.
In this simple example, the spatial and temporal relationship
between \texttt{a} and \texttt{b} is easy to understand. However,
more complex stencil computations can be much harder to understand
and subsequently more prone to error. For example, Figure~\ref{ref:navier-stokes-fragment}
shows a couple of lines from a Navier-Stokes fluid simulator in
which two arrays are read from with different data access patterns,
across two dimensions. The interaction is much harder to understand,
with the potential for the developer to accidentally introduce an
error, for example writing $\texttt{(i-1,j)}$ instead of
$\texttt{(i+1,j)}$.

We introduce a simple specification for stencil comptuations
that abstracts the fine grained detail of their access patterns.
In practise, most stencil computations have a regular shape that
can be described simply and abstractly. In the case of our
first Laplace example, our analysis tool infers the following specification:
%
\begin{SpecVerbatim}
%=  stencil centered, depth=1 dims=1 :: a
%=  stencil dependency a, mutual    :: b
\end{SpecVerbatim}
%
The first line explains that \texttt{a} is read from
with a symmetrical stencil pattern (``centered'') to a depth
of one in each direction in its first (and only) dimension.
The second line explains the temporal relationship between
\texttt{b} and \texttt{a}: that the previous time step for \texttt{b}
is actually provided by \texttt{a}. The dimension \texttt{t} is a
reserved dimension identifiers for temporal specifications. In
the case of the Navier-Stokes example, its inferred stencil is shown
in Figure~\ref{ref:navier-stokes-fragment}(b).

In this paper, we make the following contributions:
%
\begin{itemize}
\item we introduce a specification language for
stencil computations that captures many common forms
of data access pattern, both spatial and temporal (Section~\ref{sec:lang});
\item we detail inference and checking
algorithms for stencil specifications (Section~\ref{sec:analysis});
\item we evaluate our implementation of the approach
in the CamFort tool for Fortran verification, studying
a number of example programs to assess the usefulness
of this approach.
\dnote{insert results here}
\end{itemize}
%

\begin{figure}
\begin{ExmVerbatim}[firstnumber=20]
du2dx = ((u(i,j)+u(i+1,j))*(u(i,j)+u(i+1,j))+
    gamma*abs(u(i,j)+u(i+1,j))*(u(i,j)-u(i+1,j))-
    (u(i-1,j)+u(i,j))*(u(i-1,j)+u(i,j))-
    gamma*abs(u(i-1,j)+u(i,j))*(u(i-1,j)-u(i,j)))
    /(4.0*delx)

duvdy = ((v(i,j)+v(i+1,j))*(u(i,j)+u(i,j+1))+
   gamma*abs(v(i,j)+v(i+1,j))*(u(i,j)-u(i,j+1))-
   (v(i,j-1)+v(i+1,j-1))*(u(i,j-1)+u(i,j))-
   gamma*abs(v(i,j-1)+v(i+1,j-1))*(u(i,j-1)-u(i,j)))
   /(4.0*dely)

laplu = (u(i+1,j)-2.0*u(i,j)+u(i-1,j))/delx/delx+
          (u(i,j+1)-2.0*u(i,j)+u(i,j-1))/dely/dely
\end{ExmVerbatim}
(a). Excerpt from the Fortran code,
showing highly-detailed stencil computations. \\

%\begin{SpecVerbatim}[xleftmargin=0.1cm]
%20-24  u: centered depth=1 dim=1
%6-30  u: centered depth=1 dim=2
%      v: forward depth=1 dim=1, backward depth=1 dim=1
%32-33  u: centered depth=1 dim=1,2
%\end{SpecVerbatim}

\begin{SpecVerbatim}[xleftmargin=0.3cm]
%=   stencil (centered, depth=1, dim=1)
%=         + (centered, depth=1, dim=2) :: u

%=   stencil (forward, depth=1, dim=1)
%=         + (backward, depth=1, dim=1) :: v
\end{SpecVerbatim}
(b). Inferred stencil specification from CamFort
\caption{Fragment of Navier-Stokes fluid simulator and its specification}
\label{ref:navier-stokes-fragment}
\end{figure}

\section{Stencil specification language}
\label{sec:lang}

Our specification system is based on the observation
that most forms of array access in numberical code have
a fixed, statically-determined access pattern. For example, the
``\emph{five-point stencil}'' on a two-dimensional array reads from array
indices $(i, j)$, $(i-1, j)$, $(i+1, j)$, $(i, j-1)$, and $(i, j+1)$
for all $i, j$ within the inner boundary of the array (to avoid
out-of-bounds access at the edges). We revisit this hypothesis
in Section~\ref{sec:evaluation} where the inference of
such regular stencil patterns on a corpus of numerical programs (both
small and large). We found that indeed \dnote{..}.

We outline the specification language here. Section~\ref{sec:syntax}
oultines the syntax. Section~\ref{sec:semantics} defines its semantics
via a simple multi-set interpretation over indices. Section~\ref{sec:eqs} provides an
equational theory for specifications via relation ($\equiv$) and a
theory of approximation via relation ($<:$). These equations are then
proven sound with respecto the multi-set semantics of Section~\ref{sec:semantics}.

\subsection{Syntax}
\label{sec:syntax}

Figure~\ref{fig:syntax}
gives the syntax, which we introduces in stages below.
The top-level is given by the \textit{spec} production
which splits into either a \textit{spatial} specification or
a \textit{temporal specification}.
%%

\begin{figure}[t]
\begin{align*}
\def\arraystretch{1.2}
\setlength{\arraycolsep}{0.2em}
\newcommand{\dimTy}{\mathbb{D}}
\begin{array}{rl}
\textit{specification} ::= & [\textit{regionDec} \; \{\texttt{$\backslash$n} \;\;
                             \textit{regionDec} \}] \;
                             \textit{specDec} \\
\textit{specDec} ::= & \texttt{stencil} \; \textit{spec} \mid \epsilon
  \\
\textit{regionDec} ::= &  \texttt{region} \; \textit{rvar} \; \texttt{=} \;
                         \textit{region}\\
\textit{spec} ::= & \textit{spatial} \mid \textit{temporal}   \\[1em]
\textit{spatial} ::= & [\textit{mod} \; \{, \textit{mod} \}] \; \textit{region} \\
\textit{mod} ::= &
 \texttt{read-once} \\
\mid \; & \texttt{reflexive} \; \texttt{dims=}\dimTy \{\texttt{,} \; \dimTy\} \\
\mid \; & \texttt{irreflexive} \; \texttt{dims=}\dimTy \{\texttt{,} \;
          \dimTy\} \\[0.1em]
\textit{region} ::=
     \; & \texttt{forward} \; [,\texttt{depth=}\mathbb{N}] \; , \texttt{dim=}\dimTy  \\
\mid \; & \texttt{backward} \; [,\texttt{depth=}\mathbb{N}] \; , \texttt{dim=}\dimTy \\
\mid \; & \texttt{centered} \; [,\texttt{depth=}\mathbb{N}] \; , \texttt{dim=}\dimTy \\
\mid \; & \textit{region} \; \texttt{+} \; \textit{region} \\
\mid \; & \textit{region} \; \texttt{*} \; \textit{region} \\
\mid \; & \textit{rvar}  \\[0.5em]
\textit{temporal} ::= \; & \texttt{dependency} \; (v \; \{ , v \})
  \\[0.5em]
\dimTy ::= \; & \mathbb{N}_{>0} \\
\textit{rvar} ::= \; & [\text{\texttt{a}-\texttt{z}$\,$\texttt{A}-\texttt{Z}$\,$\texttt{0}-\texttt{9}}]+
\end{array}
\end{align*}
\caption{Specification syntax (EBNF grammar)}
\label{fig:syntax}
\end{figure}

\paragraph{Spatial specifications}


\paragraph{Temporal specifications}


\subsection{A semantic model for specifications}
\label{sec:semantics}

\newcommand{\relix}{(\mathbb{Z}_\bot)^\mathbb{D}}

We give a model of our specifications which explains
the range of indices which correspond to the specifications.
The model is based on multi-sets since array indices my be repeated
in an expression.

Figure~\ref{fig:model} describes the model in terms of
an interpretation function $\interp{-}$ which maps specifications
to multisets of relative indices.

\begin{definition}Relative indices are drawn
from $(\mathbb{Z}_{\bot})^{\mathbb{D}}$, essentially an infinite
tuple of offsets. For example, $(i, j+1)$ corresponds to
the relative index $(0, 1, \bot, \bot, \ldots)$
\end{definition}

The main interpretation function is overloaded on lists of
variable-spec pairs $\interp{\overline{v : S}}$,
returning multisets of variable-relative-index pairs. This provides
the top-level definition of the model, with $\interp{-}$ overloaded
on $S$, $\interp{S}$ mqpping to multisets of
relative indices not associated to an array avariable.

\begin{figure}

\noindent
We define a lattice model of array access patterns which serves as both a
denotational model of the semantics of our specification language and an
abstract interpretation domain for source code. This model (1) serves to
explain the meaning of our specifications; (2) is used in the inference and
checking algorithms (\Cref{sec:analysis}); (3) justifies an equational theory
for specifications in the next section; (4) is used to optimise specifications
using lattice identities; and (5) can be used to guide correct implementations.

The model is defined over vectors of sets of integers which we call
\emph{index schemes}.  As an initial informal example, consider the
following simple stencil computation:
%
\begin{minted}{fortran}
do i = 1, n
  b(i) = a(i,0) + a(i+1,0)
end do
\end{minted}
%
The access pattern on array \term{a}, relative
to induction variable \term{i}, is captured
by a vector of length 2 containing integer sets $\langle{\{0, 1\},
  \mathbb{Z}\}}\rangle$. This describes that, in the first dimension,
the array is read at offsets of $0$ and $1$ from an induction
variable. In the second dimension, the index is unconstrained
as it is a constant.

Index schemes form a lattice which provides a rich set of equations
and properties, which we exploit. We first set up the 
domain of the model (\S\ref{sec:domain}), using it to define a semantics
for our specification language (\S\ref{sec:semantics}) and then as
the target for an abstract interpretation on imperative
code (\S\ref{sec:fromcode}).

\subsection{Lattice model of regions}
\label{sec:domain} 

%\begin{defn}[Extended integers]
%  We define the set \zinf{} as
%  $\mathbb{Z}$ extended with $\infty$ and $-\infty$. For any $a$ in
%  \zinf{}, we have $-\infty \leq a \leq \infty$. The resulting set is
%  a total order with top and bottom elements $-\infty$ and
%  $\infty$ respectively.
%\end{defn}

\begin{defn}[Index scheme]
  An $N$-dimensional \emph{index scheme} is a vector of length $N$ of
  integer sets, \ie{}, a member of $\mathcal{P}(\bz)^{N}$.
%  An equivalent view of these vectors is as an $N$-times finite Cartesian
%  product on subsets of $\bz$.
  We use $S, T, U$ to denote index schemes. Henceforth, we assume index schemes
  are all $N$-dimensional for some $N$.

  Index schemes can be \emph{projected} in the $i^{\textit{th}}$ dimension by
  $\pi_i : {\mathcal{P}(\bz{})}^N \to \mathcal{P}(\bz{})$. For an index scheme
  $I$, we refer to $\pi_i(S)$ as the \emph{$i^{th}$ component} of $S$. We assume
  that $i$, when used for projection, always lies between $1$ and $N$.
\end{defn}


\begin{restatable}{lem}{vectorIntersect}
\label{lem:vector-intersect}
  Intersection distributes over index schemes. That is, for index schemes $S, T
  \in \mathcal{P}(\bz)^N$
%
  \begin{equation*}
    S \cap T = \prod_{i = 1}^{N} \pi_i(S) \cap \pi_i(T)
  \end{equation*}
\end{restatable}

Union does not distribute over index schemes, however, a more restricted
property holds.

\begin{restatable}{lem}{vectorUnion}
\label{lem:vector-union}
$S$ and $T$ are index schemes such that $\pi_i(S) = \pi_i(T)$ for all $1 \leq i
\leq N$ apart from some dimension $k$, then:
%
  \begin{equation*}
    S \cup  T
    =
    \pi_1(S) \times \cdots \times
    (\pi_k(S) \cup \pi_k(T)) \times \cdots \times
    \pi_N(S)
  \end{equation*}
\end{restatable}
%
\begin{defn}[Intervals with an optional hole]
  We define an extended notion of closed interval on $\bz{}$ which may contain a
  \emph{hole} at the origin, written \interv{a}{b}{c} where $a$ and $b$ are
  drawn from \bz{} with $a \leq 0 \leq b$ and $c$ is drawn from $\mathbb{B} = \{
  \mathsf{true}, \mathsf{false} \}$. Intervals are interpreted as sets as
  follows:
%
  \begin{equation*}
    \interv{a}{b}{c} \triangleq
      \{ n \mid a \leq n \leq b \wedge (\neg c \implies n \neq 0) \}
  \end{equation*}

  We also add the distinguished interval $\interv{-\infty}{\infty}{}$, which is simply an
  alias for \bz{}, but this notation prevents handling infinite interval
  separately in the following definitions, theorems, and proofs. Here, $-\infty$
  and $\infty$ behaves like top and bottom elements to \bz{} respectively.

  We denote the set of all such intervals (sets) as $\textit{Interval}$. If the
  superscript to the interval is omitted it is treated as $\mathsf{true}$.
\end{defn}
%
\begin{restatable}{lem}{intervalIdentities}
 \label{lem:interval-identities}
  We have the following dual identities for \bz{} intervals:
%
  \begin{align*}
    \interv{a}{b}{c} \cap \interv{d}{e}{f} & =
      \interv{\max \{a,d\}}{\min \{b,e\}}{c \wedge f} \\
    \interv{a}{b}{c} \cup \interv{d}{e}{f} & =
      \interv{\min \{a,d\}}{\max \{b,e\}}{c \vee f}
  \end{align*}
\end{restatable}
%

We define two specialisations (and subset spaces) of index scheme:
\emph{subscript scheme} and \emph{interval scheme}:

\begin{defn}[Interval scheme]\label{def:interval-scheme}
  An interval scheme is a finite Cartesian product of intervals on \bz{},
  denoted by the set $\textit{Interval}^N$ for a product of $N$ intervals.
\end{defn}

\begin{defn}[Subscript scheme]
  A \emph{subscript scheme} is an index scheme scheme where:
  %
  \begin{equation*}
    \forall i.\ 1 \leq i \leq N \implies
      \pi_i (S) = \{ p \}
      \; \vee \;
      \pi_i (S) = \interv{-\infty}{\infty}{}
    \end{equation*}
%
  That is, the $i^{th}$ component of the set is either a singleton in \bz{} or
  the infinite interval.
\end{defn}
%
\begin{defn}[Region]
  A region is an index scheme and \region{N} is the set of all regions (\ie{},
  $\region{N} \subseteq \bz^N)$. The set of all regions is  defined
  as the smallest set satisfying the following:
%
  \begin{enumerate}
    \item If $R$ is in $\textit{Interval}^N$, then $R$ is in \region{N}.
    \item If $R$ and $S$ are in \region{N}, then so are $R \cap S$ and
      $R \cup S$.
  \end{enumerate}
\end{defn}
%
\begin{restatable}{prop}{regionLattice}
  \label{prop:regionLattice}
  $(\region{N},\cup,\cap,\subseteq)$ is a bounded distributive lattice with top
  $\top = \bz{}^N$ and bottom $\bot = \emptyset$.
\end{restatable}
%
\begin{proof}
  Straightforward, the join and meet are mapped to $\cup$ and $\cap$, the set is
  inductively designed to be closed under these operations. Union and
  intersection are associative, commutative, and absroptive under closed sets
  and then they are also for \region{N}. This is enough to show that, it is a
  lattice.

  Further, we have $\bz{}^N \cap R = R$ and $\emptyset \cup R = R$ with
  $\bz{}^N$ and $\emptyset$ belonging to \region{N}. This makes the lattice a
  bounded one.

  Finally, the lattice is distributive since union distributes over intersection
  and vice versa when the set is closed under these operations.
\end{proof}

\begin{defn}
  $\mathsf{Mult}$ and $\mathsf{Approx}$ are parametric labelled variant types
  with injections given by their definition:
%
  \begin{align*}
    \mathsf{Mult} \;\; a \;\; &
      \triangleq \mathsf{mult} \; a \;\mid\; \mathsf{only} \; a \\
    \mathsf{Approx} \;\; a \;\; &
      \triangleq \mathsf{exact} \; a \;\mid\; \mathsf{lower} \; a \;\mid\;
        \mathsf{upper} \; a
  \end{align*}
%
  \eg{}, $\mathsf{lower}$ is an injection $\mathsf{lower} : a \to \mathsf{Approx}
  \; a$ etc.
  These will be used in the following subsection to give meaning to the
  specification modifiers for approximation and multiplicity.
\end{defn}

\subsection{Denotational semantics for specifications}
\label{sec:semantics}

\noindent
An interpretation function $\interp{-}_N$ maps closed\footnote{That
  is, we assume there are no occurrences of \textit{rvar} in a
  specification being modelled.  Any \emph{open} specification
  containing region variables can be made closed by straightforward
  syntactic substitution with a (closed) \textit{region}.}
specifications to sets of $N$-dimensional index schemes with modifier
information, \ie{} specifications are mapped to
$\textsf{Mult} (\textsf{Approx} (\region{N}))$.

The interpretation is overloaded on \emph{regions} in
\Cref{subfig:region-model}. Various intermediate notions are used.

\begin{defn}
  Let $\textit{promote}_N : \mathbb{N}^+ \times \textit{Interval} \to
  \textit{Interval}^N$ be a function generating an interval scheme such that if
  $v$ is $\vecgen{N}{i}{\interv{a}{b}{c}}$, then $\pi_i(v) = \interv{a}{b}{c}$
  and $\pi_j(v) = \bz{}$ in all other dimensions $j$.
\end{defn}

%
\begin{figure}[!t]
\begin{subfigure}[t]{0.5\textwidth}
\begin{align*}
  \interp{-}_N & : \textit{region} \rightarrow \region{N} \\
%
  \interp{\stencil{p}{$i$}{}{}}_N & =
    \vecgen{N}{i}{\interv{0}{0}{\mathsf{true}}}\\
%
  \interp{\stencil{c}{$i$}{$k$}{\textcap{p}}}_N & =
    \vecgen{N}{i}{\interv{-k}{k}{\interp{\textcap{p}}}} \\
%
  \interp{\stencil{f}{$i$}{$k$}{\textcap{p}}}_N & =
    \vecgen{N}{i}{\interv{0}{k}{\interp{\textcap{p}}}} \\
%
  \interp{\stencil{b}{$i$}{$k$}{\textcap{p}}}_N & =
  \vecgen{N}{i}{\interv{-k}{0}{\interp{\textcap{p}}}}
\\
  \interp{\texttt{\textcap{r} + \textcap{s}}}_N & =
    \interp{\textcap{r}}_N \vee \interp{\textcap{s}}_N
\\
  \interp{\texttt{\textcap{r} * \textcap{s}}}_N & =
    \interp{\textcap{r}}_N \wedge \interp{\textcap{s}}_N \\[-1em]
\end{align*}
\caption{Interpretation of regions}
\label{subfig:region-model}
\end{subfigure}
\hspace{1em}
\begin{subfigure}[t]{0.4\textwidth}
\begin{align*}
\interpApprox{-} & : \textit{approx} \rightarrow (A \rightarrow
  \textsf{Approx} \, A) \\
\interpApprox{\texttt{atLeast}} & = \mathsf{lower} \\
  \interpApprox{\texttt{atMost}} & = \mathsf{upper} \\
  \interpApprox{\epsilon} & = \mathsf{exact} \\ \\
  \interpMult{-} & : \textit{mult} \rightarrow (A \rightarrow
  \textsf{Mult} \, A) \\
  \interpMult{\texttt{readOnce}} & = \mathsf{once} \\
  \interpMult{\epsilon} & = \mathsf{mult}
\end{align*}
\caption{Interpretation of modifiers}
\label{subfig:modifier-model}
\end{subfigure}
\label{fig:semantics}
\caption{Semantic model of specifications}
\end{figure}


The first four equations of \Cref{subfig:region-model} model region 
contants. The final two equations model the \term{+} and \term{*}
operators in terms of the join (union) and meet (intersection)
of interval schemes. Thus regions are modelled as members
of $\region{N}$.

We mark in our model the presence of modifiers such as
\texttt{readOnce} and \texttt{atMost} as introduced in \Cref{}.
Approximation modifiers are interpreted as injections into the
$\mathsf{Approx}$ variant by $\interpApprox{}$ in
\Cref{subfig:modifier-model}.  The $\textsf{Approx}$ type corresponds
to the presence or absence of the spatial approximation modifier, with
\textsf{exact} when there is no such modifier and \textsf{lower} and
\textsf{upper} for \term{atLeast} and \term{atMost}. In a similar way,
multiplicity modifiers are interpreted as injections in the
$\mathsf{Mult}$ variant by $\interpMult{}$, corresponding to the to
the presence or absence of the \term{readOnce} modifier as shown in
\Cref{subfig:modifier-model}.

\begin{defn}[Semantics of specifications]
The intermediate interpretations of \Cref{fig:semantics}
are composed to give a model for the top-level specification
syntax as:
%
\begin{equation*}
  \interp{\texttt{stencil \textcap{mult}, \textcap{approx}, \textcap{region}}}_N =
    \interpMult{\textcap{mult}} \;
           {(\interpApprox{\textcap{approx}} \;
                    {\interp{\textcap{region}}_N)}}
\end{equation*}
\end{defn}
%
\begin{thm}
The semantic model $\interp{-}_N$ of $N$-dimensional specifications
is sound with respects to the equational theory of the language,
that is:
%
\begin{equation*}
\forall S, T, N . \quad
S \equiv T \; \Rightarrow \;
\interp{S}_N = \interp{T}_N
\end{equation*}
%
We define the equational theory in \Cref{sec:equational-theory}.
\end{thm}

\subsection{Denotational semantics for array subscripts}
\label{sec:fromcode}

\begin{defn}
  Recall array subscript terms of the form $a(\bar{e})$ from
  \Cref{def:array-subs}. We interpret these terms with the partial
  interpretation $\interp{-}^{\mathit{aterm}} : \textit{array-term}
  \prightarrow{} \mathcal{P}(\bz{})^N$. The interpretation is defined when
  all indices are either constant or neighbourhood indices as defined in
  \Cref{def:neighbour-ix}.
%
  \begin{align*}
    \interp{a(\bar{e})}^{\mathit{aterm}} =
      \prod_{1 \leq i \leq N} \mathit{subscript}(\bar{e}_i) & &
  %
    \textit{subscript}(e) = \begin{cases}
      \{ c \} & e \equiv i \pm c \\
      \bz & e \; \mbox{is constant}
    \end{cases}
  \end{align*}
\end{defn}

%%% Local Variables:
%%% mode: latex
%%% TeX-master: t
%%% End:

%\textit{$a^m \in A$ means there are $n$ copies of $a$ in
%  the multi-set $A$}
\caption{Multi-set model of specifications}
\label{fig:model}
\end{figure}

\subsection{Equational theory and approximations}
\label{sec:eqs}

Figure~\ref{fig:equations}

\begin{figure}
\begin{align*}
\hspace{-1em}
\setlength{\arraycolsep}{0.05em}
\begin{array}{c}
\framebox{$\overline{\texttt{v} : S} \equiv \overline{\texttt{u} : T}$} \\[1em]
\begin{array}{rl}
(\textsc{coalesceF}) \;\; & (\texttt{a} : \texttt{fwd} \;
                       \texttt{dims}=ds \; \texttt{depth}=n; \\
&  \;\, \texttt{a} :  \texttt{fwd} \; \texttt{dims}=ds \; \texttt{depth}=n+1)\\
\equiv \; & \texttt{a} :  \texttt{fwd} \; \texttt{depth}=n+1,
         \texttt{dims}=ds \\[0.75em]
(\textsc{coalesceB}) \;\; & (\texttt{a} : \texttt{bwd} \;
                       \texttt{dims}=ds \; \texttt{depth}=n; \\
&  \;\, \texttt{a} :  \texttt{bwd} \; \texttt{dims}=ds \; \texttt{depth}=n+1) \\
\equiv \; & \texttt{a} :  \texttt{bwd} \; \texttt{depth}=n+1,
         \texttt{dims}=ds \\[0.75em]
(\textsc{idem}) \;\; & (\texttt{a} : S; \texttt{a} : S) \equiv (\texttt{a}
                  : S)
\end{array} \\ \\
\framebox{$\texttt{v} : S \equiv \texttt{v'} : S'$} \\[1em]
\begin{array}{ll}
(\texttt{a} : \; \texttt{bwd} \; \texttt{depth=1\{b\}} \;
  \texttt{dim=t})
\, & \equiv \,
(\texttt{b} : \; \texttt{bwd} \; \texttt{depth=1\{a\}} \;
  \texttt{dim=t})
\\
(\texttt{a} : \; \texttt{fwd} \; \texttt{depth=1\{b\}}\;
  \texttt{dim=t})
\, & \equiv \,
(\texttt{b} : \; \texttt{fwd} \; \texttt{depth=1\{a\}} \;
  \texttt{dim=t})
\end{array}
\\ \\
\framebox{$S \equiv S'$} \\[0.75em]
\begin{array}{ll}
(a : \texttt{read-once} \; S) \; & \equiv \; (a : \texttt{read-once} \;\,
  \texttt{read-once} \; \, S)
\end{array}
\end{array}
\end{align*}
\caption{Equations on specifications}
\label{fig:equations}
\end{figure}


\begin{theorem}[Soundness]
\[
\overline{\texttt{v} : S}\equiv \overline{\texttt{u} : T}
\; \Rightarrow \;
\interp{\overline{\texttt{v} : S}} = \interp{\overline{\texttt{u} : T}}
\]
\end{theorem}

\paragraph{Proof} (see Appendix)


\subsubsection{Safety and sub-specifications}

Figure~\ref{fig:inequations}

\begin{figure}[t]
{\framebox{
\begin{minipage}{0.88\linewidth}
\begin{align*}
\hspace{-0.7em}
\begin{array}{c}
\dfrac{}{S <: S}(\textsc{refl}) \qquad \dfrac{R <: S \quad S <: T}{R <:
  T}(\textsc{trans}) \\[1.5em]
\dfrac{}{S <: \; \texttt{unspecified dims=}ds}(\top)
\\[1.5em]
\dfrac{}{\texttt{read-once} \, S <: S}(\textsc{rep})
\\[1.5em]
\setlength{\arraycolsep}{0.1em}
\dfrac{\hspace{3em} ds \subseteq es \; \wedge \; n \leq m \hspace{3em}}
{\begin{array}{rl}
\texttt{fwd} \; \texttt{depth=}n \; \texttt{dim=}ds & <: \texttt{fwd} \;
  \texttt{depth=}m \; \texttt{dim=}es
%& \;\, ds \subseteq es \wedge n \leq m
\\
\wedge \; \texttt{bwd} \; \texttt{depth=}n \; \texttt{dim=}ds & <: \texttt{bwd} \;
  \texttt{depth=}m \; \texttt{dim=}es
%& \;\, ds \subseteq es \wedge n \leq m
\\
\wedge \; \texttt{fwd} \; \texttt{depth=}n \; \texttt{dim=}ds & <: \texttt{sym} \;
  \texttt{depth=}m \; \texttt{dim=}es
%& \;\, ds \subseteq es \wedge n \leq m
\\
\wedge \; \texttt{bwd} \; \texttt{depth=}n \; \texttt{dim=}ds & <: \texttt{sym} \;
  \texttt{depth=}m \; \texttt{dim=}es
\end{array}}
%{\footnotesize{(\textsc{cover})}}
%\hspace{-0.1em}
\end{array}
\end{align*}
\end{minipage}}}
\caption{Definition of sub-specification relation}
\label{fig:inequations}
\end{figure}


\begin{theorem}[Soundness]
\[
\overline{\texttt{v} : S} <: \overline{\texttt{u} : T}
\; \Rightarrow \;
\interp{\overline{\texttt{v} : S}} \subseteq \interp{\overline{\texttt{u} : T}}
\]
\end{theorem}

\paragraph{Proof} (see Appendix)



\section{Examples and common patterns}

\begin{example}[FTCS (Forward time, centered space)]

  Otherwise known as the \emph{explicit method}, is a common
  pattern since it tends to be more efficient than other approches
  (see BTCS and CTCS below). However, its disadvantage is that it is
  unstable if the resolution is not well chosen.

\begin{ExmVerbatim}
b(x) = a(x) + r*(a(x-1) - 2*a(x) + a(x+1))
\end{ExmVerbatim}
%
For this statement, the inference provides the specification:
%
\begin{SpecVerbatim}
%=  stencil centered, depth=1, dim=0 :: a
\end{SpecVerbatim}



\end{example}

\paragraph{BTCS (Backward time, centered space) -- implicit method}

\paragraph{CTCS (Centered time, centered space) - Crank-Nicolson method}

\section{Specification inference and checking}
\label{sec:analysis}

\mnote{Few cases that we might like to consider (there are examples
    of each in \#camfort-main Slack channel):
  \begin{itemize}
    \item Matrix operator overloading $a + b$, where both are arrays
    \item Scalar operator overloading $a + b$, where a is an array b is a
      scalar. $b$ is added to all.
    \item Sliding. $a = b$ where the dimensionsare $a(1:10)$, $b(2:11)$,
      so the assignment causes shifting.
  \end{itemize}
}

In the following, we defined
\emph{base induction variables} to be the variables
 defined by a ``for'' loop (\eg{}, that get incremented each time the
loop is re-entered).

\begin{definition}[Constant translations]
An indexing expression $e$ is a \emph{constant translation} if,
for a base induction variable $i$, then $e \equiv i + a$ or $e \equiv i - a$
where $a$ is a constant. The relation $\equiv$ identifies terms
up-to commutativity of $+$ and the inverse
relation of $+$ and $-$ (\eg{}, $(-b) + i \equiv i - b$).
In the following, we classify constant translation expressions
using the predicate $\textsf{trans}$.
\end{definition}

\paragraph{Step 1: Data-access analysis}

\dnote{Might want to do 'stencil' (gather) and 'mould' (scatter)
  patterns separately}
\begin{align*}
\begin{array}{lll}
\interp{e_1 = e_2}    & = \interp{e_2} \\
\interp{e_1 e_2}      & = \interp{e_1} \cup \interp{e_2} \\
\interp{a(\tilde{e})} & = \{e | e \in \tilde{e}, \textsf{trans}(e)\} & \textit{where each $e \in
                                          \tilde{e}$ is a constant or constant translation}
\end{array}
\end{align*}

Index analysis is a \emph{coeffect analysis}.

\paragraph{Step 2: Coalesce contiguous indices into regions}

% fairly mechanical

\subsection{Infering temporal specifications}

The reserved dimension \dimId{t}

\section{Evaluation}
\label{sec:evaluation}

\section{Discussion}
\label{sec:discussion}


\bibliography{references}


\onecolumn
\appendix

\section{Correctness}


\begin{theorem}[Soundness]
\[
\overline{\texttt{v} : S}\equiv \overline{\texttt{u} : T}
\; \Rightarrow \;
\interp{\overline{\texttt{v} : S}} = \interp{\overline{\texttt{u} : T}}
\]
\end{theorem}

\paragraph{Proof}
\begin{itemize}
\item \textsc{coalesceF}
\begin{align*}
\llbracket & \texttt{v} : \texttt{fwd} \; \texttt{dims}=ds \; \texttt{depth}=n; \,
  \texttt{v} :  \texttt{fwd} \; \texttt{dims}=ds \;
  \texttt{depth}=n+1)\rrbracket \\
& =  \{ (\texttt{v}, ix)^m \mid \forall \, m, ix : \relix, d : \mathbb{N}
  \, . \, d \in \textit{ds} \Rightarrow ix \, d \leq (n+1) \} \; \cup \; \{ ix^m \mid \forall \, m, ix : \mathbb{Z}^{\mathbb{N}}, d : \mathbb{N}
  \, . \, d \in \textit{ds} \Rightarrow ix \, d \leq n \}\\
& =  \{ (\texttt{v}, ix)^m \mid \forall \, m, ix : \relix, d : \mathbb{N}
  \, . \, d \in \textit{ds} \Rightarrow ((ix \, d \leq (n+1)) \vee (ix \,
  d \leq n)) \} \\
& =  \{ (\texttt{v}, ix)^m \mid \forall \, m, ix : \relix, d : \mathbb{N}
  \, . \, d \in \textit{ds} \Rightarrow ix \, d \leq (n+1) \} \\
& = \interp{\texttt{v} : \texttt{fwd} \; \texttt{dims}=ds \;
  \texttt{depth}=n + 1} \qquad \Box
\end{align*}
\end{itemize}

\end{document}
