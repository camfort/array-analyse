\newcommand{\off}{S}
\newcommand{\offP}{T}

\vectorIntersect*

\begin{proof}
  \begin{align*}
    \off \; \cap \; \offP &
    = \{x \mid \bigwedge_{1 \leq i \leq N } x_i \in \pi_i(\off) \}
      \;\cap\;
      \{x \mid \bigwedge_{1 \leq i \leq N } x_i \in \pi_i(\offP) \} \\
    & = \{x \mid \bigwedge_{1 \leq i \leq N }
      (x_i \in \pi_i(\off) \wedge x_i \in \pi_i(\offP)) \} \\
    & = \prod_{i = 1}^{N} \pi_i(\off) \cap \pi_i(\offP)
  \end{align*}
\end{proof}

\vectorUnion*

\begin{proof}
  \begin{align*}
    \off \; \cup \; \offP &
    = \{x \mid
      \bigwedge_{1 \leq i \leq N } x_i \in \pi_i(\off) \}
      \;\cup\;
      \{x \mid
          \bigwedge_{1 \leq i \leq N } x_i \in \pi_i(\offP) \} \\
    & = \{x \mid
          \bigwedge_{\substack{1 \leq i \leq N \\ i \neq k}}
            x_i \in \pi_i(\off) \wedge x_k \in \pi_k(\off) \vee
          \bigwedge_{\substack{1 \leq i \leq N \\ i \neq k}} x_i \in
            \pi_i(\offP) \wedge x_k \in \pi_k(\offP)
        \} \\
    & = \{x \mid
          \bigwedge_{\substack{1 \leq i \leq N \\ i \neq k}} x_i \in
            \pi_i(\off) \wedge
            x_k \in \pi_k(\off) \cup \pi_k(\offP)
        \} \\
        & = \pi_1(\off) \times \cdots \times
        (\pi_k(\off) \cup \pi_k(\offP)) \times \cdots \times
        \pi_N(\off)
  \end{align*}
\end{proof}

\intervalIdentities*

\begin{proof}
  We give the proof of the first identity and the second one is similar.
  \begin{align*}
    \interv{a}{b}{c} \cap \interv{d}{e}{f} = &
      \; \{ n \;|\; a \leq n \leq b \wedge (\neg c \implies n \neq 0) \}
      \;\cap \\
      & \; \{ n \;|\; d \leq n \leq e \wedge (\neg f \implies n \neq 0) \}
      \\
    = & \; \{ n \;|\; \max \{a,d\} \leq n \leq \min \{b,e\} \wedge (\neg c
      \vee \neg f \implies n \neq 0) \} \\
    = & \; \interv{\max{\{a,d\}}}{\min{\{b,e\}}}{c \wedge f}
  \end{align*}
\end{proof}

\centeredApprox*

\begin{proof}
  Proof of the first inequality:
  \begin{align*}
    \stencil[s]{f}{$n$}{$k$}{\textcap{p}}
      & \preceq
        \stencil[s]{f}{$n$}{$k$}{\textcap{p}}
        \;\texttt{+}\;
        \stencil[s]{b}{$n$}{$k$}{\textcap{p}}
      & (\textbf{Combined}) \\
    & \equiv \stencil[s]{c}{$n$}{$k$}{\textcap{p}} & (\textbf{Centered})
  \end{align*}

  Proof of the second inequality:
  \begin{align*}
    \stencil[s]{b}{$n$}{$k$}{\textcap{p}}
      & \preceq
        \stencil[s]{f}{$n$}{$k$}{\textcap{p}}
        \;\texttt{+}\;
        \stencil[s]{b}{$n$}{$k$}{\textcap{p}}
      & (\textbf{Combined}) \\
    & \equiv \stencil[s]{c}{$n$}{$k$}{\textcap{p}} & (\textbf{Centered})
  \end{align*}
\end{proof}

\pointApprox*

\begin{proof}
  Proof of the first inequality:
  \begin{align*}
    \stencil[s]{p}{$n$}{}{}
      & \preceq \stencil{R}{$n$}{$k$}{nonpointed}
                \;\texttt{+}\;
                \stencil[s]{p}{$n$}{}{}
      & (\textbf{Combined}) \\
    & \equiv \stencil{R}{$n$}{$k$}{} & (\textbf{Overlapping pointed})
  \end{align*}

  Proof of the second inequality:
  \begin{align*}
    \stencil{R}{$n$}{$k$}{nonpointed}
      & \preceq \stencil{R}{$n$}{$k$}{nonpointed}
                \;\texttt{+}\;
                \stencil[s]{p}{$n$}{}{}
      & (\textbf{Combined}) \\
    & \equiv \stencil{R}{$n$}{$k$}{} & (\textbf{Overlapping pointed})
  \end{align*}
\end{proof}

\equationalSoundness*

\begin{proof}
  Both $\equiv$ and $=$ are equivalence relations. We only need to show that
  each property associated with specifications equivalence holds as an equality
  in the model.
%
  \begin{description}
    \item[Basic \& Subsumption] These follow from basic properties of lattices.
%
    \item[Dist] This is immediate from the definition of distributive
      lattice which the model is one.
%
    \item[Overlapping pointed] Counterparts of individual regions for some
      Boolean $p$ are given below:
%
      \begin{align*}
        \interp{\stencil{\textcap{R}}{$n$}{$k$}{\texttt{nonpointed}}}_N
        & = \vecgen{N}{n}{\interv{-k}{k}{\mathsf{false}}} \\
%
        \interp{\stencil{p}{$n$}{}{}}_N
        &= \vecgen{N}{n}{\interv{0}{0}{\mathsf{true}}} \\
%
        \interp{\stencil{\textcap{R}}{$n$}{$k$}{}}_N
        &= \vecgen{N}{n}{\interv{-k}{k}{\mathsf{true}}}
      \end{align*}

      Using \Cref{lem:vector-union}, it is sufficient to show
      $ \interv{-k}{k}{p} \;\cup\; \interv{0}{0}{\mathsf{true}} =
        \interv{-k}{k}{\mathsf{true}} $ holds. This immediately follows from
      \Cref{lem:zinf-identities}.
%
    \item[Centered] Let $\interp{\textcap{p1}}$, $\interp{\textcap{p2}}$, and
      $\interp{\textcap{p3}}$ be $p1$, $p2$, and $p3$ respectively. We
      know that $p1$ is $p2 \vee p3$ from (\textbf{Centered}) detinition. Then
      we get the following interpretation for region constants involved in the
      equivalence:
%
      \begin{align*}
        \interp{\stencil{c}{$k$}{$n$}{$p1$}}_N
          &= \vecgen{N}{n}{\interv{-k}{k}{p2 \vee p3}} \\
%
        \interp{\stencil{f}{$k$}{$n$}{$p2$}}_N
          &= \vecgen{N}{n}{\interv{0}{k}{p2} }\\
%
        \interp{\stencil{b}{$k$}{$n$}{$p3$}}_N
          &= \vecgen{N}{n}{\interv{0}{k}{p3}}
      \end{align*}
%
      Using \Cref{lem:vector-union}, it is sufficient to show $\interv{-k}{k}{p2
      \vee p3} = \interv{0}{k}{p2} \cup \interv{-k}{0}{p3}$ holds. This
      immediately follows from \Cref{lem:zinf-identities}.
  \end{description}
\end{proof}

\approxSoundness*

\begin{proof}
  Both $\preceq$ and $\subseteq$ are partial orders on sets. We only need to
  show that each inequality holds in the model.
%
  \begin{description}
    \item[Equivalence] Region equavilance is modeled with set equality and the
      partial order on regions is modeled with subset relation. Partial orders
      are reflexive and the model of eqivalence agrees with that of the partial
      order, so this rule holds.
%
    \item[Combined] \texttt{*} and \texttt{+} maps to meet and join operations
      in the lattice. All meets and joins in a lattice satisfy these
      inequalities.

    \item[Depth] When $m$ is $k$ and $l$ in
      $\stencil{\textcap{R}}{$n$}{$m$}{\textcap{p}}$, we obtain the
      interpretations $\vecgen{N}{n}{\textit{int}_k}$ and
      $\vecgen{N}{n}{\textit{int}_l}$ for some $N$ as interpretations of
      regions. Call these \textit{ks} and \textit{ls} respectively. One way to
      show $\textit{ks} \subseteq \textit{ls}$ is to show that the intersection
      of \textit{ks} and \textit{ls} is \textit{ks}.

      To show that the intersection is \textit{ks}, we use
      \Cref{lem:vector-intersect} which reduces the problem to showing that the
      intersection of $\textit{int}_k$ and $\textit{int}_l$ is the former since
      at every point except $n$ the intervals agree, hence the intersection
      remains the same at these dimensions. Then we use
      \Cref{lem:zinf-identities} to calculate the intersection. By assumption we
      have $k \leq l$, so the maximum of the lower bounds is $-k$ and the
      minimum of lower bounds is $k$ this is same as interval $\textit{int}_k$.
      Note that pointedness is the same for both intervals and depths are
      positive integers, so the pointedness is preserved under intersection.
      This shows the intersection of \textit{ls} and \textit{ks} is indeed
      \textit{ks} and concludes the proof.
  \end{description}
\end{proof}
