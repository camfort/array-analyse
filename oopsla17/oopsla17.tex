\documentclass[acmlarge,review,anonymous,10pt]{acmart}\settopmatter{printfolios=true}

%% Some recommended packages.
\usepackage{booktabs}   %% For formal tables:
                        %% http://ctan.org/pkg/booktabs
\usepackage{subcaption} %% For complex figures with subfigures/subcaptions
                        %% http://ctan.org/pkg/subcaption

\usepackage{hyperref}
\usepackage{natbib}
\usepackage{amsmath}
\usepackage{amssymb}
\usepackage{amsthm}
\usepackage{syntax}
\usepackage{indentfirst}
\usepackage{cleveref}
\usepackage{xcolor}
\usepackage{multirow}
\usepackage{siunitx} % For pretty-printing numeric values and SI units
                     % of measure. e.g., the tabular column type S is
                     % used to print nice-looking tables of numbers.
\sisetup{ % defaults
  group-separator={,},
  group-minimum-digits={3},
  output-decimal-marker={.},
  table-format = 6
}

\usepackage{minted}
\usepackage{etoolbox}
\usepackage{enumitem}
\usepackage{thmtools}
\usepackage{thm-restate}

\newcommand{\fortran}[1]{\mintinline{fortran}{#1}}

\errorcontextlines 10000
\usepackage{stencilmacros}

%% Bibliography style
\bibliographystyle{ACM-Reference-Format}
%% Citation style
%% Note: author/year citations are required for papers published as an
%% issue of PACMPL.
\citestyle{acmauthoryear}   %% For author/year citations


\fvset{
  linenos=true,
  fontsize=\footnotesize,
  breaklines=true,
  breakafter=w),
  xleftmargin=3\parindent,
  numbersep=1em
}

\usemintedstyle{vs}

\setlength{\grammarindent}{3em} % increase separation between LHS/RHS

\newcommand{\inbar}{\ \ \textbar\ \ }

% For evluation and empirical study
% Defines the commands (headers)
\newcommand{\numPackages}{11}
\newcommand{\SSArpackFiles}{312}
\newcommand{\SSArpackLoC}{50208}
\newcommand{\SSArpackLoCRaw}{144081}

\newcommand{\SSArpackFilesParsed}{290}
\newcommand{\SSArpackLoCParsed}{47453}
\newcommand{\SSArpackLoCParsedRaw}{139290}

%%

\newcommand{\SSBlasFiles}{151}
\newcommand{\SSBlasLoC}{16046}
\newcommand{\SSBlasLoCRaw}{40882}

\newcommand{\SSBlasFilesParsed}{149}
\newcommand{\SSBlasLoCParsed}{15993}
\newcommand{\SSBlasLoCParsedRaw}{40679}

%%

\newcommand{\SSCliffsFiles}{30}
\newcommand{\SSCliffsLoC}{2424}
\newcommand{\SSCliffsLoCRaw}{3149}

\newcommand{\SSCliffsFilesParsed}{30}
\newcommand{\SSCliffsLoCParsed}{2424}
\newcommand{\SSCliffsLoCParsedRaw}{3149}

%%

\newcommand{\SSCPFiles}{52}
\newcommand{\SSCPLoC}{2334}
\newcommand{\SSCPLoCRaw}{3978}

\newcommand{\SSCPFilesParsed}{48}
\newcommand{\SSCPLoCParsed}{2121}
\newcommand{\SSCPLoCParsedRaw}{3632}

%%

\newcommand{\SSEThreeMEFiles}{167}
\newcommand{\SSEThreeMELoC}{44935}
\newcommand{\SSEThreeMELoCRaw}{73545}

\newcommand{\SSEThreeMEFilesParsed}{154}
\newcommand{\SSEThreeMELoCParsed}{39700}
\newcommand{\SSEThreeMELoCParsedRaw}{62238}

%%

%OLDER
%\newcommand{\SSgeosChemFiles}{604}
%\newcommand{\SSgeosChemLoC}{467472}
%\newcommand{\SSgeosChemLoCRaw}{848647}

%\newcommand{\SSgeosChemFilesParsed}{196}
%\newcommand{\SSgeosChemLoCParsed}{213387}
%\newcommand{\SSgeosChemLoCParsedRaw}{263694}


\newcommand{\SSgeosChemFiles}{604}
\newcommand{\SSgeosChemLoC}{445660}
\newcommand{\SSgeosChemLoCRaw}{848647}

\newcommand{\SSgeosChemFilesParsed}{336}
\newcommand{\SSgeosChemLoCParsed}{269674}
\newcommand{\SSgeosChemLoCParsedRaw}{410654}


%%

\newcommand{\SShybridFiles}{29}
\newcommand{\SShybridLoC}{4831}
\newcommand{\SShybridLoCRaw}{8361}

\newcommand{\SShybridFilesParsed}{29}
\newcommand{\SShybridLoCParsed}{4831}
\newcommand{\SShybridLoCParsedRaw}{8361}

%%

\newcommand{\SSmudpackFiles}{88}
\newcommand{\SSmudpackLoC}{54753}
\newcommand{\SSmudpackLoCRaw}{78652}

\newcommand{\SSmudpackFilesParsed}{88}
\newcommand{\SSmudpackLoCParsed}{54753}
\newcommand{\SSmudpackLoCParsedRaw}{78652}

%%

\newcommand{\SSnavierFiles}{6}
\newcommand{\SSnavierLoC}{505}
\newcommand{\SSnavierLoCRaw}{696}

\newcommand{\SSnavierFilesParsed}{6}
\newcommand{\SSnavierLoCParsed}{505}
\newcommand{\SSnavierLoCParsedRaw}{696}

%%

\newcommand{\SSspecfemFiles}{555}
\newcommand{\SSspecfemLoC}{137468}
\newcommand{\SSspecfemLoCRaw}{232356}

\newcommand{\SSspecfemFilesParsed}{475}
\newcommand{\SSspecfemLoCParsed}{103328}
\newcommand{\SSspecfemLoCParsedRaw}{178317}


%%

\newcommand{\SSumFiles}{2540}
\newcommand{\SSumLoC}{635525}
\newcommand{\SSumLoCRaw}{1010936}

\newcommand{\SSumFilesParsed}{2269}
\newcommand{\SSumLoCParsed}{541540}
\newcommand{\SSumLoCParsedRaw}{866406}

%% Totals

\newcommand{\SSFiles}{\the\numexpr(\SSArpackFiles+\SSBlasFiles+\SSCliffsFiles+\SSCPFiles+\SSEThreeMEFiles+\SSgeosChemFiles+\SShybridFiles+\SSnavierFiles+
\SSmudpackFiles +\SSspecfemFiles+\SSumFiles)}

\newcommand{\SSLoC}{\the\numexpr(\SSArpackLoC+\SSBlasLoC+\SSCliffsLoC+\SSCPLoC+\SSEThreeMELoC+\SSgeosChemLoC+\SShybridLoC+\SSmudpackLoC+\SSnavierLoC+\SSspecfemLoC+\SSumLoC)}

\newcommand{\SSLoCRaw}{\the\numexpr(\SSArpackLoCRaw+\SSBlasLoCRaw+\SSCliffsLoCRaw+\SSCPLoCRaw+\SSEThreeMELoCRaw+\SSgeosChemLoCRaw+\SShybridLoCRaw+\SSmudpackLoCRaw+\SSnavierLoCRaw+\SSspecfemLoCRaw+\SSumLoCRaw)}

\newcommand{\SSFilesParsed}{\the\numexpr(\SSArpackFilesParsed+\SSBlasFilesParsed+\SSCliffsFilesParsed+\SSCPFilesParsed+\SSEThreeMEFilesParsed+\SSgeosChemFilesParsed+\SShybridFilesParsed+\SSnavierFilesParsed+\SSmudpackFilesParsed+\SSspecfemFilesParsed+\SSumFilesParsed)}

\newcommand{\SSLoCParsed}{\the\numexpr(\SSArpackLoCParsed+\SSBlasLoCParsed+\SSCliffsLoCParsed+\SSCPLoCParsed+\SSEThreeMELoCParsed+\SSgeosChemLoCParsed+\SShybridLoCParsed+\SSnavierLoCParsed+\SSmudpackLoCParsed+\SSspecfemLoCParsed+\SSumLoCParsed)}

\newcommand{\SSLoCParsedRaw}{\the\numexpr(\SSArpackLoCParsedRaw+\SSBlasLoCParsedRaw+\SSCliffsLoCParsedRaw+\SSCPLoCParsedRaw+\SSEThreeMELoCParsedRaw+\SSgeosChemLoCParsedRaw+\SShybridLoCParsedRaw+\SSnavierLoCParsedRaw+\SSmudpackLoCParsedRaw+\SSspecfemLoCParsedRaw+\SSumLoCParsedRaw)}

% renews the commands with actual data
% \input{results}

\definecolor{darkgreen}{rgb}{0.0,0.5,0.0}
\definecolor{darkpurple}{rgb}{0.6,0.0,0.6}
\definecolor{orange}{rgb}{0.8,0.4,0.0}
\newcommand{\dnote}[1]{\textcolor{darkpurple}{Dom: #1}}
\newcommand{\mnote}[1]{\textcolor{darkgreen}{Mistral: #1}}
\newcommand{\anote}[1]{\textcolor{red}{Andy: #1}}

\theoremstyle{definition}
\newtheorem{defn}{Definition}

\theoremstyle{plain}
\newtheorem{thm}{Theorem}
\newtheorem{lem}{Lemma}
\newtheorem{prop}{Proposition}
\newtheorem{case}{Case}

\theoremstyle{remark}
\newtheorem{remark}{Remark}

\title{Verifying Spatial Properties of Array Computations}

\author{Dominic Orchard}
\affiliation{
  \department{School of Computing}
  \institution{University of Kent}
}
\email{d.a.orchard@kent.ac.uk}


\author{Mistral Contrastin}
\affiliation{
  \department{Computer Laboratory}
  \institution{University of Cambridge}
}
\email{Mistral.Contrastin@cl.cam.ac.uk}

\author{Matthew Danish}
\affiliation{
  \department{Computer Laboratory}
  \institution{University of Cambridge}
}
\email{Matthew.Danish@cl.cam.ac.uk}


\author{Andrew Rice}
\affiliation{
  \department{Computer Laboratory}
  \institution{University of Cambridge}
}
\email{Andrew.Rice@cl.cam.ac.uk}


\begin{abstract}
  Arrays computations are at the core of numerical modelling and
  computational science applications. However, low-level manipulation of array
  indices is a source of program error.  Many practitioners are aware
  of the need to ensure program correctness, yet very few of the
  techniques from the programming research community are applied by
  scientists. We aim to change that by providing targetted lightweight
  verification techniques for scientific code.  We focus on the all
  too common mistake of array offset errors as a generalisation of
  off-by-one errors.  Firstly, we report on a code analysis study on
  twelve real-world computational science code base, identifying
  common idioms of array usage and their spatial properties. This
  provides much needed data on array programming idioms common in
  scientific code.  From this data, we designed a lightweight
  declarative specification language capturing the majority of array
  access patterns via a small set of combinators. We detail a semantic
  model, and the design and implementation of a verification tool for
  our specification language, which both checks and infers
  specifications.  We evaluate our tool on our corpus of scientific
  code and give verification case studies of bug fixes that are
  detected by our approach. We found roughly 60,000 targets for
  specification across roughly 1.4 million lines of code, showing the
  vast majority of array computations read from arrays in a pattern with
  a simple, regular, static shape.
\end{abstract}
  
%  We focus on one error-prone aspect of numerical code: the 
%  data access pattern of arrays. 

\begin{document}
\maketitle

\input{results}

\section{Introduction}
\label{sec:introduction}
s\noindent
In the sciences, complex models are now almost always expressed as
computer programs. But how can a scientist have confidence that the
implementation of their model is as they intended? There is an
increasing awareness of the need for program verification in science
and the possibility of using (semi-)automated
tools~\cite{post2005computational,oberkampf2010verification,orchard2014computational}.
However, whilst program verification approaches are slowly maturing in
computer science they see little use in the natural and physical
sciences. This is partly due to a lack of training and awareness, but
also a lack of tools targeted at the needs of scientists. This
paper is part of a line of research on providing lightweight, easy-to-ue
verification tools targeted at common programming patterns in science,
motivated by analysis of real code.

We focus on one common concept: \emph{arrays}, the core data structure
in numerical modelling code, typically representing discrete
approximations of physical space or data sets. A common programming
pattern, sometimes referred to as the \emph{structured grid}
pattern~\cite{Asanovic2006}, traverses the index space of one or more
arrays via a loop, computing elements of another array or reducing the
elements to a single value. For example, the following Fortran code
computes the one-dimensional discrete Laplace transform approximating
a derivative:
%
\begin{minted}{fortran}
do i = 1, (n-1)
      b(i) = a(i-1) - 2*a(i) + a(i+1)
end do
\end{minted}
%
This is a \emph{stencil} computation, a programming idiom
where the elements of an array at each index $i$ from a
\emph{neighbourhood} of values around $i$ in some input array(s). This
pattern is common in scientific, graphical, and numerical
code, \eg{}, convolutions in image processing, approximations
to differential equations, cellular automata.

Such array computations are prone to error in their indexing
terms. For example, a logical off-by-one-error might manifest itself
as writing $\texttt{a(i)}$ instead of $\texttt{a(i-1)}$ (we revisit an
example we found of this in \Cref{sec:case-studies}).
Errors also arise by simple lexical mistakes when large amounts of
fine-grained indexing are involved in a single expression. For
example, the following snippet from a Navier-Stokes
fluid model~\citet{griebel1997numerical} has two arrays are
read with different data access patterns, across two dimensions, with
dense index-manipulation code:
%%
\begin{minted}[firstnumber=20,xleftmargin=2em]{fortran}
du2dx = ((u(i,j)+u(i+1,j))*(u(i,j)+u(i+1,j))+gamma*abs(u(i,j)+u(i+1,j))*(u(i,j)-u(i+1,j))- &
  (u(i-1,j)+u(i,j))*(u(i-1,j)+u(i,j))-gamma*abs(u(i-1,j)+u(i,j))*(u(i-1,j)-u(i,j))) /(4.0*delx)

duvdy = ((v(i,j)+v(i+1,j))*(u(i,j)+u(i,j+1))+gamma*abs(v(i,j)+v(i+1,j))*(u(i,j)-u(i,j+1))- &
  (v(i,j-1)+v(i+1,j-1))*(u(i,j-1)+u(i,j))-gamma*abs(v(i,j-1)+v(i+1,j-1))*(u(i,j-1)- u(i,j))) / (4.0*dely)

laplu = (u(i+1,j)-2.0*u(i,j)+u(i-1,j))/delx/delx+(u(i,j+1)-2.0*u(i,j)+u(i,j-1))/dely/dely
f(i,j) = u(i,j)+del_t*(laplu/Re-du2dx-duvdy)
\end{minted}
%
This miasma of indexing expressions is hard to read and
prone to simple textual input mistakes, \eg{}, swapping \texttt{-} and
\texttt{+}, missing an indexing term, or transforming the wrong
variable \eg{} \texttt{(i+1,j)} instead of \texttt{(i,j+1)}.

In practice, the typical development procedure for complex stencil
computations involves some ad hoc testing to ensure that no
mistakes have been made \eg{}, by visual inspections on data,
or comparison against manufactured or analytical
solutions~\cite{farrell2010automated}. Such testing is often
discarded once the code is shown correct. This is not the only
information that is discarded. The shape of the indexing pattern
is usually the result of choices made in the numerical-analysis
procedure to discretise some continuous equations. Rarely are these decisions captured in the source code,
yet the derived shape is usually uniform with a
clear and concise description \eg{},
\emph{centered space, of depth 1} referring to indexing
terms \texttt{a(i)}, \texttt{a(i-1)} and
\texttt{a(i+1)}~\cite{recktenwald2004finite}.

To support correct array programming, we propose a simple,
abstract specification language for the data access pattern of
array-loop computations. This provides a way to rule out bugs due to
indexing errors. The language design is informed by an initial
empirical study of array computations in a corpus real-world
scientific code base, totally 1.4 million lines of
code (\Cref{sec:study}). We confirm our initial hypotheses of the
ubiquity of looped array computations, but also that they have a
common form, reading from arrays in a fixed neighbourhood of
contiguous elements with a simple static pattern. From this we
designed a simple set of combinators to express common array patterns (\Cref{sec:lang}).
As an example, the Laplace program is specified in our language by:
%
\begin{minted}[linenos=false]{fortran}
!=  stencil centered(depth=1, dim=1) :: a
\end{minted}
%
That is, \texttt{a} is accessed with a symmetrical pattern in its
first dimension (``centered'') to a depth of one in each direction and
its result contributes to an array
write.  The Navier-Stokes example has two specifications:
%
\begin{minted}[linenos=false,xleftmargin=0em,xrightmargin=0.8em,breakindent=1.9em]{fortran}
!= stencil centered(depth=1,dim=1)*pointed(dim=2) + centered(depth=1,dim=2)*pointed(dim=1) :: u
!= stencil forward(depth=1,dim=1)*backward(depth=1,dim=2) :: v
\end{minted}
%
The specifications requires that \texttt{u}
is accessed with a centered pattern to depth of 1 in both dimensions
(this is known as the \emph{five-point stencil}) and \texttt{v} is
accessed in a neighbourhood bounded forwards to depth of $1$ in the
first dimension and backward to a depth of $1$ in the second
dimension. The specification is relatively small and abstract compared
with the code, with a small number of combinators that \emph{do not
  involve any indexing expressions}, \eg{} \texttt{a(i+1,j-1)}. This
contrasts with other specification approaches, \eg{}, deductive
verification tools such as ACSL~\citet{baudin2008acsl}, where
specifications about array computations must also be expressed in
terms of array-indexing expressions. Thus, any low-level mistakes that
could be made whilst programming complex indexing code could also be
made when writing its specification. Our specifications are
abstract to avoid the error-prone nature of index manipulation; our
specifications are more lightweight to aid its adoption in science.

We implemented a verification tool for our specification language as
an extension to CamFort~\cite{camfort}, an open-source program
analysis tool for Fortran. Specifications are associated to code via
comments, against which the tool checks the code for conformance. This
specify-and-check approach reduces testing effort and future-proofs
against bugs introduced during refactoring and maintenance. A
specification also concisely captures the array access pattern
and provides documentation. We also provide an inference procedure
which efficiently produces specifications with no programmer
intervention, automatically inserting specifications at appropriate
places in the source code. This aids adoption of our approach to
existing legacy code base.

The checking and inference algorithms (\Cref{sec:algorithms}) are
derived from a model of array access patterns, which is also used to
give a denotational model of our specification language
(\Cref{sec:semantics}).

Using the inference-mode of the tool, we applied our tool to our
original corpus (\Cref{sec:evaluation}). Our tool identifies and
infers specifications for roughly 60,000 array computations in the
corpus. Approximately 5,000 of the computations we found are non-trivial,
corresponding to code which is a possible source of errors. This
validates the design of our language in its capability to capture many
core patterns.

Our approach does not target a class of bugs that can be detected
automatically (\emph{push-button verification}). Instead, array
indexing bugs must be identified relative to a specification of the
intended access pattern.  Nevertheless, we report on instances of code
revisions from the commit histories of our corpus where a
correct specification would have spotted an error (which was
latter corrected) or would have assisted in refactoring 
(\Cref{sec:case-studies}).

\paragraph{Terminology and notation}

We fix some terminology for syntactic constructs in our target language.

\begin{defn}[Induction variables]
  An integer variable is a \emph{base induction variable} if it is the
  control variable of a ``\texttt{for}'' loop (\texttt{do} in
  Fortran), incremented by $1$ per iteration. A variable is
  interpreted as an induction variable only within the scope of the
  loop body. Throughout, $i, j, k$ range over induction variables.

  A \emph{derived induction variable} is an expression of the form
  $a \ast i + b$, where $a$ and $b$ are constant expressions, \ie{},
  an affine expression on an induction variable $i$.
\end{defn}

\begin{defn}[Array subscripts and indices]\label{def:array-subs}
  An \emph{array subscript}, denoted $a(\bar{e})$, is an expression
  which indicates the element of an $N$-dimensional array $a$ at the
  \emph{index} $\bar{e}$, specified by a comma-separated sequence of
  integer expressions, in expanded form as $(e_1, \ldots, e_N)$. An
  index $e_i$ is called \emph{relative} if the expression involves an
  induction variable. An \emph{absolute index} is a integer expression
  which is constant relative to the enclosing loop.
\end{defn}

\begin{defn}[Origin]\label{def:origin}
  An array subscript $a(\bar{e})$ has an \emph{origin index} if
  all $e \in \bar{e}$ are induction variables, \eg{}, \fortran{a(i,
    j)}.
\end{defn}

\begin{defn}[Neighbourhood index]\label{def:neighbour-ix}
  For an array subscript $a(\bar{e})$ an index $e \in \bar{e}$ is a
  \emph{neighbourhood index} if $e$ is of the form $e \equiv i$, $e \equiv i +
  c$, or $e \equiv i - c$, where $c$ is an integer constant. That is, a
  neighbourhood index is a constant translation of an induction variable. (The
  relation $\equiv$ identifies terms up-to commutativity of $+$ and the inverse
  relationship of $+$ and $-$ \eg{}, $(-b) + i \equiv i - b$).
\end{defn}


%%% Local Variables:
%%% mode: latex
%%% TeX-master: t
%%% End:


\section{An Empirical Study of Array Computations in Scientific Fortran code}
\label{sec:study}
At the beginning of this project, we had the following hypotheses:
\begin{enumerate}
\item Loops over arrays mainly read from arrays with a static pattern
based on constant offsets from (base or dervied) induction variables;

\item Most loop-array computations of the previous form read
from arrays with a \emph{contiguous} pattern, \eg{}:
%
\begin{minted}{fortran}
do i = 1, n
  b(n) = (a(i) + a(i+1)) / 2.0
\end{minted}
%
\item Most loop-array computations of the previous form read
from arrays with a pattern that includes the immediate
neighbours (offsets of 1 from the induction variables);

\item Many array computations are \emph{stencil computations},
with a static access pattern as described in Hypothesis 2, writing
to an array at an index based on a (possibly constant offset from induction
variables, \eg{}, the classic five-point stencil;

\item Many array computations read from each particular index just once.
\end{enumerate}
%
From these hypotheses, we conjectured that the common programming
patterns captured above could be specified declaratively with a small
set of combinators capture the shape of data access patterns based on
overlapping (hyper)rectangles. We performed a large scale source-code
analysis to validate these hypotheses and guide the design of our
language. We believe these results are also of wider value.
\dnote{Explain why: what else do they tell us?}

\subsection{Methodology}
%
\noindent
We constructed a corpus of \numPackages{} scientific computing
packages written in Fortran ranging in size and scope: The Unified
Model (UM)~\cite{um}, E3MG~\cite{RePEc:aen:journl:2006se-a12},
BLAS~\cite{blas}, Hybrid4~\cite{GBC:GBC635},
GEOS-Chem~\cite{geos-chem}, Navier (based on
\cite{griebel1997numerical}), Computational Physics 2 (CP2)
~\cite{nicholas2006computational}, ARPACK-NG~\cite{arpackng}, and
SPECFEM3D~\cite{specfem3d}. \dnote{Include the rest} This covers
approximately 1.4 million lines of physical Fortran code (2.4 million
including comments and white space).  Appendix~\ref{app:corpus}
provides further detail on these packages and their sizes, with UM and
GEOS-Chem the largest at $\approx$630kloc and $\approx$450kloc respectively. We used
Wheeler's \emph{SLOCCount} to get a count of the physical source lines
(excluding comments and blank lines)~\cite{wheeler2001sloccount}.

Half of the packages come from active research teams who are our
project partners (\ie{}, they were not selected carefully to skew
results). BLAS was selected as it is a common numerical library;
similarly CP2 was selected as it gives standard numerical
analysis algorithms from a popular computational physics textbook.
The smallest project Navier ($\approx$500loc) is based on a standard textbook on
computational fluid dynamics.

We built a code analysis tool based on CamFort, an open-source analysis
tool for Fortran programs~\cite{camfort}. Fortran source files are
parsed to an AST and various standard control-flow and data-flow
analyses are computed, including of key importance for us: induction
variable identification, and reaching definitions. The resulting AST
is traversed top-down and assignment statements inside loops are
analysed and classified.

\paragraph{Classifications}

We focus on assignment statements where
the right-hand side reads from one or more arrays. Left-hand
sides and right-hand sides are classified in the following way:

\begin{tabular}{l|l|l}
  Shorthand & Classification (of subscripts) & Example \\ \hline
 (\textsf{vars}) & Variables (just LHS) & \fortran{x = ...}
  \\
 (\textsf{IVs}) & Induction variables (just LHS) & \fortran{a(i, j)} \\
 (\textsf{const}) & Constants & \fortran{a(0, 1)} \\
 (\textsf{neigh}) & Neighbourhood offsets (of the form
                                                  $i \pm c$) & \fortran{a(i, j-1)} \\
 (\textsf{neigh+c}) & Neighbourhood offsets and constants &
                                                            \fortran{b(i, 0, j+1)} \\
 (\textsf{aff}) & Affine offsets (of the form $a * i \pm b$) &
                                     \fortran{a(2*i+1,j)} \\
 (\textsf{aff+c}) & Affine offsets and constants & \fortran{a(i+1, 0,
                                                   3*j+2)} \\
 (\textsf{other}) & Subscript expression not included above &
\fortran{x(f(i))}, \fortran{a(i `mod` 2)}
\end{tabular} \\[1em]

\noindent
Note that the \textsf{vars} and \textsf{IVs} are only give to
left-hand sides.
The classification is computed from all information that flows into an
assignment, for example, the following decomposes a one-dimensional
three-point pattern across multiple intermediate assignments:
%%
\begin{minted}{fortran}
do i = 1, n
  x = a(i)
  y = a(i+1)
  b(i) = (a(i-1) + x + y)/3.0
end do
\end{minted}
%%x
Our analysis recognises this as one array computation (rather than three), starting at
line 4, and reading from subscripts \fortran{a(i)}, \fortran{a(i+1)},
\fortran{a(i-1)}. We thus traverse the body of a loop in reverse,
using reaching-definitions to calculate of the array reads that flow
to a particular assignment. Since \fortran{x} and \fortran{y} flow to
the line 4, line 2 and line 3 are not classified, though multiple
assignments can flow to multiple classified statements.

We further sub-classify sets of array subscripts
on the right-hand side based on their spatial properties: \\

\begin{tabular}{l||l|p{0.28\linewidth}|l}
  Property   & Shorthand & Classifications (of RHS pattern) & Example \\ \hline
%%%%%%
\multirow{2}{*}{Contiguity} & (\textsf{contig}) & Contiguous  &
                                                                \fortran{a(i) + a(i+1) + a(i+2)} \\
  & (\textsf{disjoint}) & Non-contiguous & \fortran{a(i) + a(i+2)} \\ \hline
%%%%%%
\multirow{2}{*}{Shape} & (\textsf{rect}) & (Hyper)rectangle &
 \fortran{a(i,j) + a(i+1,j) + a(i,j+1) + a(i+1,j+1)} \\
           & (\textsf{sumRect}) & Composed (hyper)rectangles &
  \fortran{a(i,j) + a(i,j-1) + a(i-1,j) + a(i+1,j) + a(i,j+1)} \\ \hline
%%%%%%
  \multirow{2}{*}{Reuse} & (\textsf{readonce}) & Unique subscripts
                                         & \eg{} \fortran{b(i) = a(i) + a(i+1)} \\
  & (\textsf{mult}) & Repeated subscripts &
\eg{}  \fortran{b(i) = a(i)  + a(i)}
\end{tabular} \\[1em]

\noindent
To assess the hypothesis that 
arrays are mostly read in a pattern that includes the immediate neighbours
to the ``origin'', we categorised the position of subscript patterns for each
right-hand side, in each dimension: \\

\begin{tabular}{l||l|p{0.33\linewidth}|l}
  Property   & Shorthand & Classification (indices per
                           dimension) & Example \\ \hline
%%%%%%
  \multirow{2}{*}{Positioning} & (\textsf{origin}) & Includes origin
& \fortran{a(i)} or \fortran{a(i+1, j)} (in $2^{\textit{nd}}$
  dimension) \\
             & (\textsf{straddle}) & Within distance 1 of origin &
\fortran{a(i+1), a(i-1)} \\
             & (\textsf{away}) & Away from the origin
                                          & \fortran{a(i+2), a(i+3)}
\end{tabular} \\[1em]

\noindent
Finally, we categorised the relationship between the left-hand side
and right-hand side, in terms of the use of induction variables.
The two sides are \emph{consistent} if the same induction variables
appear in each side, used for the same dimension. This is
weakened to a \emph{permutation} if the roles of the induction
variables changes. This is weakened further if the left-hand side
induction variables are either
a subset or superset of the induction variables on the right-hand
side. Otherwise, the two sides are seen as inconsistent:

\begin{tabular}{l|l|l}
  (\textsf{const}) & Consistent & \fortran{a(i, j) = b(i, j) +
                                  b(i+1,j+1)} \\
  (\textsf{perm} & Permutation & \fortran{a(i, j) = c(j, i)} or \fortran{a(i,
                                                               0) =
                                 b(0, i)} \\
  (\textsf{LHSsub} & LHS subset & \fortran{a(i) = b(i, j) + b(i, j-1)}
  \\
  (\textsf{LHSsup} & LHS superset & \fortran{a(i, j) = b(i)} \\
  (\textsf{inconst} & Inconsistent & \fortran{a(i) = b(j)}
\end{tabular} \\[1em]
%


%\begin{itemize}
% (Vars, ...)
%\item An assignment to a variable, with an RHS comprising some array
%  computation, tends to correspond to a reduction (e.g., calculating
%the max value in an array).
%\end{itemize}

\subsection{Results}

%\begin{enumerate}
%\item Loops over arrays mainly read from arrays with a static pattern
%based on constant offsets from (base or dervied) induction variables;
%\end{enumerate}

%\begin{tabular}
%Neighbourhood or Constant Affine RHS &  Total
%\end{tabular}

\noindent
We identified $108,773$ instances of assignments within loops in
which an array subscript flows to the right-hand side. We refer
to each one of these as an \emph{array computation}.
We performed the classification described in the methodology,
and then grouped the data by based only on the right-hand side
classification:
%%
\begin{center}
\begin{tabular}{lrrcc}
\textbf{RHS classification} & \textbf{Number} & \textbf{\%} &
\textit{grouping 1} & \textit{grouping 2} \\ \hline
Affine                          & 34        & 0.03\%  &
                                                            \multirow{4}{*}{92.61\%}
  \\ \cline{1-3}\cline{5-5}
All constant                    & 12160     & 11.18\%  \\ \cline{1-3}\cline{5-5}
Neighbourhood                   & 65334     & 60.06\% &
                                                          & \multirow{2}{*}{81.4\%}
                                                           \\ \cline{1-3}
Neighbourhood + constants       & 23209     & 21.34\%  \\ \hline
Other                           & 8036      & 7.39\%  \\ \hline \hline
Total                           & 108773    &  \\
\end{tabular}
\end{center}

\noindent
Since, affine, consant, and neighbour are all static patterns, we
group them (\textit{grouping 1}) to see that $92.61\%$ of the array computations have
a static pattern that are either constants or constant of offsets
of base/derived induction variables, thus confirming
\emph{hypothesis 1}. Note that affine (derived induction variables) are very
rare. Overwhelmingly the main category is neighbourhood offsets
(\emph{grouping 2})
(possibly mixed with constants in some dimensions), at $81.4\%$ in
total. The ``affine + constants'' class is not represented at all.

\section{A Specification Language for the Shape of Array Patterns}
\label{sec:lang}

\begin{figure}[t]
\begin{align*}
\def\arraystretch{1}
\setlength{\arraycolsep}{0.2em}
\newcommand{\dimTy}{$\mathbb{N}_{>0}$}
\newcommand{\depthRange}{$\mathbb{N}_{>0}$}
\begin{array}{rl}
\nonterm{specification} ::= & \nonterm{regionDec} \mid \nonterm{specDec} \\
\nonterm{specDec} ::= & \term{stencil} \; \nonterm{spec} \;
                        \texttt{::} \; v \\
\nonterm{regionDec} ::= &  \texttt{region} \; \texttt{::} \; \nonterm{rvar} \; \texttt{=} \;
                         \nonterm{region}\\[0.4em]
%\nonterm{spec} ::= & \nonterm{spatial} \mid \nonterm{temporal}
%\\[1em]
\nonterm{spec} ::= & [\nonterm{mult},] \; [\nonterm{approx},] \; \nonterm{region} \\
\nonterm{mult} ::= & \term{readOnce} \\
\nonterm{approx} ::= & \term{atMost} \; \mid \; \term{atLeast} \\[0.1em]
\nonterm{region} ::= & \nonterm{rvar} \; \mid \;
   \nonterm{rconst} \; \mid \; \nonterm{region} \, \term{+}
  \, \nonterm{region} \; \mid \; \nonterm{region} \; \term{*} \;
    \nonterm{region} \\
\nonterm{rconst} ::= & \stencil{p}{\dimTy}{}{} \; \\
\mid \; & \stencil{f}{\dimTy}{\depthRange}{\;[, \nonpointed]} \\
\mid \; & \stencil{b}{\dimTy}{\depthRange}{\;[, \nonpointed]} \\
\mid \; & \stencil{c}{\dimTy}{\depthRange}{\;[, \nonpointed]} \\[0.25em]
\nonterm{rvar} ::= \; & [\text{\term{a}-\term{z}$\,$\term{A}-\term{Z}$\,$\term{0}-\term{9}}]+\\[-1em]
\end{array}
\end{align*}
\caption{Specification syntax (EBNF grammar)}
\label{fig:syntax}
\end{figure}

\subsection{Specification syntax}
\label{subsec:syntax}

\noindent
\Cref{fig:syntax} gives the syntax of stencil specifications, which is
detailed below. The entry point is the \nonterm{specification}
production which splits into either a \emph{region declaration} or a
\emph{specification declaration}.  Regions comprise \emph{region
  constants} which are combined via
region operators \term{+} and \term{*}.

Region constants (non-terminal \textit{rconst}) specify a finite interval in a 
single dimension starting at the origin and are either \term{pointed},
\term{forward}, \term{backward}, or \term{centered}. The region names are
inspired by numerical analysis terminology, \eg{} the standard explicit method for
approximating PDEs is known as the \emph{Forward Time, Centered Space} (FTCS)
scheme~\citep{dawson1991finite}.
Each region
constant has a dimension identifier $d$ given as a positive natural number.
Each constant except \term{pointed} has a depth
parameter $n$ given as a positive natural number; \term{pointed}
regions implicitly have a depth of $0$.

A \term{forward} region of depth $n$ specifies a contiguous
region in dimension $d$ starting at the origin. This corresponds
to specifying neighbourhood indices in dimension $d$ ranging from $i$ to $i + n$
for some induction variable $i$. Similarly, a
\term{backward} region of depth $n$ corresponds to contiguous indices
from $i$ to $i - n$ and \term{centered}
of depth $n$ from $i - n$ to $i + n$. A \term{pointed}
stencil specifies a neighbour index $i$. For example, the
following shows four specifications with four consistent stencil
kernels reading from arrays \term{a}, \term{b}, \term{c} and \term{d}:
%%
\begin{minted}{fortran}
!= stencil forward(depth=2, dim=1) :: a
e(i, 0) = a(i, 0) + a(i+1, 0) + a(i+2, 0)

!= stencil backward(depth=2, dim=1) :: b
e(i, j) = b(i, j) + b(i-1, j) + b(i-2, j)

!= stencil centered(depth=1, dim=1) :: c
e(i, j) = (c(j-1) + c(j) + c(j+1))/3.0

!= stencil pointed(dim=3) :: d
e(i, j) = d(0, 0, i)
\end{minted}
%%
Not every dimension needs to be specified, \eg{},
specifications on lines $1$, $4$, and $10$ leave some dimensions unspecified.
The \term{forward}, \term{backward}, and \term{centered} regions may
all have an additional attribute \term{nonpointed} which marks absence
of the origin.  For example, the following is a
\term{nonpointed} \term{backward} stencil
%
\begin{minted}{fortran}
!= stencil backward(depth=2, dim=1, nonpointed) :: a
b(i) = a(i-1) + 10*a(i-2)
\end{minted}
%
Not every dimension needs to be specified, \eg{},
specifications on lines $1$, $3$, and $7$ leave some dimensions
unspecified which leaves indices in these dimensions unconstrained.

The \term{forward}, \term{backward}, and \term{centered} regions may
all have an additional attribute \term{nonpointed} which marks absence
of the origin.  For example, the following is a
\term{nonpointed} \term{backward} stencil
%
\begin{minted}{fortran}
!= stencil backward(depth=2, dim=1, nonpointed) :: a
b(i) = a(i-1) + 10*a(i-2)
\end{minted}

\paragraph{Combining regions}

The region operators \term{+} and \term{*} respectively combine
regions by union and intersection. The intersection of two regions
$r \term{*} s$ means that any indices in the specified code must be
consistent with both $r$ and $s$ simultaneously.
 Dually, for the union of two regions
 $r \term{+} s$ means that indices in the specified code must be
 in consistent with one of $r$ or $s$, or both.
For example, the following \emph{nine-point stencil}
has a specification given by the product of two \texttt{centered}
regions in each dimension:
%%
\begin{minted}[breakindent=2.9em]{fortran}
x = a(i, j)   + a(i-1, j)   + a(i+1, j)
y = a(i, j-1) + a(i-1, j-1) + a(i+1, j-1)
z = a(i, j+1) + a(i-1, j+1) + a(i+1, j+1)
!= stencil centered(depth=1, dim=1) * centered(depth=1, dim=2) :: a
b(i, j) = (x + y + z) / 9.0
\end{minted}
%
%This pattern is common in image convolution applications.
The specification ranges over the
values that flow to the array subscript on the left-hand side,
and so ranges over the intermediate assignments to \term{x},
\term{y}, and \term{z}. Each index in the code is consistent
with both specifications simultaneously, \eg{}, \texttt{a(i-1, j+1)}
is within the centered region in dimension $1$ and the centered region
in dimension $2$.

The union of two regions $r \term{+} s$ means that any indices
in the specified code must be consistent with either of $r$ or $s$.
For example, the following gives the specification of a five-point
stencil which is the sum of two compound \texttt{pointed} and
\texttt{centered} regions in each dimension:
%
\begin{minted}{fortran}
!= stencil centered(depth=1, dim=1)*pointed(dim=2) + centered(depth=1, dim=2)*pointed(dim=1) :: a
b(i,j) = -4*a(i,j) + a(i-1,j) + a(i+1,j) + a(i,j-1) + a(i,j+1)
\end{minted}
%%
Here the left-hand side of \texttt{+} says that when the second dimension
(induction variable $j$) is fixed at the origin, the first dimension
(induction variable $i$) accesses the immediate vicinity of the origin
(to depth of one). The right hand side of \texttt{+} is similar but the dimensions are reversed.
This reflects the symmetry under rotation of the five-point stencil.

\paragraph{Region declarations and variables}

Region specifications can be assigned to region variables
(\emph{rvar}) via region declarations. For example, the shape of a
``\emph{Roberts cross}'' edge-detection
convolution~\cite{davis1975survey} can be stated:
%%
\begin{minted}{fortran}
!= region :: r1 = forward(depth=1, dim=1)
!= region :: r2 = forward(depth=1, dim=2)
!= region :: robertsCross = r1*r2
!= stencil robertsCross :: a
\end{minted}
This is useful for common patterns, such as the five-point
pattern, as the regions can be defined once and reused.
%%
\paragraph{Modifiers}
%%
Region specifications can be modified
by \emph{approximation} and \emph{multiplicity} information
(in \textit{spec} in \Cref{fig:syntax}).
The \texttt{readOnce} modifier enforces that no index appears more
than once (that is, its multiplicity is one). For example, all of
the previous examples could have \texttt{readOnce} added:
%
\begin{minted}{fortran}
!= stencil readOnce, backward(depth=2, dim=1) :: a
b(i+1) = a(i) + a(i-1) + a(i-2)
\end{minted}
%
This specification would be invalid if any of the
array subscripts were repeated. This modifier provides a way to
rule out any accidental repetition of array subscripts.
The notion is similar to that of linear types~\cite{wadler1990linear}, where a value must be used
exactly once. We opt for the more informative and easily understood name
\texttt{readOnce}. This modifier is optional, so it need not
be present even if the stencil is linear.

In some cases, it is useful to give a lower and/or upper bound for a
stencil. This can be done using either the \term{atMost} or
\term{atLeast} modifiers. This is particularly useful in situations
where there is a non-contiguous stencil pattern, which cannot be expressed
precisely in our syntax. For example:
%
\begin{minted}{fortran}
!= stencil atLeast, pointed(dim=1)         :: a
!= stencil atMost, forward(depth=4, dim=1) :: a
b(i) = a(i) + a(i+4)
\end{minted}

%%% Local Variables:
%%% mode: latex
%%% TeX-master: t
%%% End:


\section{Equational \& Approximation Theories}
\label{sec:theory}
\noindent
Our region specifications are subject to an equational theory
$\equiv$, which explains which region specifications are equivalent,
and a mutually-defined approximation theory $\preceq$ for
over- and under- approximation on regions.

\subsection{Equivalences}

\noindent
We define an equivalence relation, $\equiv$. The purpose of this
relation is to allow programmers to write specifications with greater
flexibility. It allows specifications to be written in various levels
of compactness allowing for space optimisation, or greater clarity of
documentation. The relation is defined on regions as follows:

\begin{description}
  \item[Basic] \texttt{*} and \texttt{+} are both idempotent, commutative, and
    associative;
%
  \item[Subsumption] If $S$ and $R$ are regions with $S \preceq R$, then
    $S \texttt{+} R \equiv R$ and $S \texttt{*} R \equiv S$.
%
  \item[Distribution] \texttt{*} distributes over \texttt{+} and dually
    \texttt{+} distributes over \texttt{*}, meaning if \textcap{R}, \textcap{S},
    and \textcap{T} are regions, then we have the following dual equivalences:
%
    \begin{align*}
      \texttt{\textcap{R}*(\textcap{S}+\textcap{T})} & \equiv
        \texttt{(\textcap{R}*\textcap{S})+(\textcap{R}*\textcap{T})} &
      \texttt{\textcap{R}+(\textcap{S}*\textcap{T})} & \equiv
        \texttt{(\textcap{R}+\textcap{S})*(\textcap{R}+\textcap{T})}
    \end{align*}
%
  \item[Overlapping pointed] If \textcap{R} is one of \texttt{forward},
    \texttt{backward}, or \texttt{centered}, then we have the following:
%
    \begin{equation*}
      \stencil{\textcap{R}}{$n$}{$k$}{\texttt{nonpointed}} \;\texttt{+}\;
      \stencil{p}{$n$}{}{} \equiv
      \stencil{\textcap{R}}{$n$}{$k$}{}
    \end{equation*}
%
  \item[Centered] The region constants \texttt{forward} and \texttt{backward}
    are two halves of \texttt{centered} specifications:
%
    \begin{align*}
      \stencil[s]{c}{$n$}{$k$}{\textcap{p1}} \equiv
        \stencil[s]{f}{$n$}{$k$}{\textcap{p2}} \texttt{+}
        \stencil[s]{b}{$n$}{$k$}{\textcap{p3}}
    \end{align*}
%
    Here \textcap{p1} is \texttt{nonpointed} if both \textcap{p2} and
    \textcap{p3} are \texttt{nonpointed} and \texttt{pointed} otherwise.
\end{description}

\subsection{Approximations}
%
We define a partial of order of approximations, $\preceq$. This relation is used
in the equational theory and provides a means of writing a more compact lower and upper
bound specifications. The relation is defined as follows:

\begin{description}
  \item[Equivalence] If $S$ and $R$ are regions and $S \equiv R$, then we have
    $S \preceq R$.
%
  \item[Combined] If $S$ and $R$ are regions, then we have
    $S \preceq S \texttt{+} R$ and $S \texttt{*} R \preceq S$.
%
  \item[Depth] Let $k$ and $l$ be in positive integers and $k \leq l$, $n$ some
    fixed dimension, and \textcap{p} either \texttt{pointed} or
    \texttt{nonpointed}. Further, let \textcap{R} be one of \texttt{centered},
    \texttt{forward}, and \texttt{backward}. We then have
%
    \begin{equation*}
      \stencil{R}{$n$}{$k$}{\textcap{p}} \preceq \stencil{R}{$n$}{$l$}{\textcap{p}}
    \end{equation*}
%
\end{description}
%
We present some derivable inequalities that are useful
when writing specifications:

\begin{restatable}[Centered approximation]{prop}{centeredApprox}
  For any dimension $n$, depth $k$, and pointed attribute $p$,
  we have
%
  \begin{align*}
    \stencil{f}{$n$}{$k$}{\textcap{p}} & \preceq
      \stencil{c}{$n$}{$k$}{\textcap{p}} \\
%
    \stencil{b}{$n$}{$k$}{\textcap{p}} & \preceq
      \stencil{c}{$n$}{$k$}{\textcap{p}}
  \end{align*}
\end{restatable}

\begin{restatable}[Point approximation]{prop}{pointApprox}
  Let \textcap{R} be one of \texttt{forward}, \texttt{backward}, and
  \texttt{centered}, $n$ a fixed dimension, and $k$ a fixed depth, then we have
%
  \begin{align*}
    \stencil{p}{$n$}{}{} & \preceq \stencil{R}{$n$}{$k$}{} \\
%
    \stencil{R}{$n$}{$k$}{\texttt{nonpointed}} & \preceq
      \stencil{R}{$n$}{$k$}{}
  \end{align*}
\end{restatable}


\section{Semantic Model}
\label{sec:model}

\noindent
We define a lattice model of array access patterns which serves as both a
denotational model of the semantics of our specification language and an
abstract interpretation domain for source code. This model (1) serves to
explain the meaning of our specifications; (2) is used in the inference and
checking algorithms (\Cref{sec:analysis}); (3) justifies an equational theory
for specifications in the next section; (4) is used to optimise specifications
using lattice identities; and (5) can be used to guide correct implementations.

The model is defined over vectors of sets of integers which we call
\emph{index schemes}.  As an initial informal example, consider the
following simple stencil computation:
%
\begin{minted}{fortran}
do i = 1, n
  b(i) = a(i,0) + a(i+1,0)
end do
\end{minted}
%
The access pattern on array \term{a}, relative
to induction variable \term{i}, is captured
by a vector of length 2 containing integer sets $\langle{\{0, 1\},
  \mathbb{Z}\}}\rangle$. This describes that, in the first dimension,
the array is read at offsets of $0$ and $1$ from an induction
variable. In the second dimension, the index is unconstrained
as it is a constant.

Index schemes form a lattice which provides a rich set of equations
and properties, which we exploit. We first set up the 
domain of the model (\S\ref{sec:domain}), using it to define a semantics
for our specification language (\S\ref{sec:semantics}) and then as
the target for an abstract interpretation on imperative
code (\S\ref{sec:fromcode}).

\subsection{Lattice model of regions}
\label{sec:domain} 

%\begin{defn}[Extended integers]
%  We define the set \zinf{} as
%  $\mathbb{Z}$ extended with $\infty$ and $-\infty$. For any $a$ in
%  \zinf{}, we have $-\infty \leq a \leq \infty$. The resulting set is
%  a total order with top and bottom elements $-\infty$ and
%  $\infty$ respectively.
%\end{defn}

\begin{defn}[Index scheme]
  An $N$-dimensional \emph{index scheme} is a vector of length $N$ of
  integer sets, \ie{}, a member of $\mathcal{P}(\bz)^{N}$.
%  An equivalent view of these vectors is as an $N$-times finite Cartesian
%  product on subsets of $\bz$.
  We use $S, T, U$ to denote index schemes. Henceforth, we assume index schemes
  are all $N$-dimensional for some $N$.

  Index schemes can be \emph{projected} in the $i^{\textit{th}}$ dimension by
  $\pi_i : {\mathcal{P}(\bz{})}^N \to \mathcal{P}(\bz{})$. For an index scheme
  $I$, we refer to $\pi_i(S)$ as the \emph{$i^{th}$ component} of $S$. We assume
  that $i$, when used for projection, always lies between $1$ and $N$.
\end{defn}


\begin{restatable}{lem}{vectorIntersect}
\label{lem:vector-intersect}
  Intersection distributes over index schemes. That is, for index schemes $S, T
  \in \mathcal{P}(\bz)^N$
%
  \begin{equation*}
    S \cap T = \prod_{i = 1}^{N} \pi_i(S) \cap \pi_i(T)
  \end{equation*}
\end{restatable}

Union does not distribute over index schemes, however, a more restricted
property holds.

\begin{restatable}{lem}{vectorUnion}
\label{lem:vector-union}
$S$ and $T$ are index schemes such that $\pi_i(S) = \pi_i(T)$ for all $1 \leq i
\leq N$ apart from some dimension $k$, then:
%
  \begin{equation*}
    S \cup  T
    =
    \pi_1(S) \times \cdots \times
    (\pi_k(S) \cup \pi_k(T)) \times \cdots \times
    \pi_N(S)
  \end{equation*}
\end{restatable}
%
\begin{defn}[Intervals with an optional hole]
  We define an extended notion of closed interval on $\bz{}$ which may contain a
  \emph{hole} at the origin, written \interv{a}{b}{c} where $a$ and $b$ are
  drawn from \bz{} with $a \leq 0 \leq b$ and $c$ is drawn from $\mathbb{B} = \{
  \mathsf{true}, \mathsf{false} \}$. Intervals are interpreted as sets as
  follows:
%
  \begin{equation*}
    \interv{a}{b}{c} \triangleq
      \{ n \mid a \leq n \leq b \wedge (\neg c \implies n \neq 0) \}
  \end{equation*}

  We also add the distinguished interval $\interv{-\infty}{\infty}{}$, which is simply an
  alias for \bz{}, but this notation prevents handling infinite interval
  separately in the following definitions, theorems, and proofs. Here, $-\infty$
  and $\infty$ behaves like top and bottom elements to \bz{} respectively.

  We denote the set of all such intervals (sets) as $\textit{Interval}$. If the
  superscript to the interval is omitted it is treated as $\mathsf{true}$.
\end{defn}
%
\begin{restatable}{lem}{intervalIdentities}
 \label{lem:interval-identities}
  We have the following dual identities for \bz{} intervals:
%
  \begin{align*}
    \interv{a}{b}{c} \cap \interv{d}{e}{f} & =
      \interv{\max \{a,d\}}{\min \{b,e\}}{c \wedge f} \\
    \interv{a}{b}{c} \cup \interv{d}{e}{f} & =
      \interv{\min \{a,d\}}{\max \{b,e\}}{c \vee f}
  \end{align*}
\end{restatable}
%

We define two specialisations (and subset spaces) of index scheme:
\emph{subscript scheme} and \emph{interval scheme}:

\begin{defn}[Interval scheme]\label{def:interval-scheme}
  An interval scheme is a finite Cartesian product of intervals on \bz{},
  denoted by the set $\textit{Interval}^N$ for a product of $N$ intervals.
\end{defn}

\begin{defn}[Subscript scheme]
  A \emph{subscript scheme} is an index scheme scheme where:
  %
  \begin{equation*}
    \forall i.\ 1 \leq i \leq N \implies
      \pi_i (S) = \{ p \}
      \; \vee \;
      \pi_i (S) = \interv{-\infty}{\infty}{}
    \end{equation*}
%
  That is, the $i^{th}$ component of the set is either a singleton in \bz{} or
  the infinite interval.
\end{defn}
%
\begin{defn}[Region]
  A region is an index scheme and \region{N} is the set of all regions (\ie{},
  $\region{N} \subseteq \bz^N)$. The set of all regions is  defined
  as the smallest set satisfying the following:
%
  \begin{enumerate}
    \item If $R$ is in $\textit{Interval}^N$, then $R$ is in \region{N}.
    \item If $R$ and $S$ are in \region{N}, then so are $R \cap S$ and
      $R \cup S$.
  \end{enumerate}
\end{defn}
%
\begin{restatable}{prop}{regionLattice}
  \label{prop:regionLattice}
  $(\region{N},\cup,\cap,\subseteq)$ is a bounded distributive lattice with top
  $\top = \bz{}^N$ and bottom $\bot = \emptyset$.
\end{restatable}
%
\begin{proof}
  Straightforward, the join and meet are mapped to $\cup$ and $\cap$, the set is
  inductively designed to be closed under these operations. Union and
  intersection are associative, commutative, and absroptive under closed sets
  and then they are also for \region{N}. This is enough to show that, it is a
  lattice.

  Further, we have $\bz{}^N \cap R = R$ and $\emptyset \cup R = R$ with
  $\bz{}^N$ and $\emptyset$ belonging to \region{N}. This makes the lattice a
  bounded one.

  Finally, the lattice is distributive since union distributes over intersection
  and vice versa when the set is closed under these operations.
\end{proof}

\begin{defn}
  $\mathsf{Mult}$ and $\mathsf{Approx}$ are parametric labelled variant types
  with injections given by their definition:
%
  \begin{align*}
    \mathsf{Mult} \;\; a \;\; &
      \triangleq \mathsf{mult} \; a \;\mid\; \mathsf{only} \; a \\
    \mathsf{Approx} \;\; a \;\; &
      \triangleq \mathsf{exact} \; a \;\mid\; \mathsf{lower} \; a \;\mid\;
        \mathsf{upper} \; a
  \end{align*}
%
  \eg{}, $\mathsf{lower}$ is an injection $\mathsf{lower} : a \to \mathsf{Approx}
  \; a$ etc.
  These will be used in the following subsection to give meaning to the
  specification modifiers for approximation and multiplicity.
\end{defn}

\subsection{Denotational semantics for specifications}
\label{sec:semantics}

\noindent
An interpretation function $\interp{-}_N$ maps closed\footnote{That
  is, we assume there are no occurrences of \textit{rvar} in a
  specification being modelled.  Any \emph{open} specification
  containing region variables can be made closed by straightforward
  syntactic substitution with a (closed) \textit{region}.}
specifications to sets of $N$-dimensional index schemes with modifier
information, \ie{} specifications are mapped to
$\textsf{Mult} (\textsf{Approx} (\region{N}))$.

The interpretation is overloaded on \emph{regions} in
\Cref{subfig:region-model}. Various intermediate notions are used.

\begin{defn}
  Let $\textit{promote}_N : \mathbb{N}^+ \times \textit{Interval} \to
  \textit{Interval}^N$ be a function generating an interval scheme such that if
  $v$ is $\vecgen{N}{i}{\interv{a}{b}{c}}$, then $\pi_i(v) = \interv{a}{b}{c}$
  and $\pi_j(v) = \bz{}$ in all other dimensions $j$.
\end{defn}

%
\begin{figure}[!t]
\begin{subfigure}[t]{0.5\textwidth}
\begin{align*}
  \interp{-}_N & : \textit{region} \rightarrow \region{N} \\
%
  \interp{\stencil{p}{$i$}{}{}}_N & =
    \vecgen{N}{i}{\interv{0}{0}{\mathsf{true}}}\\
%
  \interp{\stencil{c}{$i$}{$k$}{\textcap{p}}}_N & =
    \vecgen{N}{i}{\interv{-k}{k}{\interp{\textcap{p}}}} \\
%
  \interp{\stencil{f}{$i$}{$k$}{\textcap{p}}}_N & =
    \vecgen{N}{i}{\interv{0}{k}{\interp{\textcap{p}}}} \\
%
  \interp{\stencil{b}{$i$}{$k$}{\textcap{p}}}_N & =
  \vecgen{N}{i}{\interv{-k}{0}{\interp{\textcap{p}}}}
\\
  \interp{\texttt{\textcap{r} + \textcap{s}}}_N & =
    \interp{\textcap{r}}_N \vee \interp{\textcap{s}}_N
\\
  \interp{\texttt{\textcap{r} * \textcap{s}}}_N & =
    \interp{\textcap{r}}_N \wedge \interp{\textcap{s}}_N \\[-1em]
\end{align*}
\caption{Interpretation of regions}
\label{subfig:region-model}
\end{subfigure}
\hspace{1em}
\begin{subfigure}[t]{0.4\textwidth}
\begin{align*}
\interpApprox{-} & : \textit{approx} \rightarrow (A \rightarrow
  \textsf{Approx} \, A) \\
\interpApprox{\texttt{atLeast}} & = \mathsf{lower} \\
  \interpApprox{\texttt{atMost}} & = \mathsf{upper} \\
  \interpApprox{\epsilon} & = \mathsf{exact} \\ \\
  \interpMult{-} & : \textit{mult} \rightarrow (A \rightarrow
  \textsf{Mult} \, A) \\
  \interpMult{\texttt{readOnce}} & = \mathsf{once} \\
  \interpMult{\epsilon} & = \mathsf{mult}
\end{align*}
\caption{Interpretation of modifiers}
\label{subfig:modifier-model}
\end{subfigure}
\label{fig:semantics}
\caption{Semantic model of specifications}
\end{figure}


The first four equations of \Cref{subfig:region-model} model region 
contants. The final two equations model the \term{+} and \term{*}
operators in terms of the join (union) and meet (intersection)
of interval schemes. Thus regions are modelled as members
of $\region{N}$.

We mark in our model the presence of modifiers such as
\texttt{readOnce} and \texttt{atMost} as introduced in \Cref{}.
Approximation modifiers are interpreted as injections into the
$\mathsf{Approx}$ variant by $\interpApprox{}$ in
\Cref{subfig:modifier-model}.  The $\textsf{Approx}$ type corresponds
to the presence or absence of the spatial approximation modifier, with
\textsf{exact} when there is no such modifier and \textsf{lower} and
\textsf{upper} for \term{atLeast} and \term{atMost}. In a similar way,
multiplicity modifiers are interpreted as injections in the
$\mathsf{Mult}$ variant by $\interpMult{}$, corresponding to the to
the presence or absence of the \term{readOnce} modifier as shown in
\Cref{subfig:modifier-model}.

\begin{defn}[Semantics of specifications]
The intermediate interpretations of \Cref{fig:semantics}
are composed to give a model for the top-level specification
syntax as:
%
\begin{equation*}
  \interp{\texttt{stencil \textcap{mult}, \textcap{approx}, \textcap{region}}}_N =
    \interpMult{\textcap{mult}} \;
           {(\interpApprox{\textcap{approx}} \;
                    {\interp{\textcap{region}}_N)}}
\end{equation*}
\end{defn}
%
\begin{thm}
The semantic model $\interp{-}_N$ of $N$-dimensional specifications
is sound with respects to the equational theory of the language,
that is:
%
\begin{equation*}
\forall S, T, N . \quad
S \equiv T \; \Rightarrow \;
\interp{S}_N = \interp{T}_N
\end{equation*}
%
We define the equational theory in \Cref{sec:equational-theory}.
\end{thm}

\subsection{Denotational semantics for array subscripts}
\label{sec:fromcode}

\begin{defn}
  Recall array subscript terms of the form $a(\bar{e})$ from
  \Cref{def:array-subs}. We interpret these terms with the partial
  interpretation $\interp{-}^{\mathit{aterm}} : \textit{array-term}
  \prightarrow{} \mathcal{P}(\bz{})^N$. The interpretation is defined when
  all indices are either constant or neighbourhood indices as defined in
  \Cref{def:neighbour-ix}.
%
  \begin{align*}
    \interp{a(\bar{e})}^{\mathit{aterm}} =
      \prod_{1 \leq i \leq N} \mathit{subscript}(\bar{e}_i) & &
  %
    \textit{subscript}(e) = \begin{cases}
      \{ c \} & e \equiv i \pm c \\
      \bz & e \; \mbox{is constant}
    \end{cases}
  \end{align*}
\end{defn}

%%% Local Variables:
%%% mode: latex
%%% TeX-master: t
%%% End:


\section{Analysis, Checking, and Inference}
\label{sec:algorithms}
\noindent
We outline here the procedures for checking conformance
of source code against specifications (\Cref{subsec:checking})
and for inferring specifications from code (\Cref{subsec:inference}).
Both rely on a program analysis that converts array subscripts
 into sets of index schemes. We outline this analysis
first (\Cref{subsec:analysis}). Note that the analysis
can be made arbitrarily more complex and wide-ranging independent
of the checking and inference procedures. At the moment, the analysis
is largely \emph{syntactic}, with only a small amount of
semantic interpretation of the code.

\begin{example}
\label{exm:checking}
We demonstrate analysis, checking, and inference on the
five-point stencil example:
%%
\begin{minted}{fortran}
b(i, j) = (a(i, j) + a(i-1, j) + a(i+1, j) + a(i, j-1) + a(i, j+1)) / 5.0
\end{minted}
\end{example}

\subsection{Static analysis of array accesses}
\label{subsec:analysis}

\newcommand{\neigh}{\textsf{neigh}}
\noindent
The analysis builds on standard program analyses:
%
\textbf{(1)} basic blocks (CFG); \textbf{(2)} induction variables per
basic block; \textbf{(3)} (interprocedural) data-flow analysis,
providing a \emph{flows to} graph (reaching definitions); and
\textbf{(4)} type information per variable.  The analysis traverses
the control-flow graph of a program top-down and traverses statements
inside of loops bottom-up. The right-hand side of any assignment
statement in a loop is classified based on whether it has array
subscripts flowing to it which are neighbourhood offsets (or a
combination of neighbourhood and absolute in some dimensions). These
are then converted into our model domain by the interpretation
$\interp{-}^{\mathit{aterm}}$ (\Cref{defn:subscript}) and grouped into
a finite map from array variables to set of subscript schemes.  We
denote sets of index/subscripts schemes by $\mathcal{S}$.

For our example, the set of subscript schemes from the analysis is:
%
\begin{equation*}
\mathcal{S}_0 = \{\{0\} \times \{0\}, \{-1\} \times \{0\},
\{0\} \times \{-1\}, \{0\} \times \{1\}, \{1\} \times \{0\}\}
\end{equation*}
%
This set of index schemes is then augmented with multiplicity
information (\textsf{only} or \textsf{mult}) depending on whether
subscripts are unique or not in the analysed statement.
Thus, for each assignment, the analysis generates a map from array
variables to values in $\mathsf{Mult}(\mathcal{P}(\bz)^N)$.

Any assignment statement from which array subscripts flow to the
current assignment is marked as visited such that the main analysis
does not classify it as the root of an array computation.  In the
inference procedure, we assign specifications only to these array
computation roots.

\subsection{Checking code against specifications}
\label{subsec:checking}

\noindent
Checking verifies the access pattern of an array computation in the
source language against any associated specifications. Checking
proceeds by generating a model from a specification and generating a
model from the source code (above), and comparing them for
consistency.  Since the model of \Cref{sec:model} interprets both
array indices and specifications as sets of points in $N$-dimensional
space, the notion of consistency is then intuitively whether these two
(potentially infinite) sets of points are equal. This leads to a
simple notion of consistency $\mathit{consistent}(M_C, M_S)$ which
tests the consistency of a model $M_C$ of source code against
a model $M_S$ of a specification, where
$\mathit{consistent} :
    \mathsf{Mult}(\mathcal{P}(\bz{}^N)) \times
    \mathsf{Mult}(\mathsf{Approx}(\region{N})) \to \mathbb{B}$ is defined:

\begin{align*}
  \mathit{consistent}(\mathit{ixs}, \mathit{model}) & = \begin{cases}
    \mathsf{false} & \mathit{model} = \mathsf{once}(x) \wedge ix =
    \mathsf{mult}(y) \\
    \mathsf{false} & \mathit{model} = \mathsf{mult}(x) \wedge ix = \mathsf{once}(y) \\
    \mathit{consistent'}(\mathit{peel}(ix), \mathit{peel}(model)) & \textit{otherwise}
  \end{cases}
\end{align*}

with the intermediate
$\mathit{consistent'} : \mathcal{P}(\bz{}^N) \times
\mathsf{Approx}(\region{N}) \to \mathbb{B}$ defined as:

\begin{align*}
  \mathit{consistent'}(\mathit{ixs}, \mathit{model}) & = \begin{cases}
    m = \mathit{ixs} & \mathit{model} = \mathsf{exact}(m) \\
    m \supseteq \mathit{ixs} & \mathit{model} = \mathsf{upper}(m) \\
    m \subseteq \mathit{ixs} & \mathit{model} = \mathsf{lower}(m)
  \end{cases}
\end{align*}
%
The $\mathit{consistent}$ function checks whether linearity of the
specification matches that of the indices, \ie{} if the specification allows
indices to be repeated or not. It then delegates to $\mathit{consistent'}$
to check if the points observed in the array terms match the space
defined by the specification. Lower bounds, marked \texttt{atLeast},
require the space defined by the specification to remain inside the
set of indices, while an upper bound, marked by \texttt{atMost},
requires the opposite: enclosure. In the absence of such modifiers, we expect
the space defined by observed indices corresponds exactly with those defined by the
region in the specification, hence requiring set equality.

As explained in \Cref{subsec:union-normal-form}, the sets being compared are
potentially infinite thus equality cannot be computed by exhaustively
comparing elements in each set. Instead, we compile the region into
interval constraints and subscript schemes into membership constraints and pass
these to the \textsc{Z3} SMT solver~\citep{de2008z3} to see if they are equal. The
query is expressed in quantifier-free linear arithmetic, which is decidable.

Although satisfiability is super-exponential in the length of the formula in
the worst case~\cite{fischer1974super}, the performance of consistency checking
using satisfiability is fast in practice as the length of the formula is linearly
related with two factors: (1) dimensionality times the number of regions composed
with \texttt{+} and (2) dimensionality times the number of array terms flowing
into an assignment. In \Cref{subsec:additional-data}, we
established that 99.6\% of array computations have
dimensionality at most four and 97\% of array computations
involve at most 4 array subscript terms. In
\Cref{sec:evaluation}, we show that all specifications in the corpus
comprises no more than two \texttt{+} operations. Thus, our approach
using Z3 is practical and efficient in all but corner cases.


\subsection{Inferring specification automatically}
\label{subsec:inference}
%
\noindent
We provide an inference procedure for generating specifications from
code which are inserted automatically as comments for the root
array computations in a loop. This supports maintenance of a legacy
code base, and aids adoption of the specification language.

The program analysis converts the concrete syntax of
array subscripts into sets of subscript schemes. Inference then has
two parts. First, ``adjacent'' index schemes are coalesced into a
smaller sets of index schemes, remaining in union normal
form. Secondly, the resulting union normal form is converted to the
specification syntax.

\subsubsection{Covering}
A covering of (possibly overlapping) intervals is calculated
by coalescing \emph{adjacent} index schemes
until a fixed-point is reached.

\newcommand{\contig}[2]{\mathit{adjacent}(#1, #2)}
\begin{defn}[Adjacent]
  \label{def:contiguity}
  Two index schemes $S$ and $T$ are \emph{adjacent} written
  $\contig{S}{T}$, iff $\pi_i(S) = \pi_i(T)$ for all $1 \leq i \leq N$
  apart from some dimension $k$ such that $\pi_k(S) = [a, b]$ and
  $\pi_k(T) = [b+1, c]$.
\end{defn}
%
Given a set of indexing schemes $\mathcal{S}$ and a particular index scheme $S$
we generate a set from coalescing $S$ with adjacent index schemes:
%
\begin{equation*}
\mathit{coalesce}(S, \mathcal{S}) =
  \{\,S \cup T \mid T \in \mathcal{S} \, \wedge \,
  \contig{S}{T} \}
\end{equation*}
%
This is then used by the following recursive procedure:
%
\begin{align*}
  \mathit{coalesceStep}(\mathcal{S}) =
  \exists T \in \mathcal{S} .
  \begin{cases}
    \{T\} \cup \mathit{coalesceStep}(\mathcal{S} - \{T\}) &
    \mathit{coalesce}(T, \mathcal{S}) = \emptyset \\
    \mathit{coalesceStep}(\mathit{coalesce}(T, \mathcal{S}) \cup
                                        \mathcal{S} - \{T\}) & \textit{otherwise}
  \end{cases}
\end{align*}
%
This gives a specification rather than an implementation. Our
implementation represents sets by a list, and so
$\exists T \in \mathcal{S}$ above corresponds to deconstructing
the list into its head element $S$ (picking an element from the set).
If $T$ has no adjacent index schemes, then coalescing is attempted on
the rest of the set, and $T$ is returned in the result. Otherwise,
the set of coalesced index scheme is computed and this is passed to
a recursive procedure called $\mathit{coalesceStep}$, along with the rest of
the elements).

The fixed-point of $\textit{coalesceStep}(\mathcal{S})$ is computed to
give a covering over the initial subscript space.
%
\begin{restatable}{lem}{closureInference}
  \label{lem:closuer-inference}
  For a set of index schemes $\mathcal{S}$, then
  $\mathit{coalesceSet}(\mathcal{S})$ is a set of
  index schemes in union normal form.
\end{restatable}
%
For our example, the fixed-pointed of $\mathit{coalesceStep}$ is reached within two
steps:
%
\begin{align*}
  \mathit{coalesceStep}(\mathcal{S}_0) & =
  \{[-1,0] \times [0,0],\,[0,0] \times [-1, 0],\,[0,0] \times
    [0,1],\,[0,1] \times [0,0]\} \\
  \mathit{coalesceStep}^2(\mathcal{S}_0) & =
  \{[-1, 1] \times [0, 0],\,[0, 0] \times [-1, 1]\} = \mathit{coalesceStep}^3(\mathcal{S}_0)
\end{align*}
%
%\todo{Prove that this is in $Region_N$}

\subsubsection{Index schemes to syntax}

\newcommand{\finalSet}{\mathcal{S}_\omega}
Let the covering indexing scheme from the fixed point of
$\mathit{coalesceStep}$ on input $\mathcal{S}_0$ be written as
$\finalSet{}$.  Next, $\finalSet{}$ is translated into specification
syntax. This happens in three stages.
%
\begin{enumerate}[leftmargin=2em]
  \item We check whether $\finalSet{}$
  is the top indexing scheme $\bz{}^N$. This occurs if absolute indices
  appear in each of the dimension across the array subscripts,
  leading to an unconstrained access pattern, which is not represented
  in our syntax, ending inference.

  \item Otherwise, we determine if the indexing schemes are interval
  schemes (\Cref{def:interval-scheme}) so that they can be represented
  exactly by our syntax. It not, they are altered into interval
  schemes to be represented as approximations.

  \item Interval schemes are mapped into a joins of meets
  of region constants, where some intervals split into multiple separate
  region constants.
\end{enumerate}
%
An index scheme is an interval scheme (\Cref{def:interval-scheme}) if
each vector can be represented as a vector of holed intervals
$\interv{a}{b}{c}$.  Since index schemes of the previous stage are all
adjacent and form closed intervals (in the usual mathematical sense)
then an interval with a lower bound $\leq 0$ and an
upper bound $\geq 0$ is sufficient to form an interval scheme.
Otherwise, an approximate specification is generated.

An upper bound is established by
\emph{elongating} any index schemes in multiple dimensions such that
they become interval schemes. The elongation function is
given as
%
\begin{align*}
  \mathit{elongate}([a,b]) = \begin{cases}
    \interv{0}{b}{\textsf{false}} & a > 1 \\
    \interv{a}{0}{\textsf{false}} & b < -1
  \end{cases}
\end{align*}
%
If any dimension of an index scheme is elongated, then the whole index
scheme is elongated. In $\finalSet$, all index schemes that are
representable as interval schemes form a lower bound, whilst those
that do not produce an upper bound by their elongation. A lower bound
may not be generated if none of the indexing schemes are interval schemes.
In the case of our example, the resulting $\finalSet$ comprises a set
of interval schemes, thus we can generate an exact specification.

\paragraph{Translation from interval schemes}

Recall the model of region constants in \Cref{subfig:region-model}
where $\interp{-}$ maps to holed intervals which have either $0$ as
the lower-bound or upper-bound for \fortran{forward} and
\fortran{backward} respectively, or both $0$ for \fortran{pointed}, or
$-k$ and $k$ lower and upper bounds for \fortran{centered}. We can
invert this map to generate region constants from holed
intervals. However, some intervals computed from the above steps
may not match the constraints of this interpretation,
\eg{}, $\interv{-2}{1}{} \times \interv{-1}{1}{}$.  By
\Cref{lem:vector-union}, we can split interval schemes into two into
the relevant form, for example, producing
$\interv{-2}{0}{} \times \interv{-1}{1}{}$ and
$\interv{0}{1}{} \times \interv{-1}{1}{}$. 
The following function
$\textit{split}$ iterates over the components of an index scheme,
splitting apart interval schemes when necessary:
%
\begin{align*}
  %\textit{split}(\emptyset) = \emptyset \quad
  \textit{split}(\interv{a}{b}{c} \times S) =
   \begin{cases}
  \{ \interv{a}{0}{c} \times T \mid T \in \textit{split}(S) \}
  \cup
  \{ \interv{0}{b}{c} \times T \mid T \in \textit{split}(S) \}
  & a < 0 \vee b > 0 \vee (|a| \neq b) \\
  \{ \interv{a}{b}{c} \times T \mid T \in \textit{split}(S) \}
  & \textit{otherwise}
\end{cases}
\end{align*}
%
This forms a set of interval schemes to which $\interp{-}^{-1}$
(\Cref{subfig:region-model}) is well-defined for each component.

The final stage of inference thus maps the component
of every interval scheme to a region constant via $\interp{-}^{-1}$,
combining each component with \texttt{*} can combining the
resulting regions by \texttt{+}, that is:
%
%interval schemes to region constants and combines them. %Individual
%schemes are decomposed using \Cref{lem:vector-intersect} such that
%each of the resulting index schemes have all but one dimensions set to
%\interv{-\infty}{\infty}{}.
%Each of these interval scheme are mapped to region constants using
%\Cref{subfig:region-model} and connected with \texttt{*}.
%
\begin{align}
\textit{convert}_N(\mathcal{S}) = \sum_{S \in \mathcal{S}} \; \prod_{\pi_i(S) \neq
  \interv{-\infty}{\infty}{}} \interp{\pi_i(S)}^{-1}
  \label{eq:convert}
\end{align}
%
Where summation is by the syntactic \term{+} and product is by the
syntactic \term{*}.

In the example, $\interv{-1}{1}{} \times \interv{0}{0}{}$ is mapped to
\texttt{\stencil{c}{1}{1}{} * \stencil{p}{2}{}{}}, and
$\interv{0}{0}{} \times \interv{-1}{1}{}$ is similar. When combined by
\term{+} we obtain the following specification:
%
\begin{equation*}
  \texttt{stencil readOnce,} \;
  \stencil[s]{c}{1}{1}{} \texttt{*} \stencil[s]{p}{2}{}{} \; \texttt{+} \;
  \stencil[s]{p}{1}{}{}  \texttt{*} \stencil[s]{c}{2}{1}{}
\end{equation*}
\vspace{-2em}
\begin{restatable}[Inference soundness]{lem}{inferenceSoundness}
  For all region specs $R$, then $\textit{infer}(\interp{R}_N)_N \equiv R$
\end{restatable}


\section{Evaluation}
\label{sec:evaluation}

\newcommand\regname[1]{\texttt{#1}}
\newcommand\pointed{\regname{pointed}}
\newcommand\forward{\regname{forward}}

To study the effectiveness of our approach,
we built a corpus of over 1 million lines of Fortran code from a
range of scientific computing packages: The Unified Model (UM)~\cite{um},
E3ME~\cite{RePEc:aen:journl:2006se-a12}, BLAS~\cite{blas},
Hybrid4~\cite{GBC:GBC635}, GEOS-Chem~\cite{geos-chem}, Navier (based
on \cite{griebel1997numerical}), Computational Physics
ed. 1~\cite{giordano1997computational},
ARPACK-NG~\cite{arpackng}, MUDPACK~\cite{MUD}, Cliffs~\cite{Cliffs}, and
SPECFEM3D~\cite{specfem3d}.

We first examined how frequently stencil computations occur. We parsed
\num{\overalllinesParsed} lines of Fortran code and found that \overalltickAssignPercent\%
(\num{\overalltickAssign}) of statements have a left-hand side as an
array subscript on neighbourhood indices. This supports the idea that
stencil-like computations are common in scientific code.  We then used
the inference procedure of the previous section to generate
specifications for stencils in the corpus to assess the design of the
language.

We would not expect to infer a stencil for each array statement we
found because our analysis restricts the array-subscript-statements
that we classify as a stencil. For example, \num{\overalltickLHSvar}
potential stencils were discarded because the left-hand side was not
an array access. Overall, we were able to infer a stencil from
\overalltickAssignSuccessPercentOfTickAssign\% of the statements
involving an array assignment. A single statement can involve multiple
arrays and we ended up with \num{\overallnumStencilSpecs}
specifications. This shows that we can express a large number of
stencil shapes within our high-level abstraction and validates our
initial hypothesis that many stencil computations have a regular
shape.

The majority of specifications generated were relatively simple but we
found significant numbers of more complex shapes. We grouped common
patterns into categories:

\textbf{All pointed} \num{\overalljustPointed} of the stencils we
found involved only \pointed\ regions.
% 39,681 of these were pointed in all dimensions.
Common examples of this were pointwise transformations on data (such
as scaling).

\textbf{Single-action} specifications comprise one
forward, backward, or centered region constant combined via \term{+}
or \term{*} with any number of \pointed\ regions. We identified \num{\overallsingleAction} single-action
specifications, of which \num{\overallsingleActionIrr} were single-action with a
\texttt{nonpointed} modifier.

\textbf{Multi-action} specifications comprise at least two
 forward, backward, or centered regions, combined with
any number of \pointed\ regions. We identified \num{\overallmultiAction} multi-action
specifications out of which \num{\overallmultiActionMulOnly} had regions combined only with
$\term{*}$ and \num{\overallmultiActionPlusAndMul} combined with a mix of $\term{*}$ and \term{+}.


% \begin{figure}[t]\begin{minted}[fontsize=\scriptsize,breakindent=0em,linenos=false,xleftmargin=0em,breakafter=)]{fortran}
% !=stencil readOnce,(forward(depth=1,dim=3,nonpointed))*(backward(depth=1,dim=1))*(backward(depth=1,dim=2))+(forward(depth=1,dim=3))*(backward(depth=1,dim=1))*(pointed(dim=2))+(forward(depth=1,dim=3))*(backward(depth=1,dim=1,nonpointed))*(backward(depth=1,dim=2,nonpointed))+(forward(depth=1,dim=3))*(backward(depth=1,dim=2))*(pointed(dim=1))+(backward(depth=1,dim=1))*(backward(depth=1,dim=2,nonpointed))*(pointed(dim=3))+(backward(depth=1,dim=1,nonpointed))*(backward(depth=1,dim=2))*(pointed(dim=3))::x
% \end{minted}
% \caption{Complex specification inferred from
%   \textbf{UM}\label{fig:smagorinsky}}
% \vspace{-1em}
% \end{figure}

% The single- and multi-action classes represent more complex stencils
% with a real possibility for programmer error. As an extreme example, the Unified
% Model has an implementation of the Smagorinsky subgrid-scale model for
% calculating turbulence on which our inference yields 39 specs from 340
% lines of code. This is a large reduction given
% the complexity of the algorithm.  We show one example
% in~\Cref{fig:smagorinsky} which specifies the access pattern to
% a 3-dimensional array (which we have renamed to
% \texttt{x}) arising from a kernel of 93 lines of code involving 142
% array subscripts. The specification was the most complex
% seen in our corpus, yet it still represents a significant abstraction
% of the spatial behaviour given size and complexity of the kernel it describes.

We measured the frequency at which individual specifications involved
multiple occurrences of \term{*}
and \term{+}:
\begin{center}
\vspace{0.25em}
\setlength{\tabcolsep}{0.57em}
{\small{
\hspace{-1em}\begin{tabular}{c|cccccccc}
& 0 & 1 & 2 & 3 & 4 & 5 & 6  \\ \hline
\term{*} & \num{\overallmulOpszero} & \num{\overallmulOpsone} & \num{\overallmulOpstwo} & \num{\overallmulOpsthree} & \num{\overallmulOpsfour} &  & \num{\overallmulOpssix} \\
\term{+} & \num{\overallplusOpszero} & \num{\overallplusOpsone} & \num{\overallplusOpstwo}
\end{tabular}}}
\vspace{0.2em}
\end{center}

\textbf{Bounded} specifications occured with \num{\overallatMost}
\texttt{atMost} bounds and \num{\overallatLeast} \texttt{atLeast}, the latter of which were
always also paired with an upper bound.


% \eg{}, roughly half the specifications did not involve $\term{*}$,
% a quarter use one $\term{*}$ and just under a quarter
% use two $\term{*}$ operators.

\subsection{Limitations}

There were various reasons why we did not infer
specifications on every looped array computation:
\begin{itemize}
\item \textbf{Non-subset induction variables} occur when the
induction variables on the RHS are not a subset of those in the LHS. These
cases are not stencils by our definition. The degenerate case of this
is to have only constant indices on the LHS. We see lots of examples
of this in loops as accumulators \eg{} computing the sum over an array;

\item \textbf{Derived induction variables} where an
index (\mintinline{fortran}{x}) is derived from an
induction variable (\mintinline{fortran}{i}) as in
\mintinline{fortran}{x = len - i};

\item \textbf{Inconsistent induction dimensions} occur when
an induction variable is used to specify more than one array dimension
on the RHS or multiple induction variables are used for the same
dimension on the RHS. These are common in matrix operations such as
LU-decomposition with assignments such as
\mintinline{fortran}{a(l) = a(l) - a(m) * b(l, m)}.
\end{itemize}

\subsection{Detecting errors in the 2-D Jacobi iteration}
\label{sec:jacobi}
One common example of a stencil computation is the two-dimensional
Jacobi iteration that repeatedly goes through each cell in a matrix
and computes the average value of the four adjacent cells. The kernel
is given by:
\begin{minted}{fortran}
  a(i,j) = (a(i-1,j)+a(i+1,j)+a(i,j+1)+a(i,j-1))/4
\end{minted}
We infer a precise specification of its shape as:
\begin{minted}[breakafter=+:,breakindent=-0.6em,breaksymbolsep=0.4em,linenos=false,xleftmargin=-0.5em]{fortran}
  != stencil pointed(dim=1)*centered(depth=1,dim=2,nonpointed)+pointed(dim=2)*centered(depth=1,dim=1,nonpointed)::a
\end{minted}
%
To test our implementation,
we examined whether programmer errors would be detected by replacing
the array index offsets with $-1$, $0$, or $1$ and running our
verification algorithm. Our checking procedure correctly reported a verification
failure in each of 6,537 permutations corresponding to an error.
The iteration computes the average of four adjacent
cells so $24$ ($4$ factorial) of the possible array index perturbations
are correct, all of which are accepted by our checker.

\subsection{Verification case studies}
\label{sec:case-studies}

\subsubsection{Catching bugs}

One problem we face while examining the millions of line of source
code is that we do not have the original programmer's intention at
hand. But sometimes we have the next best thing, the log entry in the
revision control system. Scanning through the UM we identified an
example of an off-by-1 error that could have been avoided through the
use of stencil specification. The buggy code was identified as
\pointed\ in all dimensions:

\begin{minted}[breakafter=+:,breakindent=-0.6em,breaksymbolsep=0.4em,linenos=false,xleftmargin=-0.5em]{fortran}
  != stencil readOnce, pointed(dim=1)*pointed(dim=2)*pointed(dim=3)
\end{minted}

But the programmer made it clear in the comments after the bug fix
that the intention had been \forward\ in the third dimension:

\begin{minted}[breakafter=+:,breakindent=-0.6em,breaksymbolsep=0.4em,linenos=false,xleftmargin=-0.5em]{fortran}
  != stencil readOnce, forward(depth=1, dim=3, nonpointed)*pointed(dim=1)*pointed(dim=2)
\end{minted}


\subsubsection{Aiding refactoring}

One of the initial goals of CamFort was to provide tools that enable
optimisation and refactoring without changing behaviour, and the
array specification feature was designed with that in mind. But
while we were perusing real-world source bases, we realised that there
were also many cases where CamFort could help with refactoring
precisely because they required changes in behaviour. For example, a
relatively common change observed in the logs of revision control for
our corpus is the refactoring of array dimensions. Either
re-ordering, adding or deleting dimensions. Any of these, if not
perfectly propagated throughout all uses, could result in unexpected
outcomes. However, CamFort will pick up the difference during its
stencil specification checking. Ideally, an experienced programmer may
take advantage of the region variable feature of CamFort in order to
minimise the number of specifications that need changing and gain the
full time-saving advantage. But even if not, a quick search for the
specifications can have them updated quickly and then any code that is
not updated will trigger an error when checked.

An example of a refactored region adding a single dimension to an array, adapted from the UM:
\begin{minted}[breakafter=+:,breakindent=-0.6em,breaksymbolsep=0.4em,linenos=false,xleftmargin=-0.5em]{fortran}
  ! before
  != region :: r = readOnce, pointed(dim=1)
  ! after
  != region :: r = readOnce, pointed(dim=1)*pointed(dim=2)
\end{minted}


% \subsubsection{Stress-testing CamFort}


\section{Related Work}
\label{sec:related-work}
\noindent
Various deductive verification tools can express array indexing in their
specifications, \eg{}, ACSL of \citet{baudin2008acsl} for C
(\eg{}~\citet[Example 3.4.1]{burghardt2010acsl}). A specification can be given
for a stencil computation but must use fine-grained indexing as in code and
therefore is similarly prone to indexing errors. Our approach is much more
abstract-- it does not aim to reify indexing in the specification, but
provides simple spatial descriptions which capture a large number of common
patterns.

\citet{kamil2016verified} propose \emph{verified lifting} to extract a
functionally-complete mathematical representation of low-level, and potentially
optimised, stencils in Fortran code. This extracted predicate representation of
a stencil is used to generate code for the \textsc{Halide} high-performance
compiler~\citep{ragan2013halide}. Thus they must capture the full meaning of a
stencil computation which requires significant program analysis. For example,
they report that some degenerate stencil kernels take up to 17 hours to analyse
and others require programmer intervention for correct invariants to be
inferred.

% ACR-I think we could skip this in the interests of space
%This led them to develop \textsf{STNG}, a loop invariant and postcondition
%finder through syntax-guided synthesis. Using this approach they restrict the
%search space of postconditions to predicates of a certain form, one that is
%compatible with \textsf{Halide}. They then look at concrete loop iterations and
%form an hypothesis about the invariant, which is later confirmed or put back
%into the system using a SMT solver.

Our approach differs significantly. Rather than full representation of
array computations, we focus on specifying just the spatial behaviour
in a lightweight way.  Thus, it suffices for us to perform a
comparatively simple data-flow analysis which is efficient, scales
linearly with code size, and does not require any user intervention.
Whilst we do not perform deep semantic analysis of stencils, the
analysis part of our approach can be made arbitrarily more
sophisticated independent of the rest of the work.
% Hence, we do not require SMT solving or search of loop invariants. It
%suffices to do comparatively simple data-flow analysis.
%Since CamFort mostly does
%syntactic analysis, the inference procedure terminates quickly and
%never requires programmer intervention.
%The downside of our
%approach is that if optimisation causes the stencil computation to be heavily
%obfuscated, \textsf{STNG} would capture the access behaviour better.
%
Furthermore, Kamil \emph{et al.} do not provide a user-visible syntactic
representation of their specifications, and nor do they provide verification
from specifications \eg{}, to future-proof the code against later changes. Even
if they were to provide a syntactic representation, for complex stencils such as
Navier-Stokes from \Cref{sec:introduction}, it would be as verbose as the code
itself, making it difficult for programmer to understand the overall shape of
the indexing behaviour.

% GPGPU
Our work has similarities with efforts to verify kernels written for
General-Purpose GPU programming, such as in \citet{Blom:2014:SoCP}.
%Stencils are a form of
%kernel, and GPGPU programming can be viewed as a massively parallel
%method of transforming a large matrix.
However, their focus is mainly on the synchronisation of kernels and the
avoidance of data races., while we are interested in correctness
%embedded within a more typical general-purpose programming
%language. %Other work on GPGPU computation, such as
%\citet{Zhang:2012:CGO}, has focused primarily on generating
%optimised code based on relatively simple specifications: to
%provide performance while keeping the programmer's effort within
%reason.
%
% Sketching Stencils
\citet{Solar-Lezama:2007:PLDI} give specifications of stencils using unoptimised
``reference'' stencils, coupled with partial implementations which are completed
by a code generation tool. %These kinds of specifications are just simple
%implementations, so this tool is useful for hand-written, optimised
%stencils.
The primary purpose of this tool is optimisation rather than correctness, and
the language of specification is more elaborate than ours.

% Pochoir
\citet{Tang:2011:SPAA} define a specification language for writing stencils
embedded in C++ (with Cilk~\citep{blumofe1996cilk} extensions) that are then
compiled into parallel programs based on trapezoidal decompositions with
hyperspace cuts. Pochoir specifications are used for describing the kernel,
boundary conditions, and shape of the stencil. Pochoir is aimed at programmers
reluctant to implement the high-performance cache-oblivious ``hypertrapezoidal''
algorithms.  Like much of the related work, the goal is optimisation rather than
verifying program correctness.

By contrast, the work of \citet{Abe:2013:IPDPSW} studies correctness bringing a
form of model-checking to verify certain stencil computations in the context of
parallelism in partitioned global address
space languages. %These are scenarios where an array is divided into
%subarrays on multiple processors but each has the ability to access
%the others' memory. The authors avoid the state explosion dilemma of
%model-checking by relying upon the fact that accesses to the
%``boundary elements'' will require a different method by virtue of the
%non-locality of that memory. Since this access is performed in a
%different manner, those can be identified automatically by static
%analysis. To achieve this,
Abe \emph{et al.} provide a new language for writing stencil computations. Much
of the specification effort goes towards describing the distribution of the
computation over multiple processors. The code for the stencil kernel is
generated from a relatively high-level specification.  In contrast, we integrate
directly into existing, legacy codebases and established languages, bringing
the benefits of verification more easily to scientific computing.%such as
%older versions of Fortran.% We infer stencil specifications from
%Fortran code and check annotations on stencil computations within
%Fortran code.


\section{Conclusions}
\label{sec:conclusions}

%%% Local Variables:
%%% mode: latex
%%% TeX-master: t
%%% End:


\bibliography{references}

\appendix
% http://2017.splashcon.org/track/splash-2017-OOPSLA#Instructions-for-Authors
% "There is no page limit for bibliographic references and appendices,
% and, therefore, for the overall submission."
% http://2017.splashcon.org/track/splash-2017-OOPSLA#Instructions-for-Authors
% "There is no page limit for bibliographic references and appendices,
% and, therefore, for the overall submission."

\section{Details of the corpus data set}
\label{app:corpus}

\paragraph{Software Corpus}
Table~\ref{tab:corpus} shows summary statistics of the software
packages used in our evaluation, all of which are written in Fortran
90 or Fortran 77. In total we analysed \SI{\overallLoC} lines of code
from \numPackages{} packages, of which we successfully parsed
\SI{\overalllinesParsed} lines. The ``Number of files'' column shows
how many files in each corpus that we were able to analyse with
CamFort. The most common reasons for CamFort rejecting a file were
either use of a C-preprocessor, or illegal use of language features
from a modern Fortran variant.

\begin{enumerate}
\item \textbf{The Unified Model}~\cite{um} is a weather
  forecasting and climate modelling tool developed by the Met Office
  in the United Kingdom. It is used by research organisations and
  meteorological services around the world. We use the development
  branch (trunk) of the model. The code base in closed source but
  institutional licenses are available for research purposes. The Met
  Office runs a comprehensive code quality system incorporating
  dedicated committers (we counted 11) for particular parts of the
  model. We counted 120 additional contributors whose submissions are
  reviewed and tested before being accepted into the code base.

\item \textbf{E3MG} (An Energy-Environment-Economy (E3) Model at the Global
Level) is a macroeconomic model used for assessment of environmental
policy~\cite{RePEc:aen:journl:2006se-a12}. This was developed by
Cambridge Econometrics, an independent consultancy company.

\item \textbf{BLAS}~\cite{blas} (Basic Linear Algebra Subprograms) is
  a popular library providing efficient and portable routines for
  vector and matrix operations. These routines feature in many other
  libraries (including LAPACK). We used version 3.6.0. We chose to
  include this package for breadth, as it provides general numerical
  functions rather than a specialised scientific model.

\item \textbf{Hybrid4} is a vegetation and biomass model for
  simulating carbon, water and nitrogen flows~\cite{GBC:GBC635}.

\item \textbf{GEOS-Chem}~\cite{geos-chem} is a three-dimensional model
  of tropospheric chemistry developed at Harvard and used by $\sim$70
  universities and research institutions world-wide. We use v.10-01.

\item \textbf{Navier} is a small numerical simulation, giving a
  discrete approximation to the two-dimensional Navier-Stokes fluid
  equations, based on the book of~\citet{griebel1997numerical}.

\item \textbf{CP} consists of the example code from the second edition
  of the book ``Computational
  Physics''~\cite{nicholas2006computational} introducing numerical
  techniques and their application to modern physics problems such as
  fields, waves, statistical mechanics and quantum mechanics.

\item \textbf{ARPACK-NG}~\cite{arpackng} \todo{add a sentence of detail}

\item \textbf{SPECFEM3D}~\cite{specfem3d} \todo{add a little detail}
\end{enumerate}


\end{document}
