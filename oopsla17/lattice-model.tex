\documentclass{article}

\usepackage{amsmath}
\usepackage{amssymb}
\usepackage{amsthm}
\usepackage{stmaryrd}
\usepackage{indentfirst}
\usepackage{cleveref}
\usepackage{fixltx2e}

\theoremstyle{definition}
\newtheorem{defn}{Definition}

\theoremstyle{plain}
\newtheorem{lem}{Lemma}
\newtheorem{prop}{Proposition}

\newcommand{\eg}{\emph{e.g.}}

\newcommand{\zinf}{\textnormal{$\mathbb{Z}_\infty$}}
\newcommand{\interp}[1]{\llbracket\texttt{#1}\rrbracket}
\newcommand{\interv}[3]{\textnormal{$\lbrack#1,#2\rbrack^{#3}$}}
\newcommand{\vecgen}[3]{\textnormal{$\textit{vec-gen}_{#1}(#2,#3)$}}
\newcommand{\textcap}[1]{\texttt{\textit{#1}}}

\begin{document}

\section{Giving specifications meaning}

We construct a lattice and use this lattice to give meanings to regions. We
then give full meaning to specifications by wrapping regions in variant types
for multiplicity and approximation. Model defined in this section will later be
used for two purposes. First, we show soundness of the equational theory of
specifications with respect to the model. Second, we define inference and
consistency checking procedures of specifications with respect to the
interpretations of the stencils in the code and the specifications.

\subsection{Constructing the lattice model of regions}

\begin{defn}{(\zinf)}
  We define the set \zinf{} as $\mathbb{Z}$ extended with $\infty$ and
  $-\infty$. For any $n$ in \zinf{}, we have $-\infty \leq n \leq \infty$. The
  resulting set is a total order with top and bottom elements.
\end{defn}

\begin{defn}{(\emph{Offset space})}
  \emph{Offset space} is a subset of \zinf{} members. It is used to capture
  offsets from induction variables.

%  $\mathcal{P}(\zinf{})$ couple with $\supseteq$ is the power set distributive
%  lattice.
\end{defn}

\begin{defn}{(\emph{\zinf{} interval})}
  A \emph{\zinf{} interval} is of the form \interv{a}{b}{c}, where $a$ and $b$
  are drawn from \zinf{} with $a \leq 0 \leq b$ and $c$ is a Boolean value. We
  define it as follows:
%
  \begin{equation*}
    \interv{a}{b}{c} \triangleq
      \{ n \;|\; a \leq n \leq b \wedge (\neg c \implies n \neq 0) \}
  \end{equation*}

  Set of all such intervals (sets) is $\textit{Interval}$.
\end{defn}

\begin{prop}{}
  We have the following identity for \zinf{} intervals:
%
  \begin{equation*}
    \interv{a}{b}{c} \cap \interv{d}{e}{f} =
      \interv{\max \{a,d\}}{\min \{b,e\}}{\neg c \vee \neg f}
  \end{equation*}\label{def:zinf-intersect}
\end{prop}
%
\begin{proof}
  \begin{align*}
    \interv{a}{b}{c} \cap \interv{d}{e}{f} = &
      \; \{ n \;|\; a \leq n \leq b \wedge (\neg c \implies n \neq 0) \}
      \;\cap \\
      & \; \{ n \;|\; d \leq n \leq e \wedge (\neg f \implies n \neq 0) \}
      \\
    = & \; \{ n \;|\; \max \{a,d\} \leq n \leq \min \{b,e\} \wedge (\neg c
      \vee \neg f \implies n \neq 0) \} \\
    = & \; \interv{\max{\{a,d\}}}{\min{\{b,e\}}}{\neg c \vee \neg f}
  \end{align*}
\end{proof}

%\begin{prop}{}
%  $(\textit{Interval}, \supseteq)$ is a sublattice of
%  $(\mathcal{P}(\zinf{}),\supseteq)$ with \interv{-\infty}{\infty}{\textsf{p}}
%  and \interv{0}{0}{\textsf{np}} as its top and bottom elements.
%\end{prop}
%%
%\begin{proof}
%  We know $\textit{Interval} \subseteq \mathcal{P}(\zinf{})$ since intervals
%  represents sets drawn from {\zinf{}}
%
%  Let \interv{a}{b}{c} and \interv{d}{e}{f} be two arbitrary \zinf{} intervals.
%  We need to show that their meet and join both lie in \textit{Interval}.
%%
%  \begin{align*}
%    \interv{a}{b}{c} \wedge \interv{d}{e}{f} = &
%      \{ n \;|\; a \leq n \leq b \wedge (c = \textsf{np} \implies n \neq 0) \}
%      \;\cup \\
%      & \{ n \;|\; d \leq n \leq e \wedge (f = \textsf{np} \implies n \neq 0) \}
%      \\
%    = & \{ n \;|\; \min{a,d} \leq n \leq \max{b,e} \wedge ((c =
%      \textsf{np} \wedge f = \textsf{np}) \implies n \neq 0) \} \\
%    = & \interv{\min{\{a,d\}}}{\max{\{b,e\}}}{k}
%  \end{align*}
%%
%  where $k$ is \textsf{np} if both $c$ and $d$ are \textsf{np} and \textsf{p}
%  otherwise. Minimum of lower bound is still negative or zero and maximum of
%  upper bounds is still positive or zero, hence the meet lies in
%  \textit{Interval}.
%%
%  \begin{align*}
%    \interv{a}{b}{c} \vee \interv{d}{e}{f} = &
%      \{ n \;|\; a \leq n \leq b \wedge (c = \textsf{np} \implies n \neq 0) \}
%      \;\cap \\
%      & \{ n \;|\; d \leq n \leq e \wedge (f = \textsf{np} \implies n \neq 0) \}
%      \\
%    = & \{ n \;|\; \max{a,d} \leq n \leq \min{b,e} \wedge ((c =
%      \textsf{np} \vee f = \textsf{np}) \implies n \neq 0) \} \\
%    = & \interv{\max{\{a,d\}}}{\min{\{b,e\}}}{k}
%  \end{align*}
%%
%  where $k$ is \textsf{np} if either $c$ or $d$ are \textsf{np} and \textsf{p}
%  otherwise. Maximum of lower bound is still negative or zero and minimum of
%  upper bounds is still positive or zero, hence the join lies in
%  \textit{Interval}.
%
%  \interv{-\infty}{\infty}{\textsf{p}} is then \zinf{} and
%  \interv{0}{0}{\textsf{np}} is $\emptyset$. Since these are in the sublattice
%  and they are the bottom and top elements in the parent lattice, they are
%  necessarily the bottom and top elements in the sublattice.
%\end{proof}

%\begin{prop}{}
%  $(\textit{Interval},\supseteq)$ is a distributive lattice.
%\end{prop}
%%
%\begin{proof}
%  It is enough to show for arbitrary $\interv{a}{b}{c}$, $\interv{d}{e}{f}$,
%  $\interv{g}{h}{i}$ in \textit{Interval}, $\interv{a}{b}{c} \vee
%  (\interv{d}{e}{f} \wedge \interv{g}{h}{i}) = (\interv{a}{b}{c} \vee
%  \interv{d}{e}{f}) \wedge (\interv{a}{b}{c} \vee \interv{g}{h}{i})$. When this
%  holds the dual law also holds. The property immediately follows from
%  definition due to distributivity of set intersection over union.
%\end{proof}

\begin{defn}{(\emph{Offsets vector})}
  An \emph{offsets vector} is a $N$ dimensional vector of offset spaces denoted
  by $(\textit{offs}_1, \dots, \textit{offs}_N)$ to mean $\textit{offs}_1 \times
  \dots \times \textit{offs}_N$.

  The set of $N$ dimensional offsets vectors is
  $\mathcal{P}{(\zinf{}^{\mkern-19mu{N}})}$ and they trivially form a
  distributive lattice under the superset relation.
\end{defn}

\begin{defn}{(\emph{Vector interval})}
  A vector interval is an $N$ dimensional object which has \zinf{} intervals at
  each dimension. It is denoted with $(\textit{int}_1, \dots, \textit{int}_n)$,
  where $\textit{int}_i$ is a \zinf{} interval. If \textit{vint} is a vector
  interval, $\textit{vint}_i$ denotes the \zinf{} interval at the $i^{th}$
  dimension.

  The set of $N$ dimensional vector intervals is $\textit{Interval}^N$.
\end{defn}

\begin{lem}{}
  Intersection distributes over vector intervals.
%
  \begin{equation*}
    (\textit{vint}_1, \dots, \textit{vint}_N) \;\cap\;
      (\textit{vint'}_1, \dots, \textit{vint'}_N)
      =
    (\textit{vint}_1 \;\cap\; \textit{vint'}_1, \dots,
     \textit{vint}_N \;\cap\; \textit{vint'}_N)
  \end{equation*}\label{lem:vector-dist}
\end{lem}
%
\begin{prop}{}
  We have the following identity for $N$ dimensional vector intervals:
%
  \begin{gather*}
    (\interv{a_1}{b_1}{c_1}, \dots, \interv{a_N}{b_N}{c_N}) \; \cap \;
      (\interv{a'_1}{b'_1}{c'_1}, \dots, \interv{a'_N}{b'_N}{c'_N}) \\
    = \\
    (\interv{\max \{a_1,a'_1\}}{\min \{b_1,b'_1\}}{\neg c_1 \vee \neg c'_1},
      \dots, \interv{\max \{a_N,a'_N\}}{\min \{b_N,b'_N\}}{\neg c_N \vee \neg c'_N})
  \end{gather*}
\end{prop}
%
\begin{proof}
  This immediately follows from \Cref{lem:vector-dist} and
  \Cref{def:zinf-intersect}.
\end{proof}

\begin{defn}{(\emph{Region space})}
  Region space defines the set \textit{Region} in $N$ dimensions inductively as
  follows:
%
  \begin{enumerate}
    \item If $R$ is in $\textit{Interval}^N$, then $R$ is in \textit{Region}.
    \item If $R$ and $S$ are in \textit{Region}, then so are $R \cap S$ and $R
      \cup S$.
  \end{enumerate}
\end{defn}

\begin{prop}{}
  $(\textit{Region},\supseteq)$ is a distributive lattice.
\end{prop}
%
\begin{proof}
  Any sublattice of a distributive lattice is a distributive lattice. Power sets
  coupled with partially ordered with superset or equals relation are
  distributive lattices. \textit{Region} is a subset of
  $\mathcal{P}{(\zinf{}^{\mkern-19mu{N}})}$ since all vector intervals are
  subsets of this power set and union and intersection cannot add an element
  that is not already in the powerset. Therefore, \textit{Region} is a
  distributive sublattice of $\mathcal{P}{(\zinf{}^{\mkern-19mu{N}})}$.
\end{proof}

\begin{defn}{}
  \textsf{Mult} and \textsf{Approx} are parametric labelled variant types with
  injections given by their definition:
%
  \begin{align*}
    \textsf{Mult} \; a &
      \triangleq \textsf{mult} \; a \;\mid\; \textsf{only} \; a \\
    \textsf{Approx} \; a &
      \triangleq \textsf{exact} \; a \;\mid\; \textsf{lower} \; a \;\mid\;
        \textsf{upper} \; a \;\mid\; \textsf{both} \; a \; a
  \end{align*}
%
  \eg{}, \textsf{lower} is an injection $\mathsf{lower} : a \to \mathsf{Approx}
  \; a$ etc.

  These will be used in the following subsection to give meaning to the
  specification modifiers for approximation and multiplicity.
\end{defn}

\subsection{Denotational semantics for specifications}

We give some basic interpretations: $\interp{\texttt{n}} = n$ where $n$ is in
$\mathbb{Z}$, $\interp{pointed} = \llbracket \epsilon \rrbracket =
\textsf{true}$, $\interp{\texttt{nonpointed}} = \textsf{false}$.
Typographically, cursive font is used within specifications to indicate a
partial textual capture.

Let $\textit{vec-gen}_N : \mathbb{N} \times \textit{Interval} \to
\textit{Interval}^N$ be the function generating an $N$ dimensional vector
interval such that \vecgen{N}{i}{\interv{a}{b}{c}} has
$\interv{-\infty}{\infty}{\textsf{p}}$ in all dimensions except the $i^{th}$
one. There, it has \interv{a}{b}{c}.

First, we give meaning to region constants (introduced in \Cref{}).
%
\begin{align*}
  \interp{\texttt{pointed(dim=\textit{i})}}_N & =
    \vecgen{N}{\interp{\textit{i}}}{\interv{0}{0}{\textsf{true}}}\\
%
  \interp{\texttt{centered(dim=\textit{i}, depth=\textit{k}, \textit{p})}}_N & =
    \vecgen{N}
           {\interp{\textit{i}}}
           {\interv{-\interp{\textit{k}}}
                   {\interp{\textit{k}}}
                   {\interp{\textit{p}}}
           } \\
%
  \interp{\texttt{forward(dim=\textit{i}, depth=\textit{k}, \textit{p})}}_N & =
    \vecgen{N}
           {\interp{\textit{i}}}
           {\interv{0}
                   {\interp{\textit{k}}}
                   {\interp{\textit{p}}}
           } \\
%
  \interp{\texttt{backward(dim=\textit{i}, depth=\textit{k}, \textit{p})}}_N & =
    \vecgen{N}
           {\interp{\textit{i}}}
           {\interv{-\interp{\textit{k}}}
                   {0}
                   {\interp{\textit{p}}}
           }
\end{align*}

Second, we provide the semantics for the permissive region operator, \texttt{+},
and the restrictive region operator, \texttt{*}, by associating them with meet
and join respectively. Let \texttt{\textit{R}} and \texttt{\textit{S}} be
regions.
%
\begin{align*}
  \interp{\texttt{\textit{R} + \textit{S}}}_N & =
    \interp{\texttt{\textit{R}}}_N \wedge \interp{\texttt{\textit{S}}}_N
&
  \interp{\texttt{\textit{R} * \textit{S}}}_N & =
    \interp{\texttt{\textit{R}}}_N \vee \interp{\texttt{\textit{S}}}_N
\end{align*}

Third, we need to mark presence of modifiers such as \texttt{readOnce} and
\texttt{atMost} as introduced in \Cref{}. In the context of
multiplicity modifiers we have the following semantics:
%
\begin{align*}
  \interp{readOnce} & = \textsf{mult} &
  \llbracket\epsilon\rrbracket & = \textsf{once}
\end{align*}

In the context of approximation modifiers, we have
%
\begin{align*}
  \interp{atLeast} & = \textsf{lower} &
  \interp{atMost} & = \textsf{upper} &
  \llbracket \epsilon \rrbracket = \textsf{exact}
\end{align*}

Finally, we can give full semantics of a stencil specification:
%
\begin{equation*}
  \interp{stencil \textit{Mult}, \textit{Appr}, \textit{Region}} =
    \interp{\textit{Mult}}{(\interp{\textit{Appr}}(\interp{\textit{Region}}))}
\end{equation*}

\section{Equational \& approximation theories}

We know introduce the equational and approximation theory of stencil
specifications based on $\equiv$ and $<\:$.

\subsection{Equivalences}

\begin{description}
  \item[Basic] \texttt{*} and \texttt{+} are both idempotent, commutative, and
    associative;
%
  \item[Dist] \texttt{*} distributed over \texttt{+} and dually
    \texttt{+} distributes over \texttt{*}, meaning if \textcap{R}, \textcap{S},
    and \textcap{T} are regions, then we have the following equivalences:
%
    \begin{align*}
      \texttt{\textcap{R}*(\textcap{S}+\textcap{T})} & \equiv
        \texttt{(\textcap{R}*\textcap{S})+(\textcap{R}*\textcap{T})} &
      \texttt{\textcap{R}+(\textcap{S}*\textcap{T})} & \equiv
        \texttt{(\textcap{R}+\textcap{S})*(\textcap{R}+\textcap{T})}
    \end{align*}
%
  \item[Overlapping pointed] If \textcap{R} is one of \texttt{forward},
    \texttt{backward}, or \texttt{centered}, then we have
    $\texttt{\textcap{R}(depth=\textcap{k},dim=\textcap{n},\_) +
    pointed(dim=\textcap{n})} \equiv
    \texttt{\textcap{R}(depth=\textcap{k},dim=\textcap{n})}$.
%
  \item[Centered] The region constants \texttt{forward} and \texttt{backward}
    are two halves of \texttt{centered} specifications:
%
    \begin{gather*}
      \texttt{centered(depth=\textcap{k},dim=\textit{n},\textcap{p1})} \\
      \equiv \\
      \texttt{forward(depth=\textcap{k},dim=\textit{n},\textcap{p2}) +
        backward(depth=\textcap{k},dim=\textit{n},\textcap{p3})}
    \end{gather*}
%
    Here \textcap{p1} is \texttt{nonpointed} if both \textcap{p1} and
    \textcap{p3} are \texttt{nonpointed} and \texttt{pointed} otherwise.
\end{description}

\subsection{Approximations}

\section{Analysis, Consistency, Inference}

\subsection{Analysis}

\subsection{Consisteny}

\subsection{Inference}

\end{document}
