\documentclass[10pt]{sigplanconf}

\usepackage{amssymb}
\usepackage{amsmath}
\usepackage{amsthm}
\usepackage{stmaryrd}
\usepackage{color}
\usepackage{graphics}
\usepackage{fancyvrb}
\CustomVerbatimEnvironment{SpecVerbatim}{Verbatim}{fontsize=\footnotesize,xleftmargin=0.65cm,
xrightmargin=0.2cm,commandchars=\\\{\},baselinestretch=0.97}
\CustomVerbatimEnvironment{ExmVerbatim}{Verbatim}{fontsize=\footnotesize,xleftmargin=0.65cm,
xrightmargin=0.2cm,commandchars=\\\{\},baselinestretch=0.97}
\CustomVerbatimEnvironment{IVerbatim}{Verbatim}{fontsize=\relsize{-1},xleftmargin=0.65cm,
xrightmargin=0.2cm,commandchars=\\\{\},baselinestretch=0.97}


\definecolor{darkgreen}{rgb}{0.0,0.5,0.0}
\definecolor{darkpurple}{rgb}{0.6,0.0,0.6}
\definecolor{orange}{rgb}{0.8,0.4,0.0}
\definecolor{darkorange}{rgb}{0.5,0.2,0.0}
\definecolor{marco}{rgb}{0.0,0.3,0.5}
\definecolor{gray}{rgb}{0.2,0.2,0.2}

\newcommand{\dnote}[1]{\textcolor{darkpurple}{Dom: #1}}

\newcounter{block}

\newtheorem{theorem}[block]{Theorem}
\newtheorem{lemma}[block]{Lemma}
\newtheorem{proposition}[block]{Proposition}
%\newtheorem{definition}[block]{Definition}

\theoremstyle{definition}
\newtheorem{remark}[block]{Remark}
\newtheorem{example}[block]{Example}
\newtheorem{definition}[block]{Definition}

% Writing macros
\newcommand{\ie}{\emph{i.e..}}
\newcommand{\eg}{\emph{e.g.}}

% Semantics related
\newcommand{\interp}[1]{\llbracket{#1}\rrbracket}

\title{Abstract Shape Specifications for Stencil Verification}
\authorinfo{}{}

\begin{document}
\maketitle

\begin{abstract}
\end{abstract}

\bibliographystyle{plain}

\section{Introduction}


\section{Specification language}

\begin{align*}
spec ::= & \; S  \; \texttt{,} \; spec \, \mid \, S \\[0.5em]
%
S ::= &  \; \texttt{forward} \; \texttt{depth=}\textit{depth} \; \texttt{dims=}\textit{dims} \\
& \; \mid \texttt{backward} \; \texttt{depth=}\textit{depth} \; \texttt{dims=}\textit{dims} \\
& \; \mid \texttt{centered} \; \texttt{depth=}\textit{depth} \; \texttt{dims=}\textit{dims} \\
& \; \mid \texttt{unspecified} \; \texttt{dims=}[\textit{dims}] \\
& \; \mid \texttt{read-once} \; S \\[0.5em]
\textit{depth} ::= & \; \mathbb{N} \\
\textit{dims}  ::= & \; \textit{var} \; \texttt{,} \; \textit{dims} \,
                      \mid \, \textit{var}
\end{align*}

\subsection{Common patterns}

\begin{example}[FTCS (Forward time, centered space)]

  Otherwise known as the \emph{explicit method}, is a common
  pattern since it tends to be more efficient than other approches
  (see BTCS and CTCS below). However, its disadvantage is that it is
  unstable if the resolution is not well chosen. 

\begin{ExmVerbatim}
b(x) = a(x) + r*(a(x-1) - 2*a(x) + a(x+1))
\end{ExmVerbatim}
%
For this statement, the inference provides the specification: 
%
\begin{SpecVerbatim}
a: centered depth=1, dims=0
\end{SpecVerbatim}

\end{example}

\paragraph{BTCS (Backward time, centered space) -- implicit method}

\paragraph{CTCS (Centered time, centered space) - Crank-Nicolson method}



\section{Static analysis}

In the following, we defined
\emph{base induction variables} to be the variables
 defined by a ``for'' loop (\eg{}, that get incremented each time the
loop is re-entered). 

\begin{definition}[Constant translations]
An indexing expression $e$ is a \emph{constant translation} if,
for a base induction variable $i$, then $e \equiv i + a$ or $e \equiv i - a$ 
where $a$ is a constant. The relation $\equiv$ identifies terms
up-to commutativity of $+$ and the inverse
relation of $+$ and $-$ (\eg{}, $(-b) + i \equiv i - b$). 
In the following, we classify constant translation expressions 
using the predicate $\textsf{trans}$.
\end{definition}

\paragraph{Step 1: Data-access analysis}

\dnote{Might want to do 'stencil' (gather) and 'mould' (scatter)
  patterns separately}
\begin{align*}
\begin{array}{lll}
\interp{e_1 = e_2}    & = \interp{e_2} \\
\interp{e_1 e_2}      & = \interp{e_1} \cup \interp{e_2} \\
\interp{a(\tilde{e})} & = \{e | e \in \tilde{e}, \textsf{trans}(e)\} & \textit{where each $e \in
                                          \tilde{e}$ is a constant or constant translation}
\end{array}
\end{align*}

Index analysis is a \emph{coeffect analysis}.

\paragraph{Step 2: Coalesce contiguous indices into regions}

% fairly mechanical

\section{Evaluation}

\section{Discussion}


\bibliography{references}

\end{document}