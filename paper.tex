\documentclass[10pt]{sigplanconf}

\usepackage{amssymb}
\usepackage{amsmath}
\usepackage{amsthm}
\usepackage{stmaryrd}
\usepackage{color}
\usepackage{graphics}
\usepackage{fancyvrb}
\usepackage{subfigure}
\usepackage{amsthm}

\CustomVerbatimEnvironment{SpecVerbatim}{Verbatim}{fontsize=\footnotesize,xleftmargin=0.65cm,
xrightmargin=0.2cm,commandchars=\\\{\},baselinestretch=0.97}
\CustomVerbatimEnvironment{ExmVerbatim}{Verbatim}{fontsize=\footnotesize,xleftmargin=0.65cm,
xrightmargin=0.2cm,baselinestretch=0.97,numbers=left}
\CustomVerbatimEnvironment{IVerbatim}{Verbatim}{fontsize=\relsize{-1},xleftmargin=0.65cm,
xrightmargin=0.2cm,commandchars=\\\{\},baselinestretch=0.97}


\definecolor{darkgreen}{rgb}{0.0,0.5,0.0}
\definecolor{darkpurple}{rgb}{0.6,0.0,0.6}
\definecolor{orange}{rgb}{0.8,0.4,0.0}
\definecolor{darkorange}{rgb}{0.5,0.2,0.0}
\definecolor{marco}{rgb}{0.0,0.3,0.5}
\definecolor{gray}{rgb}{0.2,0.2,0.2}

\newcommand{\bn}{\mathbb{N}}

\newcommand{\dnote}[1]{\textcolor{darkpurple}{Dom: #1}}
\newcommand{\mnote}[1]{\textcolor{darkgreen}{Mistral: #1}}

\newcounter{block}

\newtheorem{lemma}[block]{Lemma}
\newtheorem{proposition}[block]{Proposition}
%\newtheorem{definition}[block]{Definition}

\theoremstyle{definition}

\newtheorem{theorem}[block]{Theorem} 
\newtheorem{remark}[block]{Remark}
\newtheorem{example}[block]{Example}
\newtheorem{definition}[block]{Definition}

% Writing macros
\newcommand{\ie}{\emph{i.e..}}
\newcommand{\eg}{\emph{e.g.}}

\newcommand{\dimId}{\texttt{dim}}

% Semantics related
\newcommand{\interp}[1]{\llbracket{#1}\rrbracket}

\title{Time and space specifications for stencils}
\authorinfo{}{}

\begin{document}
\maketitle

\begin{abstract}
\end{abstract}

\bibliographystyle{plain}

\section{Introduction}

\emph{Stencils} are a ubiquitous programming pattern in many numerical
applications. Informally, a stencil computation computes
an array, where the value at each index $i$ of this array is
calculated from a set of neighbouring values to $i$ in some input
array(s), \eg{}, the Game of Life, convolutions in image processing, 
approximations to differential equations. For example, the following
psuedo-code iteratively applies the discrete Laplace
operator (an approximation to a derivative):
%
\begin{ExmVerbatim}
do iter = 0, itermax
   do i = 1, (n-1)
      b(i) = a(i-1) - 2*a(i) + a(i+1)
   a = b
\end{ExmVerbatim}
%
Line 3 is the core of the stencil computation, calculating
the value at \texttt{b[i]} from a neighbourhood of elements about
\texttt{a[i]}. Line 4 swaps 
\texttt{a} and \texttt{b} between iterations, where \texttt{b} becomes the
input for the next iteration.

%This stencil computation exhibits statically decideable
%spatial and temporal relationships between $\texttt{a}$ and
%$\texttt{b}$. 
In this simple example, the spatial and temporal relationship
between \texttt{a} and \texttt{b} is easy to understand. However,
more complex stencil computations can be much harder to understand
and subsequently more prone to error. For example, Figure~\ref{ref:navier-stokes-fragment}
shows a couple of lines from a Navier-Stokes fluid simulator in
which two arrays are read from with different data access patterns,
across two dimensions. The interaction is much harder to understand,
with the potential for the developer to accidentally introduce an
error, for example writing $\texttt{(i-1,j)}$ instead of
$\texttt{(i+1,j)}$.

We introduce a simple specification for stencil comptuations
that abstracts the fine grained detail of their access patterns.
In practise, most stencil computations have a regular shape that
can be described simply and abstractly. In the case of our
first Laplace example, our analysis tool infers the following specification:
%
\begin{SpecVerbatim}
a : centered depth=1 dims=1
b : backward depth=1\{a\} dims=t
\end{SpecVerbatim}
%
The first line explains that \texttt{a} is read from
with a symmetrical stencil pattern (``centered'') to a depth
of one in each direction in its first (and only) dimension.
The second line explains the temporal relationship between
\texttt{b} and \texttt{a}: that the previous time step for \texttt{b}
is actually provided by \texttt{a}. The dimension \texttt{t} is a
reserved dimension identifiers for temporal specifications. In
the case of the Navier-Stokes example, its inferred stencil is shown
in Figure~\ref{ref:navier-stokes-fragment}(b).

In this paper, we make the following contributions:
%
\begin{itemize}
\item we introduce a specification language for
stencil computations that captures many common forms
of data access pattern, both spatial and temporal (Section~\ref{sec:lang});
\item we detail inference and checking
algorithms for stencil specifications (Section~\ref{sec:analysis});
\item we evaluate our implementation of the approach
in the CamFort tool for Fortran verification, studying
a number of example programs to assess the usefulness
of this approach.
\dnote{insert results here}
\end{itemize}
%

\begin{figure}
\begin{ExmVerbatim}[firstnumber=20]
du2dx = ((u(i,j)+u(i+1,j))*(u(i,j)+u(i+1,j))+ 
    gamma*abs(u(i,j)+u(i+1,j))*(u(i,j)-u(i+1,j))- 
    (u(i-1,j)+u(i,j))*(u(i-1,j)+u(i,j))- 
    gamma*abs(u(i-1,j)+u(i,j))*(u(i-1,j)-u(i,j))) 
    /(4.0*delx)

duvdy = ((v(i,j)+v(i+1,j))*(u(i,j)+u(i,j+1))+ 
   gamma*abs(v(i,j)+v(i+1,j))*(u(i,j)-u(i,j+1))- 
   (v(i,j-1)+v(i+1,j-1))*(u(i,j-1)+u(i,j))-  
   gamma*abs(v(i,j-1)+v(i+1,j-1))*(u(i,j-1)-u(i,j))) 
   /(4.0*dely)

laplu = (u(i+1,j)-2.0*u(i,j)+u(i-1,j))/delx/delx+ 
          (u(i,j+1)-2.0*u(i,j)+u(i,j-1))/dely/dely
\end{ExmVerbatim} 
(a). Excerpt from the Fortran code, 
showing highly-detailed stencil computations. \\

\begin{SpecVerbatim}[xleftmargin=0.1cm]
20-24  u: centered depth=1 dim=1
26-30  u: centered depth=1 dim=2
       v: forward depth=1 dim=1, backward depth=1 dim=1
32-33  u: centered depth=1 dim=1,2
\end{SpecVerbatim}
(b). Inferred stencil specification from CamFort
\caption{Fragment of Navier-Stokes fluid simulator and its specification}
\label{ref:navier-stokes-fragment}
\end{figure}

\section{Stencil specification language}
\label{sec:lang}

Our specification system is based on the observation
that most forms of array access in numberical code have 
a fixed, statically-determined access pattern. For example, the 
``\emph{five-point stencil}'' on a two-dimensional array reads from array
indices $(i, j)$, $(i-1, j)$, $(i+1, j)$, $(i, j-1)$, and $(i, j+1)$
for all $i, j$ within the inner boundary of the array (to avoid 
out-of-bounds access at the edges). We revisit this hypothesis 
in Section~\ref{sec:evaluation} where the inference of
such regular stencil patterns on a corpus of numerical programs (both
small and large). We found that indeed \dnote{..}. 

We outline the specification language here. Section~\ref{sec:syntax}
oultines the syntax. Section~\ref{sec:semantics} defines its semantics
via a simple multi-set interpretation over indices. Section~\ref{sec:eqs} provides an
equational theory for specifications via relation ($\equiv$) and a
theory of approximation via relation ($<:$). These equations are then
proven sound with respecto the multi-set semantics of Section~\ref{sec:semantics}.

\subsection{Syntax}
\label{sec:syntax}

Figure~\ref{fig:syntax}
gives the syntax, which we introduces in stages below. 
%%

\begin{figure}
\begin{align*}
\textit{specs} ::= & \; \textit{spec} \; \texttt{;} \; \textit{specs} \, \mid \, \textit{spec} \\
%
\textit{spec} :: = & \; \texttt{v} : S \\[0.5em]
S ::= &  \; \texttt{forward} \; [\texttt{depth=}\textit{depth}] \; \texttt{dims=}\textit{dims} \\
& \; \mid \texttt{backward} \; [\texttt{depth=}\textit{depth}] \; \texttt{dims=}\textit{dims} \\
& \; \mid \texttt{centered} \; [\texttt{depth=}\textit{depth}] \; \texttt{dims=}\textit{dims} \\
& \; \mid \texttt{unspecified} \; [\texttt{dims=}\textit{dims}] \\
& \; \mid \texttt{read-once} \; S \\[0.5em]
\textit{depth} ::= & \; \mathbb{N} \, \mid \, \textit{temp-spec} \\ 
\textit{temp-spec} ::= & \; \mathbb{N}\{var\} \\
\textit{dims} ::= & \; \mathbb{D} \; \texttt{,} \; \textit{dims} \, \mid
                    \, \mathbb{D} \\
\mathbb{D}  ::= & \; \mathbb{N}_{>0} \, \mid \, \texttt{t}
\end{align*}
\caption{Syntax of specifications}
\label{fig:syntax}
\end{figure}

\paragraph{Spatial specifications}

\paragraph{Time specifications}
\dnote{The below is old so probably out of date}

\begin{ExmVerbatim}
for t = 1, itermax:
   for i = 1, (n-1):
      a[t, i] = a[t-1, i-1] - 2*a[t-1, i] + a[t-1, i+1]
\end{ExmVerbatim}

\begin{SpecVerbatim}
a(t=0) : bwd, depth=1
a(1)   : centered, depth=1
\end{SpecVerbatim}

\begin{SpecVerbatim}
a : centered depth=1 dim=1
a : backward depth=1 dim=0
\end{SpecVerbatim}

\subsection{A semantic model for specifications}
\label{sec:semantics}

We give a model of our specifications which explains
the range of indices which correspond to the specifications.
The model is based on multi-sets since array indices my be repeated
in an expression.

Figure~\ref{fig:model} describes the model in terms of 
an interpretation function $\interp{-}$ which maps specifications
to multisets of relative indices. 

\begin{definition}Relative indices are drawn
from $(\mathbb{Z}_{\bot})^{\mathbb{D}}$, essentially an infinite
tuple of offsets. For example, $(i, j+1)$ corresponds to
the relative index $(0, 1, \bot, \bot, \ldots)$
\end{definition}

The main interpretation function is overloaded on lists of 
variable-spec pairs $\interp{\overline{v : S}}$, 
returning multisets of variable-relative-index pairs. This provides
the top-level definition of the model, with $\interp{-}$ overloaded
on $S$, $\interp{S}$ mqpping to multisets of
relative indices not associated to an array avariable.

\begin{figure}
\begin{align*}
\hspace{-2em}
\begin{array}{c}
\hspace{-1em}
\setlength{\arraycolsep}{0.1em}
\begin{array}{rl}
\interp{\textit{spec}; \textit{specs}} & = \interp{spec} \cup
  \interp{\textit{specs}} \\
%% TEMPORAL SPECS
% \interp{\texttt{v} : \texttt{forward} \; \texttt{depth}=n\texttt{\{u\}} \;
%  \texttt{dims=t}} & =  \{ (\texttt{u}, \textit{ix})^m \mid \forall m,
%                     ix : \mathbb{Z}^{\mathbb{D}}, \textit{ix} \,
%                     \texttt{t} = 0 \} \\[0.4em]
%% SPATIAL SPECS
\interp{\texttt{v} : S} & = \{(\texttt{v}, ix)^m \mid ix^m \in
  \interp{S}\} \\
\end{array}\\[1em]
%\framebox{$\interp{S}$} \\[1em]
\begin{array}{ll}
& \interp{\texttt{reflexive}}
  = \\
& \hspace{0.9em}  \{ ix^m \mid \forall m, \, ix : \mathbb{Z}^{\mathbb{D}}, d : \mathbb{D}
  \, . \, d \in \textit{ds} \Rightarrow ix \, d = 0 \} \\[0.4em]
& \interp{\texttt{forward} \;\; \texttt{depth}=n \;\;
  \texttt{dims}=\textit{ds}}  = \;\; (\texttt{t} \not\in ds) \\
& \hspace{0.9em}  \{ ix^m \mid \forall m, \, ix : \mathbb{Z}^{\mathbb{D}}, d : \mathbb{D}
  \, . \, d \in \textit{ds} \Rightarrow 0 \leq (ix \, d) \leq n \} \\[0.4em]
& \interp{\texttt{backward} \;\; \texttt{depth}=n \;\; \texttt{dims}=\textit{ds}}
  = \; \; (\texttt{t} \not\in ds) \\
& \hspace{0.9em}  \{ ix^m \mid \forall m, \, ix : \mathbb{Z}^{\mathbb{D}}, d : \mathbb{D}
  \, . \, d \in \textit{ds} \Rightarrow (-n) \leq (ix \, d) \leq 0 \}
  \\[0.4em]
& \interp{\texttt{read-once} \; S} = \{ ix^1 \mid ix^m \in \interp{S}
  \} 
\end{array}
\end{array}
\end{align*}
\textit{$a^m \in A$ means there are $n$ copies of $a$ in
  the multi-set $A$}
\caption{Multi-set model of specifications}
\label{fig:model}
\end{figure}

\subsection{Equational theory and approximations}
\label{sec:eqs}

Figure~\ref{fig:equations}

\begin{figure}
\begin{align*}
\hspace{-1em}
\setlength{\arraycolsep}{0.05em}
\begin{array}{c}
\framebox{$\overline{\texttt{v} : S} \equiv \overline{\texttt{u} : T}$} \\[1em]
\begin{array}{rl}
(\textsc{coalesceF}) \;\; & (\texttt{a} : \texttt{fwd} \;
                       \texttt{dims}=ds \; \texttt{depth}=n; \\
&  \;\, \texttt{a} :  \texttt{fwd} \; \texttt{dims}=ds \; \texttt{depth}=n+1)\\
\equiv \; & \texttt{a} :  \texttt{fwd} \; \texttt{depth}=n+1,
         \texttt{dims}=ds \\[0.75em]
(\textsc{coalesceB}) \;\; & (\texttt{a} : \texttt{bwd} \;
                       \texttt{dims}=ds \; \texttt{depth}=n; \\
&  \;\, \texttt{a} :  \texttt{bwd} \; \texttt{dims}=ds \; \texttt{depth}=n+1) \\
\equiv \; & \texttt{a} :  \texttt{bwd} \; \texttt{depth}=n+1,
         \texttt{dims}=ds \\[0.75em]
(\textsc{idem}) \;\; & (\texttt{a} : S; \texttt{a} : S) \equiv (\texttt{a}
                  : S)
\end{array} \\ \\
\framebox{$\texttt{v} : S \equiv \texttt{v'} : S'$} \\[1em]
\begin{array}{ll}
(\texttt{a} : \; \texttt{bwd} \; \texttt{depth=1\{b\}} \;
  \texttt{dim=t})
\, & \equiv \,
(\texttt{b} : \; \texttt{bwd} \; \texttt{depth=1\{a\}} \;
  \texttt{dim=t})
\\ 
(\texttt{a} : \; \texttt{fwd} \; \texttt{depth=1\{b\}}\;
  \texttt{dim=t})
\, & \equiv \,
(\texttt{b} : \; \texttt{fwd} \; \texttt{depth=1\{a\}} \;
  \texttt{dim=t})
\end{array}
\\ \\
\framebox{$S \equiv S'$} \\[0.75em]
\begin{array}{ll}
(a : \texttt{read-once} \; S) \; & \equiv \; (a : \texttt{read-once} \;\,
  \texttt{read-once} \; \, S)
\end{array}
\end{array}
\end{align*}
\caption{Equations on specifications}
\label{fig:equations}
\end{figure}


\begin{theorem}[Soundness]
\[
\overline{\texttt{v} : S}\equiv \overline{\texttt{u} : T}
\; \Rightarrow \;
\interp{\overline{\texttt{v} : S}} = \interp{\overline{\texttt{u} : T}}
\]
\end{theorem}

\paragraph{Proof} (see Appendix)


\subsubsection{Safety and sub-specifications}

Figure~\ref{fig:inequations}

\begin{figure}[t]
{\framebox{
\begin{minipage}{0.88\linewidth}
\begin{align*}
\hspace{-0.7em}
\begin{array}{c}
\dfrac{}{S <: S}(\textsc{refl}) \qquad \dfrac{R <: S \quad S <: T}{R <:
  T}(\textsc{trans}) \\[1.5em]
\dfrac{}{S <: \; \texttt{unspecified dims=}ds}(\top) 
\\[1.5em]
\dfrac{}{\texttt{read-once} \, S <: S}(\textsc{rep})
\\[1.5em]
\setlength{\arraycolsep}{0.1em}
\dfrac{\hspace{3em} ds \subseteq es \; \wedge \; n \leq m \hspace{3em}}
{\begin{array}{rl}
\texttt{fwd} \; \texttt{depth=}n \; \texttt{dim=}ds & <: \texttt{fwd} \;
  \texttt{depth=}m \; \texttt{dim=}es 
%& \;\, ds \subseteq es \wedge n \leq m 
\\
\wedge \; \texttt{bwd} \; \texttt{depth=}n \; \texttt{dim=}ds & <: \texttt{bwd} \;
  \texttt{depth=}m \; \texttt{dim=}es 
%& \;\, ds \subseteq es \wedge n \leq m 
\\
\wedge \; \texttt{fwd} \; \texttt{depth=}n \; \texttt{dim=}ds & <: \texttt{sym} \;
  \texttt{depth=}m \; \texttt{dim=}es 
%& \;\, ds \subseteq es \wedge n \leq m 
\\
\wedge \; \texttt{bwd} \; \texttt{depth=}n \; \texttt{dim=}ds & <: \texttt{sym} \;
  \texttt{depth=}m \; \texttt{dim=}es 
\end{array}}
%{\footnotesize{(\textsc{cover})}}
%\hspace{-0.1em}
\end{array}
\end{align*}
\end{minipage}}}
\caption{Definition of sub-specification relation}
\label{fig:inequations}
\end{figure}


\begin{theorem}[Soundness]
\[
\overline{\texttt{v} : S} <: \overline{\texttt{u} : T}
\; \Rightarrow \;
\interp{\overline{\texttt{v} : S}} \subseteq \interp{\overline{\texttt{u} : T}}
\]
\end{theorem}

\paragraph{Proof} (see Appendix)



\section{Examples and common patterns}

\begin{example}[FTCS (Forward time, centered space)]

  Otherwise known as the \emph{explicit method}, is a common
  pattern since it tends to be more efficient than other approches
  (see BTCS and CTCS below). However, its disadvantage is that it is
  unstable if the resolution is not well chosen. 

\begin{ExmVerbatim}
b(x) = a(x) + r*(a(x-1) - 2*a(x) + a(x+1))
\end{ExmVerbatim}
%
For this statement, the inference provides the specification: 
%
\begin{SpecVerbatim}
a: centered depth=1, dims=0
\end{SpecVerbatim}



\end{example}

\paragraph{BTCS (Backward time, centered space) -- implicit method}

\paragraph{CTCS (Centered time, centered space) - Crank-Nicolson method}

\section{Specification inference and checking}
\label{sec:analysis}

\mnote{Few cases that we might like to consider (there are examples
    of each in \#camfort-main Slack channel): 
  \begin{itemize}
    \item Matrix operator overloading $a + b$, where both are arrays
    \item Scalar operator overloading $a + b$, where a is an array b is a 
      scalar. $b$ is added to all.
    \item Sliding. $a = b$ where the dimensionsare $a(1:10)$, $b(2:11)$,
      so the assignment causes shifting.
  \end{itemize}
}

In the following, we defined
\emph{base induction variables} to be the variables
 defined by a ``for'' loop (\eg{}, that get incremented each time the
loop is re-entered). 

\begin{definition}[Constant translations]
An indexing expression $e$ is a \emph{constant translation} if,
for a base induction variable $i$, then $e \equiv i + a$ or $e \equiv i - a$ 
where $a$ is a constant. The relation $\equiv$ identifies terms
up-to commutativity of $+$ and the inverse
relation of $+$ and $-$ (\eg{}, $(-b) + i \equiv i - b$). 
In the following, we classify constant translation expressions 
using the predicate $\textsf{trans}$.
\end{definition}

\paragraph{Step 1: Data-access analysis}

\dnote{Might want to do 'stencil' (gather) and 'mould' (scatter)
  patterns separately}
\begin{align*}
\begin{array}{lll}
\interp{e_1 = e_2}    & = \interp{e_2} \\
\interp{e_1 e_2}      & = \interp{e_1} \cup \interp{e_2} \\
\interp{a(\tilde{e})} & = \{e | e \in \tilde{e}, \textsf{trans}(e)\} & \textit{where each $e \in
                                          \tilde{e}$ is a constant or constant translation}
\end{array}
\end{align*}

Index analysis is a \emph{coeffect analysis}.

\paragraph{Step 2: Coalesce contiguous indices into regions}

% fairly mechanical

\subsection{Infering temporal specifications}

The reserved dimension \dimId{t}

\section{Evaluation}
\label{sec:evaluation}

\section{Discussion}
\label{sec:discussion}


\bibliography{references}


\onecolumn
\appendix

\section{Correctness}


\begin{theorem}[Soundness]
\[
\overline{\texttt{v} : S}\equiv \overline{\texttt{u} : T}
\; \Rightarrow \;
\interp{\overline{\texttt{v} : S}} = \interp{\overline{\texttt{u} : T}}
\]
\end{theorem}

\paragraph{Proof}
\begin{itemize}
\item \textsc{coalesceF}
\begin{align*}
\llbracket & \texttt{v} : \texttt{fwd} \; \texttt{dims}=ds \; \texttt{depth}=n; \,
  \texttt{v} :  \texttt{fwd} \; \texttt{dims}=ds \;
  \texttt{depth}=n+1)\rrbracket \\
& =  \{ (\texttt{v}, ix)^m \mid \forall \, m, ix : \mathbb{Z}^{\mathbb{N}}, d : \mathbb{N}
  \, . \, d \in \textit{ds} \Rightarrow ix \, d \leq (n+1) \} \; \cup \; \{ ix^m \mid \forall \, m, ix : \mathbb{Z}^{\mathbb{N}}, d : \mathbb{N}
  \, . \, d \in \textit{ds} \Rightarrow ix \, d \leq n \}\\
& =  \{ (\texttt{v}, ix)^m \mid \forall \, m, ix : \mathbb{Z}^{\mathbb{N}}, d : \mathbb{N}
  \, . \, d \in \textit{ds} \Rightarrow ((ix \, d \leq (n+1)) \vee (ix \,
  d \leq n)) \} \\
& =  \{ (\texttt{v}, ix)^m \mid \forall \, m, ix : \mathbb{Z}^{\mathbb{N}}, d : \mathbb{N}
  \, . \, d \in \textit{ds} \Rightarrow ix \, d \leq (n+1) \} \\
& = \interp{\texttt{v} : \texttt{fwd} \; \texttt{dims}=ds \;
  \texttt{depth}=n + 1} \qquad \Box
\end{align*}
\end{itemize}

\end{document}
