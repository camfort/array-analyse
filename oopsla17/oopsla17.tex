\documentclass[acmlarge,review,anonymous,10pt]{acmart}\settopmatter{printfolios=true}

%% Some recommended packages.
\usepackage{booktabs}   %% For formal tables:
                        %% http://ctan.org/pkg/booktabs
\usepackage{subcaption} %% For complex figures with subfigures/subcaptions
                        %% http://ctan.org/pkg/subcaption

\usepackage{hyperref}
\usepackage{natbib}
\usepackage{amsmath}
\usepackage{amssymb}
\usepackage{amsthm}
\usepackage{syntax}
\usepackage{indentfirst}
\usepackage{cleveref}
\usepackage{xcolor}
\usepackage{multirow}
\usepackage{siunitx} % For pretty-printing numeric values and SI units
                     % of measure. e.g., the tabular column type S is
                     % used to print nice-looking tables of numbers.
\sisetup{ % defaults
  group-separator={,},
  group-minimum-digits={3},
  output-decimal-marker={.},
  table-format = 6
}

\usepackage{minted}
\usepackage{etoolbox}
\usepackage{enumitem}
\usepackage{thmtools}
\usepackage{thm-restate}

\newcommand{\fortran}[1]{\mintinline{fortran}{#1}}

\errorcontextlines 10000
\usepackage{stencilmacros}

%% Bibliography style
\bibliographystyle{ACM-Reference-Format}
%% Citation style
%% Note: author/year citations are required for papers published as an
%% issue of PACMPL.
\citestyle{acmauthoryear}   %% For author/year citations


\fvset{
  linenos=true,
  fontsize=\footnotesize,
  breaklines=true,
  breakafter=w),
  xleftmargin=3\parindent,
  numbersep=1em
}

\usemintedstyle{vs}

\setlength{\grammarindent}{3em} % increase separation between LHS/RHS

\newcommand{\inbar}{\ \ \textbar\ \ }

% For evluation and empirical study
% Defines the commands (headers)
\newcommand{\overallLoC}[0]{0}
\newcommand{\overalllinesParsed}[0]{0}
\newcommand{\numPackages}{9}
\newcommand{\SSArpackFiles}{312}
\newcommand{\SSArpackLoC}{50208}
\newcommand{\SSArpackLoCRaw}{144081}

\newcommand{\SSArpackFilesParsed}{252}
\newcommand{\SSArpackLoCParsed}{42391}
\newcommand{\SSArpackLoCParsedRaw}{127231}

%%

\newcommand{\SSBlasFiles}{151}
\newcommand{\SSBlasLoC}{16046}
\newcommand{\SSBlasLoCRaw}{40882}

\newcommand{\SSBlasFilesParsed}{149}
\newcommand{\SSBlasLoCParsed}{15993}
\newcommand{\SSBlasLoCParsedRaw}{40679}

%%

\newcommand{\SSCliffsFiles}{30}
\newcommand{\SSCliffsLoC}{2424}
\newcommand{\SSCliffsLoCRaw}{3149}

\newcommand{\SSCliffsFilesParsed}{30}
\newcommand{\SSCliffsLoCParsed}{2424}
\newcommand{\SSCliffsLoCParsedRaw}{3149}

%%

\newcommand{\SSCPFiles}{52}
\newcommand{\SSCPLoC}{2334}
\newcommand{\SSCPLoCRaw}{3978}

\newcommand{\SSCPFilesParsed}{48}
\newcommand{\SSCPLoCParsed}{2121}
\newcommand{\SSCPLoCParsedRaw}{3632}

%%

\newcommand{\SSEThreeMEFiles}{167}
\newcommand{\SSEThreeMELoC}{44935}
\newcommand{\SSEThreeMELoCRaw}{73545}

\newcommand{\SSEThreeMEFilesParsed}{154}
\newcommand{\SSEThreeMELoCParsed}{39700}
\newcommand{\SSEThreeMELoCParsedRaw}{62238}

%%

%OLDER
%\newcommand{\SSgeosChemFiles}{604}
%\newcommand{\SSgeosChemLoC}{467472}
%\newcommand{\SSgeosChemLoCRaw}{848647}

%\newcommand{\SSgeosChemFilesParsed}{196}
%\newcommand{\SSgeosChemLoCParsed}{213387}
%\newcommand{\SSgeosChemLoCParsedRaw}{263694}


\newcommand{\SSgeosChemFiles}{604}
\newcommand{\SSgeosChemLoC}{445660}
\newcommand{\SSgeosChemLoCRaw}{848647}

\newcommand{\SSgeosChemFilesParsed}{336}
\newcommand{\SSgeosChemLoCParsed}{269674}
\newcommand{\SSgeosChemLoCParsedRaw}{410654}


%%

\newcommand{\SShybridFiles}{29}
\newcommand{\SShybridLoC}{4831}
\newcommand{\SShybridLoCRaw}{8361}

\newcommand{\SShybridFilesParsed}{29}
\newcommand{\SShybridLoCParsed}{4831}
\newcommand{\SShybridLoCParsedRaw}{8361}

%%

\newcommand{\SSmudpackFiles}{88}
\newcommand{\SSmudpackLoC}{54753}
\newcommand{\SSmudpackLoCRaw}{78652}

\newcommand{\SSmudpackFilesParsed}{88}
\newcommand{\SSmudpackLoCParsed}{54753}
\newcommand{\SSmudpackLoCParsedRaw}{78652}

%%

\newcommand{\SSnavierFiles}{6}
\newcommand{\SSnavierLoC}{505}
\newcommand{\SSnavierLoCRaw}{696}

\newcommand{\SSnavierFilesParsed}{6}
\newcommand{\SSnavierLoCParsed}{505}
\newcommand{\SSnavierLoCParsedRaw}{696}

%%

\newcommand{\SSspecfemFiles}{555}
\newcommand{\SSspecfemLoC}{137468}
\newcommand{\SSspecfemLoCRaw}{232356}

\newcommand{\SSspecfemFilesParsed}{475}
\newcommand{\SSspecfemLoCParsed}{103328}
\newcommand{\SSspecfemLoCParsedRaw}{178317}


%%

\newcommand{\SSumFiles}{2540}
\newcommand{\SSumLoC}{635525}
\newcommand{\SSumLoCRaw}{1010936}

\newcommand{\SSumFilesParsed}{2269}
\newcommand{\SSumLoCParsed}{541540}
\newcommand{\SSumLoCParsedRaw}{866406}

%% Totals

\newcommand{\SSFiles}{\the\numexpr(\SSArpackFiles+\SSBlasFiles+\SSCliffsFiles+\SSCPFiles+\SSEThreeMEFiles+\SSgeosChemFiles+\SShybridFiles+\SSnavierFiles+
\SSmudpackFiles +\SSspecfemFiles+\SSumFiles)}

\newcommand{\SSLoC}{\the\numexpr(\SSArpackLoC+\SSBlasLoC+\SSCliffsLoC+\SSCPLoC+\SSEThreeMELoC+\SSgeosChemLoC+\SShybridLoC+\SSmudpackLoC+\SSnavierLoC+\SSspecfemLoC+\SSumLoC)}

\newcommand{\SSLoCRaw}{\the\numexpr(\SSArpackLoCRaw+\SSBlasLoCRaw+\SSCliffsLoCRaw+\SSCPLoCRaw+\SSEThreeMELoCRaw+\SSgeosChemLoCRaw+\SShybridLoCRaw+\SSmudpackLoCRaw+\SSnavierLoCRaw+\SSspecfemLoCRaw+\SSumLoCRaw)}

\newcommand{\SSFilesParsed}{\the\numexpr(\SSArpackFilesParsed+\SSBlasFilesParsed+\SSCliffsFilesParsed+\SSCPFilesParsed+\SSEThreeMEFilesParsed+\SSgeosChemFilesParsed+\SShybridFilesParsed+\SSnavierFilesParsed+\SSmudpackFilesParsed+\SSspecfemFilesParsed+\SSumFilesParsed)}

\newcommand{\SSLoCParsed}{\the\numexpr(\SSArpackLoCParsed+\SSBlasLoCParsed+\SSCliffsLoCParsed+\SSCPLoCParsed+\SSEThreeMELoCParsed+\SSgeosChemLoCParsed+\SShybridLoCParsed+\SSnavierLoCParsed+\SSmudpackLoCParsed+\SSspecfemLoCParsed+\SSumLoCParsed)}

\newcommand{\SSLoCParsedRaw}{\the\numexpr(\SSArpackLoCParsedRaw+\SSBlasLoCParsedRaw+\SSCliffsLoCParsedRaw+\SSCPLoCParsedRaw+\SSEThreeMELoCParsedRaw+\SSgeosChemLoCParsedRaw+\SShybridLoCParsedRaw+\SSnavierLoCParsedRaw+\SSmudpackLoCParsedRaw+\SSspecfemLoCParsedRaw+\SSumLoCParsedRaw)}

% renews the commands with actual data
% \input{results}

\definecolor{darkgreen}{rgb}{0.0,0.5,0.0}
\definecolor{darkpurple}{rgb}{0.6,0.0,0.6}
\definecolor{orange}{rgb}{0.8,0.4,0.0}
\newcommand{\dnote}[1]{\textcolor{darkpurple}{Dom: #1}}
\newcommand{\mnote}[1]{\textcolor{darkgreen}{Mistral: #1}}
\newcommand{\anote}[1]{\textcolor{red}{Andy: #1}}

\theoremstyle{definition}
\newtheorem{defn}{Definition}

\theoremstyle{plain}
\newtheorem{thm}{Theorem}
\newtheorem{lem}{Lemma}
\newtheorem{prop}{Proposition}
\newtheorem{case}{Case}

\theoremstyle{remark}
\newtheorem{remark}{Remark}

\title{Verifying Spatial Properties of Array Computations}

\author{Dominic Orchard}
\affiliation{
  \department{School of Computing}
  \institution{University of Kent}
}
\email{d.a.orchard@kent.ac.uk}


\author{Mistral Contrastin}
\affiliation{
  \department{Computer Laboratory}
  \institution{University of Cambridge}
}
\email{Mistral.Contrastin@cl.cam.ac.uk}

\author{Matthew Danish}
\affiliation{
  \department{Computer Laboratory}
  \institution{University of Cambridge}
}
\email{Matthew.Danish@cl.cam.ac.uk}


\author{Andrew Rice}
\affiliation{
  \department{Computer Laboratory}
  \institution{University of Cambridge}
}
\email{Andrew.Rice@cl.cam.ac.uk}


\begin{abstract}
  Arrays computations are at the core of numerical modelling and
  computational science applications. However, low-level manipulation of array
  indices is a source of program error.  Many practitioners are aware
  of the need to ensure program correctness, yet very few of the
  techniques from the programming research community are applied by
  scientists. We aim to change that by providing targetted lightweight
  verification techniques for scientific code.  We focus on the all
  too common mistake of array offset errors as a generalisation of
  off-by-one errors.  Firstly, we report on a code analysis study on
  twelve real-world computational science code base, identifying
  common idioms of array usage and their spatial properties. This
  provides much needed data on array programming idioms common in
  scientific code.  From this data, we designed a lightweight
  declarative specification language capturing the majority of array
  access patterns via a small set of combinators. We detail a semantic
  model, and the design and implementation of a verification tool for
  our specification language, which both checks and infers
  specifications.  We evaluate our tool on our corpus of scientific
  code and give verification case studies of bug fixes that are
  detected by our approach. We found roughly 60,000 targets for
  specification across roughly 1.4 million lines of code, showing the
  vast majority of array computations read from arrays in a pattern with
  a simple, regular, static shape.
\end{abstract}
  
%  We focus on one error-prone aspect of numerical code: the 
%  data access pattern of arrays. 

\begin{document}
\maketitle

\section{Introduction}
\label{sec:introduction}
\paragraph{Context of this research}

There is an increasing awareness of the need for program verification
techniques in
science~\cite{post2005computational,oberkampf2010verification,orchard2014computational}.
Whilst there are various kinds of automated and semi-automated
verification tools available, they see little use in science. This is
partly due to a lack of training and a lack of awareness. But its also
due to a lack of targetting. This paper is part of a line of research
to provide lightweight verification tools targetting
programming patterns common in computational science code, 
that are easy to use by scientists, and that integrate with common
practises and workflows.


\subsection{Terminology and notation}

\begin{defn}[Induction variables]
  An integer variable is a \emph{base induction variable} if it is the control
  variable of a ``\texttt{for}'' loop, incremented by $1$ per iteration. A variable
  is interpreted as an induction variable only within the scope of the loop
  body. Throughout, $i, j, k$ range over induction variables.

  A \emph{derived induction variable} is an expression of the form $a \times i
  + b$, where $a$ and $b$ are constant expressions, \ie{}, an affine expression
  on an induction variable $i$.
\end{defn}

\begin{defn}[Array subscripts and indices]\label{def:array-subs}
  An \emph{array subscript}, denoted $a(\bar{e})$, is an expression which reads
  from an $N$-dimensional array $a$ at an \emph{index} specified by a
  comma-separated sequence of integer expressions denoted $\bar{e}$ or in
  expanded form as $(e_1, \ldots, e_N)$. An index $e_i$ is called
  \emph{relative} if the expression involves an induction variable. An
  \emph{absolute index} is a integer expression which is constant relative to
  the enclosing loop.
\end{defn}

\begin{defn}[Neighbourhood index]\label{def:neighbour-ix}
  For an array subscript $a(\bar{e})$ an index $e \in \bar{e}$ is a
  \emph{neighbourhood index} if $e$ is of the form $e \equiv i$, $e \equiv i +
  c$, or $e \equiv i - c$, where $c$ is an integer constant. That is, a
  neighbourhood index is a constant translation of an induction variable. (The
  relation $\equiv$ identifies terms up-to commutativity of $+$ and the inverse
  relationship of $+$ and $-$ \eg{}, $(-b) + i \equiv i - b$).
\end{defn}

%%% Local Variables:
%%% mode: latex
%%% TeX-master: t
%%% End:


\section{An Empirical Study of Array Computations in Scientific Fortran code}
\label{sec:study}
At the beginning of this project, we had the following hypotheses:
\begin{enumerate}
\item Loops over arrays mainly read from arrays with a static pattern
based on constant offsets from (base or dervied) induction variables;

\item Most loop-array computations of the previous form read
from a arrays with a \emph{contiguous} pattern, \eg{}:
%
\begin{minted}{fortran}
do i = 1, n
  b(n) = (a(i) + a(i+1)) / 2.0
\end{minted}
%
\item Most loop-array computations of the previous form read
from arrays with a pattern that includes the immediate
neighbours (offsets of 1 from the induction variables);

\item Many array computations are \emph{stencil computations},
with a static access pattern as described in Hypothesis 2, writing
to an array at an index based on a (possibly constant offset from induction
variables, \eg{}, the classic five-point stencil;

\item Many array computations read from each particular index just once.
\end{enumerate}
%
From these hypotheses, we conjectured that the common programming
patterns captured above could be specified declaratively with a small
set of combinators capture the shape of data access patterns based on
overlapping (hyper)rectangles. We performed a large scale source-code
analysis to validate these hypotheses and guide the design of our
language. We believe these results are also of wider value.
\dnote{Explain why: what else do they tell us?}

\subsection{Methodology}
%
\noindent
We constructed a corpus of \numPackages{} scientific computing
packages written in Fortran ranging in size and scope: The Unified
Model (UM)~\cite{um}, E3MG~\cite{RePEc:aen:journl:2006se-a12},
BLAS~\cite{blas}, Hybrid4~\cite{GBC:GBC635},
GEOS-Chem~\cite{geos-chem}, Navier (based on
\cite{griebel1997numerical}), Computational Physics 2 (CP2)
~\cite{nicholas2006computational}, ARPACK-NG~\cite{arpackng}, and
SPECFEM3D~\cite{specfem3d}. \dnote{Include the rest} This covers
approximately 1.4 million lines of physical Fortran code (2.4 million
including comments and white space).  Appendix~\ref{app:corpus}
provides further detail on these packages and their sizes, with UM and
GEOS-Chem the largest at $\approx$630kloc and $\approx$450kloc respectively. We used
Wheeler's \emph{SLOCCount} to get a count of the physical source lines
(excluding comments and blank lines)~\cite{wheeler2001sloccount}.

Half of the packages come from active research teams who are our
project partners (\ie{}, they were not selected carefully to skew
results). BLAS was selected as it is a common numerical library;
similarly CP2 was selected as it gives standard numerical
analysis algorithms from a popular computational physics textbook.
The smallest project Navier ($\approx$500loc) is based on a standard textbook on
computational fluid dynamics.

We built a code analysis tool based on CamFort, an open-source analysis
tool for Fortran programs~\cite{camfort}. Fortran source files are
parsed to an AST and various standard control-flow and data-flow
analyses are computed, including of key importance for us: induction
variable identification, and reaching definitions. The resulting AST
is traversed top-down and assignment statements inside loops are
analysed and classified. 

\paragraph{Classifications}

We focus on assignment statements where
the right-hand side reads from one or more arrays. Left-hand
sides and right-hand sides are classified in the following way:

\begin{tabular}{l|l|l}
  Shorthand & Classification (of subscripts) & Example \\ \hline
 (\textsf{vars}) & Variables (just left-hand side) & \fortran{x = ...} \\
 (\textsf{const}) & Constants & \fortran{a(0, 1)} \\
 (\textsf{IVs}) & Induction variables & \fortran{a(i, j)} \\
 (\textsf{neigh}) & Neighbourhood offsets (of the form
                                                  $i \pm c$) & \fortran{a(i, j-1)} \\
 (\textsf{neigh+c}) & Neighbourhood offsets and constants &
                                                            \fortran{b(i, 0, j+1)} \\
 (\textsf{aff}) & Affine offsets (of the form $a * i \pm b$) & 
                                     \fortran{x(2*i+1,j)} \\
 (\textsf{aff+c}) & Affine offsets and constants & \fortran{x(i+1, 0, 3*j+2)}
\end{tabular} \\[1em]

\noindent
The classification is computed from all information that flows into an
assignment, for example, the following decomposes a one-dimensional
three-point pattern across multiple intermediate assignments:
%%
\begin{minted}{fortran}
do i = 1, n
  x = a(i)
  y = a(i+1)
  b(i) = (a(i-1) + x + y)/3.0
end do
\end{minted}
%%x
Our analysis recognises this as one array computation (rather than three), starting at
line 4, and reading from subscripts \fortran{a(i)}, \fortran{a(i+1)},
\fortran{a(i-1)}. We thus traverse the body of a loop in reverse,
using reaching-definitions to calculate of the array reads that flow
to a particular assignment. Since \fortran{x} and \fortran{y} flow to
the line 4, line 2 and line 3 are not classified, though multiple
assignments can flow to multiple classified statements.

We further sub-classify sets of array subscripts
on the right-hand side based on their spatial properties: \\

\begin{tabular}{l||l|p{0.28\linewidth}|l}
  Property   & Shorthand & Classifications (of RHS pattern) & Example \\ \hline
%%%%%%
\multirow{2}{*}{Contiguity} & (\textsf{contig}) & Contiguous  &
                                                                \fortran{a(i) + a(i+1) + a(i+2)} \\
  & (\textsf{disjoint}) & Non-contiguous & \fortran{a(i) + a(i+2)} \\ \hline
%%%%%%
\multirow{2}{*}{Shape} & (\textsf{rect}) & (Hyper)rectangle &
 \fortran{a(i,j) + a(i+1,j) + a(i,j+1) + a(i+1,j+1)} \\
           & (\textsf{sumRect}) & Composed (hyper)rectangles &
  \fortran{a(i,j) + a(i,j-1) + a(i-1,j) + a(i+1,j) + a(i,j+1)} \\ \hline
%%%%%%
  \multirow{2}{*}{Reuse} & (\textsf{readonce}) & Unique subscripts
                                         & \eg{} \fortran{b(i) = a(i) + a(i+1)} \\
  & (\textsf{mult}) & Repeated subscripts &
\eg{}  \fortran{b(i) = a(i)  + a(i)}
\end{tabular} \\[1em]

\noindent
To assess the hypothesis that 
arrays are mostly read in a pattern that includes the immediate neighbours
to the ``origin'', we categorised the position of subscript patterns for each
right-hand side, in each dimension: \\

\begin{tabular}{l||l|p{0.33\linewidth}|l}
  Property   & Shorthand & Classification (indices per
                           dimension) & Example \\ \hline
%%%%%%
  \multirow{2}{*}{Positioning} & (\textsf{origin}) & Includes origin
& \fortran{a(i)} or \fortran{a(i+1, j)} (in $2^{\textit{nd}}$
  dimension) \\
             & (\textsf{straddle}) & Within distance 1 of origin &
\fortran{a(i+1), a(i-1)} \\
             & (\textsf{away}) & Away from the origin
                                          & \fortran{a(i+2), a(i+3)}
\end{tabular} \\[1em]

\noindent
Finally, we categorised the relationship between the left-hand side
and right-hand side, in terms of the use of induction variables.
The two sides are \emph{consistent} if the same induction variables
appear in each side, used for the same dimension. This is
weakened to a \emph{permutation} if the roles of the induction
variables changes. This is weakened further if the left-hand side
induction variables are either a 
a subset or superset of the induction variables on the right-hand
side. Otherwise, the two sides are seen as inconsistent:

\begin{tabular}{l|l|l}
  (\textsf{const}) & Consistent & \fortran{a(i, j) = b(i, j) +
                                  b(i+1,j+1)} \\
  (\textsf{perm} & Permutation & \fortran{a(i, j) = c(j, i)} or \fortran{a(i,
                                                               0) =
                                 b(0, i)} \\
  (\textsf{LHSsub} & LHS subset & \fortran{a(i) = b(i, j) + b(i, j-1)}
  \\
  (\textsf{LHSsup} & LHS superset & \fortran{a(i, j) = b(i)} \\
  (\textsf{inconst} & Inconsistent & \fortran{a(i) = b(j)}
\end{tabular} \\[1em]
%


%\begin{itemize}
% (Vars, ...)
%\item An assignment to a variable, with an RHS comprising some array
%  computation, tends to correspond to a reduction (e.g., calculating
%the max value in an array).
%\end{itemize}

\subsection{Results}

\paragraph{Addressing the hypotheses}

\begin{enumerate}
\item Loops over arrays mainly read from arrays with a static pattern
based on constant offsets from (base or dervied) induction variables;
\end{enumerate}

%\begin{tabular}
%Neighbourhood or Constant Affine RHS &  Total 
%\end{tabular}


\section{A Specification Language for the Shape of Array Patterns}
\label{sec:lang}

\begin{figure}[t]
\begin{align*}
\def\arraystretch{1}
\setlength{\arraycolsep}{0.2em}
\newcommand{\dimTy}{$\mathbb{N}_{>0}$}
\newcommand{\depthRange}{$\mathbb{N}_{>0}$}
\begin{array}{rl}
\nonterm{specification} ::= & \nonterm{regionDec} \mid \nonterm{specDec} \\
\nonterm{specDec} ::= & \term{stencil} \; \nonterm{spec} \;
                        \texttt{::} \; v \\
\nonterm{regionDec} ::= &  \texttt{region} \; \texttt{::} \; \nonterm{rvar} \; \texttt{=} \;
                         \nonterm{region}\\[0.4em]
%\nonterm{spec} ::= & \nonterm{spatial} \mid \nonterm{temporal}
%\\[1em]
\nonterm{spec} ::= & [\nonterm{mult},] \; [\nonterm{approx},] \; \nonterm{region} \\
\nonterm{mult} ::= & \term{readOnce} \\
\nonterm{approx} ::= & \term{atMost} \; \mid \; \term{atLeast} \\[0.1em]
\nonterm{region} ::= & \nonterm{rvar} \; \mid \;
                       \stencil{p}{\dimTy}{}{} \; \\
& {\qquad\; \mid \; \stencil{f}{\dimTy}{\depthRange}{\;[, \nonpointed]}} \\
& {\qquad\; \mid \; \stencil{b}{\dimTy}{\depthRange}{\;[, \nonpointed]}} \\
& {\qquad\; \mid \; \stencil{c}{\dimTy}{\depthRange}{\;[, \nonpointed]}} \\
& {\qquad\; \mid \; \nonterm{region} \, \term{+}
  \, \nonterm{region} \; \mid \; \nonterm{region} \; \term{*} \; \nonterm{region}} \\[0.25em]
%\multicolumn{2}{l}{\qquad\qquad \mid \; \nonterm{rvar}}
%\\[0.5em]
%\nonterm{temporal} ::= \; & \term{dependency} \; (v \; \{ , v \}) [, \texttt{mutual}]
%  \\[0.5em]
\nonterm{rvar} ::= \; & [\text{\term{a}-\term{z}$\,$\term{A}-\term{Z}$\,$\term{0}-\term{9}}]+\\[-1em]
\end{array}
\end{align*}
\caption{Specification syntax (EBNF grammar)}
\label{fig:syntax}
\end{figure}

\subsection{Specification syntax}
\label{subsec:syntax}

\Cref{fig:syntax} gives the syntax of stencil specifications, which is
detailed below. The entry point is the \nonterm{specification} production which
splits into either a \emph{region declaration} or a \emph{specification
declaration}.
Regions comprise \emph{region constants} which are
combined via region operators \term{+} and \term{*}.

Region constants specify a finite interval in a 
single dimension relative to the origin and are either \term{pointed},
\term{forward}, \term{backward}, or \term{centered}. The region names are
inspired by numerical analysis terminology, \eg{} the standard explicit method for
approximating PDEs is known as the \emph{Forward Time, Centered Space} (FTCS)
scheme~\citep{dawson1991finite}.

Each region
constant has a dimension identifier $d$ given as a positive natural number.
Each constant except \term{pointed} has a depth
parameter $n$ given as a positive natural number; \term{pointed}
regions implicitly have a depth of $0$.

A \term{forward} region of depth $n$ specifies a contiguous
region in dimension $d$ starting at the origin. This corresponds
to specifying neighbourhood indices in dimension $d$ ranging from $i$ to $i + n$
for some induction variable $i$. Similarly, a
\term{backward} region of depth $n$ corresponds to contiguous indices
from $i$ to $i - n$ and \term{centered}
of depth $n$ from $i - n$ to $i + n$. A \term{pointed}
stencil specifies a neighbour index $i$. For example, the
following shows four specifications with four consistent stencil
kernels reading from arrays \term{a}, \term{b}, \term{c} and \term{d}:
%%
\begin{minted}{fortran}
!= stencil forward(depth=2, dim=1) :: a
e(i, 0) = a(i, 0) + a(i+1, 0) + a(i+2, 0)

!= stencil backward(depth=2, dim=1) :: b
e(i, j) = b(i, j) + b(i-1, j) + b(i-2, j)

!= stencil centered(depth=1, dim=1) :: c
e(i, j) = (c(j-1) + c(j) + c(j+1))/3.0

!= stencil pointed(dim=3) :: d
e(i, j) = d(0, 0, i)
\end{minted}
%%
Not every dimension needs to be specified, \eg{},
specifications on lines $1$, $4$, and $10$ leave some dimensions unspecified.
The \term{forward}, \term{backward}, and \term{centered} regions may
all have an additional attribute \term{nonpointed} which marks absence
of the origin.  For example, the following is a
\term{nonpointed} \term{backward} stencil
%
\begin{minted}{fortran}
!= stencil backward(depth=2, dim=1, nonpointed) :: a
b(i) = a(i-1) + 10*a(i-2)
\end{minted}
%
Not every dimension needs to be specified, \eg{},
specifications on lines $1$, $3$, and $7$ leave some dimensions
unspecified which leaves indices in these dimensions unconstrained.

The \term{forward}, \term{backward}, and \term{centered} regions may
all have an additional attribute \term{nonpointed} which marks absence
of the origin.  For example, the following is a
\term{nonpointed} \term{backward} stencil
%
\begin{minted}{fortran}
!= stencil backward(depth=2, dim=1, nonpointed) :: a
b(i) = a(i-1) + 10*a(i-2)
\end{minted}

\paragraph{Combining regions}

The region operators \term{+} and \term{*} respectively combine
regions by union and intersection. The intersection of two regions
$r \term{*} s$ means that any indices in the specified code must be
consistent with both $r$ and $s$ simultaneously.
 Dually, for the union of two regions
 $r \term{+} s$ means that indices in the specified code must be
 in consistent with one of $r$ or $s$, or both.
For example, the following \emph{nine-point stencil}
has a specification given by the product of two \texttt{centered}
regions in each dimension:
%%
\begin{minted}[breakindent=2.9em]{fortran}
x = a(i, j)   + a(i-1, j)   + a(i+1, j)
y = a(i, j-1) + a(i-1, j-1) + a(i+1, j-1)
z = a(i, j+1) + a(i-1, j+1) + a(i+1, j+1)
!= stencil centered(depth=1, dim=1) * centered(depth=1, dim=2) :: a
b(i, j) = (x + y + z) / 9.0
\end{minted}
%
%This pattern is common in image convolution applications.
The specification ranges over the
values that flow to the array subscript on the left-hand side,
and so ranges over the intermediate assignments to \term{x},
\term{y}, and \term{z}. Each index in the code is consistent
with both specifications simultaneously, \eg{}, \texttt{a(i-1, j+1)}
is within the centered region in dimension $1$ and the centered region
in dimension $2$.

The union of two regions $r \term{+} s$ means that any indices
in the specified code must be consistent with either of $r$ or $s$.
For example, the following gives the specification of a five-point
stencil which is the sum of two compound \texttt{pointed} and
\texttt{centered} regions in each dimension:
%
\begin{minted}{fortran}
!= stencil centered(depth=1, dim=1)*pointed(dim=2) + centered(depth=1, dim=2)*pointed(dim=1) :: a
b(i,j) = -4*a(i,j) + a(i-1,j) + a(i+1,j) + a(i,j-1) + a(i,j+1)
\end{minted}
%%
Here the left-hand side of \texttt{+} says that when the second dimension
(induction variable $j$) is fixed at the origin, the first dimension
(induction variable $i$) accesses the immediate vicinity of the origin
(to depth of one). The right hand side of \texttt{+} is similar but the dimensions are reversed.
This reflects the symmetry under rotation of the five-point stencil.

\paragraph{Region declarations and variables}

Region specifications can be assigned to region variables via
region declarations. For example, the shape of a
``\emph{Roberts cross}'' edge-detection convolution~\cite{davis1975survey}
can be stated:
%%
\begin{minted}{fortran}
!= region :: r1 = forward(depth=1, dim=1)
!= region :: r2 = forward(depth=1, dim=2)
!= region :: robertsCross = r1*r2
!= stencil robertsCross :: a
\end{minted}
This is useful for common patterns, such as the five-point
pattern, as the regions can be defined once and reused.
%%
\paragraph{Modifiers}
%%
Region specifications can be modified
by \emph{approximation} and \emph{multiplicity} information
(in \textit{spec} in \Cref{fig:syntax}).
The \texttt{readOnce} modifier enforces that no index appears more
than once (that is, its multiplicity is one). For example, all of
the previous examples could have \texttt{readOnce} added:
%
\begin{minted}{fortran}
!= stencil readOnce, backward(depth=2, dim=1) :: a
b(i+1) = a(i) + a(i-1) + a(i-2)
\end{minted}
%
This specification would be invalid if any of the
array subscripts were repeated. This modifier provides a way to
rule out any accidental repetition of array subscripts.
The notion is similar to that of linear types~\cite{wadler1990linear}, where a value must be used
exactly once. We opt for the more informative and easily understood name
\texttt{readOnce}. This modifier is optional, so it need not
be present even if the stencil is linear.

In some cases, it is useful to give a lower and/or upper bound for a
stencil. This can be done using either the \term{atMost} or
\term{atLeast} modifiers. This is particularly useful in situations
where there is a non-contiguous stencil pattern, which cannot be expressed
precisely in our syntax. For example:
%
\begin{minted}{fortran}
!= stencil atLeast, pointed(dim=1)         :: a
!= stencil atMost, forward(depth=4, dim=1) :: a
b(i) = a(i) + a(i+4)
\end{minted}

%%% Local Variables:
%%% mode: latex
%%% TeX-master: t
%%% End:


\section{Equational \& Approximation Theories}
\label{sec:theory}
We now introduce the equational and approximation theory of stencil
specifications based on $\equiv$ and $\preceq$ in the following two subsections.

\subsection{Equivalences}

We define an equivalence relation, $\equiv$. The purpose of this relation is to
allow programmers to write specifications with greater flexibility. It allows
specifications to be written in various levels of compactness allowing
optimising for space and better documentation. The relation is defined on
regions as follows:

\begin{description}
  \item[Basic] \texttt{*} and \texttt{+} are both idempotent, commutative, and
    associative;
%
  \item[Subsumption] If $S$ and $R$ are regions with $S \preceq R$, then
    $S \texttt{+} R \equiv R$ and $S \texttt{*} R \equiv S$.
%
  \item[Dist] \texttt{*} distributed over \texttt{+} and dually
    \texttt{+} distributes over \texttt{*}, meaning if \textcap{R}, \textcap{S},
    and \textcap{T} are regions, then we have the following dual equivalences:
%
    \begin{align*}
      \texttt{\textcap{R}*(\textcap{S}+\textcap{T})} & \equiv
        \texttt{(\textcap{R}*\textcap{S})+(\textcap{R}*\textcap{T})} &
      \texttt{\textcap{R}+(\textcap{S}*\textcap{T})} & \equiv
        \texttt{(\textcap{R}+\textcap{S})*(\textcap{R}+\textcap{T})}
    \end{align*}
%
  \item[Overlapping pointed] If \textcap{R} is one of \texttt{forward},
    \texttt{backward}, or \texttt{centered}, then we have
%
    \begin{equation*}
      \stencil{\textcap{R}}{$n$}{$k$}{\texttt{nonpointed}} \;\texttt{+}\;
      \stencil{p}{$n$}{}{} \equiv
      \stencil{\textcap{R}}{$n$}{$k$}{}
    \end{equation*}
%
  \item[Centered] The region constants \texttt{forward} and \texttt{backward}
    are two halves of \texttt{centered} specifications:
%
    \begin{align*}
      \stencil[s]{c}{$n$}{$k$}{\textcap{p1}} \equiv
        \stencil[s]{f}{$n$}{$k$}{\textcap{p2}} \texttt{+}
        \stencil[s]{b}{$n$}{$k$}{\textcap{p3}}
    \end{align*}
%
    Here \textcap{p1} is \texttt{nonpointed} if both \textcap{p2} and
    \textcap{p3} are \texttt{nonpointed} and \texttt{pointed} otherwise.
\end{description}

\subsection{Approximations}

We define a partial of order of approximations, $\preceq$. The relation is used
in equational theory and provides means of writing more compact lower and upper
bound specifications. The relation is defined as follows:

\begin{description}
  \item[Equivalence] If $S$ and $R$ are regions and $S \equiv R$, then we have
    $S \preceq R$.
%
  \item[Combined] If $S$ and $R$ are regions, then we have
    $S \preceq S \texttt{+} R$ and $S \texttt{*} R \preceq S$.
%
  \item[Depth] Let $k$ and $l$ be in positive integers and $k \leq l$, $n$ some
    fixed dimension, and \textcap{p} either \texttt{pointed} or
    \texttt{nonpointed}. Further, let \textcap{R} be one of \texttt{centered},
    \texttt{forward}, and \texttt{backward}. We then have
%
    \begin{equation*}
      \stencil{R}{$n$}{$k$}{\textcap{p}} \preceq \stencil{R}{$n$}{$l$}{\textcap{p}}
    \end{equation*}
%
\end{description}

We present few inequalities that can be derived from the axioms and are useful
when writing specifications without inference:

\begin{restatable}[Centered approximation]{prop}{centeredApprox}
  For any dimension $n$, depth $k$, and pointed attribute $p$,
  we have
%
  \begin{align*}
    \stencil{f}{$n$}{$k$}{\textcap{p}} & \preceq
      \stencil{c}{$n$}{$k$}{\textcap{p}} \\
%
    \stencil{b}{$n$}{$k$}{\textcap{p}} & \preceq
      \stencil{c}{$n$}{$k$}{\textcap{p}}
  \end{align*}
\end{restatable}

\begin{restatable}[Point approximation]{prop}{pointApprox}
  Let \textcap{R} be one of \texttt{forward}, \texttt{backward}, and
  \texttt{centered}, $n$ a fixed dimension, and $k$ a fixed depth, then we have
%
  \begin{align*}
    \stencil{p}{$n$}{}{} & \preceq \stencil{R}{$n$}{$k$}{} \\
%
    \stencil{R}{$n$}{$k$}{\texttt{nonpointed}} & \preceq
      \stencil{R}{$n$}{$k$}{}
  \end{align*}
\end{restatable}


\section{Semantic Model}
\label{sec:model}
\newcommand{\singleEntry}[2]{\textbf{J}_{#2}^{#1}}
\begin{align*}
%
% REGION model
%
\interp{\stenFwd{k}{d}}^r_n & =
 \{i\singleEntry{d}{n} \mid i \in \{0, \ldots, k\} \} \\
\interp{\stenBwd{k}{d}}^r_n & =
 \{i\singleEntry{d}{n} \mid i \in \{-k, \ldots, 0\} \} \\
\interp{\stenCen{k}{d}}^r_n & =
 \{i\singleEntry{d}{n} \mid i \in \{-k, \ldots, k\} \} \\
%
% REGION PROD model
%
\interp{r_1 \; \texttt{$\ast$} \, \ldots \, \texttt{$\ast$} \; r_m}^{\ast}_n &
= \interp{r_1}^r_n \otimes \ldots \otimes \interp{r_m}^r_n \\
%
%  REGION SUM model
%
\interp{r_1 \; \texttt{+} \, \ldots \, \texttt{+} \; r_m}^{+}_n &
= \interp{r_1}^{\ast}_n \cup \ldots \cup \interp{r_m}^{\ast}_n \\
\end{align*}
%where $\singleEntry
TODO: insert explanation for the $\textbf{J}$ notation (see Matrix
textbook).



\section{Analysis, Checking, and Inference}
\label{sec:algorithms}
\noindent
We outline here the procedures for checking conformance
of source code against specifications (\Cref{subsec:checking})
and for inferring specifications from code (\Cref{subsec:inference}).
Both rely on a program analysis that converts array subscripts
 into sets of index schemes. We outline this analysis
first (\Cref{subsec:analysis}). Note that the analysis
can be made arbitrarily more complex and wide-ranging independent
of the checking and inference procedures. At the moment, the analysis
is largely \emph{syntactic}, with only a small amount of
semantic interpretation of the code.

\begin{example}
\label{exm:checking}
We demonstrate analysis, checking, and inference on the 
five-point stencil example:
%%
\begin{minted}{fortran}
b(i, j) = (a(i, j) + a(i-1, j) +a (i+1, j) + a(i, j-1) + a(i, j + 1)) / 5.0
\end{minted}
\end{example}

\subsection{Analysis of array accesses}
\label{subsec:analysis}

\newcommand{\neigh}{\textsf{neigh}}
\noindent
The analysis builds on standard program analyses:
%
\textbf{(1)} basic blocks (CFG);
\textbf{(2)} induction variables per basic block;
\textbf{(3)} (interprocedural) data-flow analysis, providing a \emph{flows to}
  graph (as shorthand, the function
  $\mathit{flowTo}$ is used, implicitly parameterised by this graph,
  mapping an expression to the set of all expressions
  with forwards data-flow to this expression, based on the transitive
  closure of the flows graph);
\textbf{(4)} type information per variable.% where we use the predicate
%\arrayTy{} to classify variables of array type.

The analysis traverses the control-flow graph of a program top-down
and traverses statements inside of loops bottom-up. The right-hand
side of any assignment statement in a loop is classified based on
whether it has array subscripts flowing to it which are neighbourhood
offsets (or potential absolute in some dimensions). These are then
converted into our model domain by the interpretation
$\interp{-}^{\mathit{aterm}}$ (\Cref{defn:subscript}) and grouped
into a finite map from array variables to set of subscript schemes.
We denote sets of index/subscripts schemes by
$\mathcal{S}$.

For our example, the set of subscript schemes from the analysis is:
%
\begin{equation*}
\mathcal{S}_0 = \{\{0\} \times \{0\}, \{-1\} \times \{0\},
\{0\} \times \{-1\}, \{0\} \times \{1\}, \{1\} \times \{0\}\}
\end{equation*}
%
This set of index schemes is then augmented with multiplicity
information (\textsf{only} or \textsf{mult}) depending on whether
subscripts are unique or not in the analysed statement.
Thus, for each assignment, the analysis generates a map from array
variables to values in $\mathsf{Mult}(\mathcal{P}(\bz)^N)$.

Any assignment statement from which array subscripts flow to the
current assignment is marked as visited such that the main analysis
does not classify it as the root of an array computation.  Thus,
combined with the bottom-up traversal inside of loops, we assign
specifications only to the leaves of dataflow paths in a loop.

\subsection{Checking code against specifications}
\label{subsec:checking}

\noindent
Checking verifies the access pattern of an array computation in the
source language against any associated specifications. Checking
proceeds by generating a model from a specification and generating a
model from the source code (above), and comparing them for
consistency.  Since the model of \Cref{sec:model} interprets both
array indices and specifications as sets of points in $N$-dimensional
space. The notion of consistency is then intuitively whether these two
(potentially infinite) sets of points are equal. This leads to a
simple notion of consistency, where $\mathit{consistent}(M_C, M_S)$
tests the consistency of a modal $M_C$ of source code against
a model $M_S$ of a specification:
%
\begin{align*}
  \mathit{consistent} & :
    \mathsf{Mult}(\mathcal{P}(\bz{}^N)) \times
    \mathsf{Mult}(\mathsf{Approx}(\region{N})) \to \mathbb{B} & \\
%
  \mathit{consistent}(\mathit{ix}, \mathit{model}) & = \begin{cases}
    \mathsf{false} & \mathit{model} = \mathsf{once}(x) \wedge ix =
    \mathsf{mult}(y) \\
    \mathsf{false} & \mathit{model} = \mathsf{mult}(x) \wedge ix = \mathsf{once}(y) \\
    \mathit{consistent'}(\mathit{peel}(ix), \mathit{peel}(model)) & \textit{otherwise}
  \end{cases} \\
%
  \mathit{consistent'} & :
    \mathcal{P}(\bz{}^N) \times
    \mathsf{Approx}(\region{N}) \to \mathbb{B} & \\
%
  \mathit{consistent'}(\mathit{ix}, \mathit{model}) & = \begin{cases}
    m = \mathit{ix} & \mathit{model} = \mathsf{exact}(m) \\
    m \supseteq \mathit{ix} & \mathit{model} = \mathsf{upper}(m) \\
    m \subseteq \mathit{ix} & \mathit{model} = \mathsf{lower}(m)
  \end{cases}
\end{align*}
%
The first function, $\mathit{consistent}$, checks whether the linearity of the
specification matches that of the indices, \ie{} if the specification allows
indices to be repeated or not. Then it delegates to $\mathit{consistent'}$
function to check if the points observed in the array terms match the space
defined by the specification. Lower bounds, marked by the \texttt{atLeast}
modifier, require the space defined by the specification to remain inside the
set of indices, while an upper bound, market by \texttt{atMost},
requires the opposite: enclosure. In the absence of such modifiers, we expect
the space defined by observed indices corresponds exactly with those defined by the
region in the specification, hence requiring set equality.

As explained in \Cref{subsec:union-normal-form}, the sets being compared are
potentially infinite thus set comparison cannot be done by exhaustively
comparing the points of one with the other. Instead, we compile the region into
interval constraints and subscript schemes into membership constraints and pass
these to the \textsc{Z3} SMT solver~\citep{de2008z3} to see if they are equal. The
query is expressed in quantifier free linear arithmetic, which is decidable.

Although satisfiability is super-exponential with the length of the formula in
the worst case~\cite{fischer1974super}, performance of consistency checking
using satisfiability is fast in practice as the length of the formula is linearly
related with two factors: (1) dimensionality times number of regions connected
with \texttt{+} and (2) dimensionality times the number of array terms flowing
into an assignment assignment. In \Cref{subsec:additional-data}, we
established that  99.6\% of array computations have 
dimensionality of at most four, and 97\% of array computations
involve at most 4 array subscript terms. In
\Cref{sec:evaluation}, we show the number of regions that are connected with
\texttt{+} is Z in 99\% of the cases.


\subsection{Inferring specification automatically}
\label{subsec:inference}
%
\noindent
We provide an inference procedure for generating specifications from
code, which are inserted automatically as comments. This helps support
maintenance of legacy code, and aids understanding of our
specification language. We illustrate inference using the
five-point stencil of \Cref{exm:checking}.

The preceding source code analysis converts the concrete syntax of
array subscripts into a set of subscript schemes in
$\mathcal{P}(\bz)^N$. These subscripts are then coalesced iteratively
into sets of index schemes.

%
%For each array variable $a \in \mathsf{dom}(U)$, the algorithm
%constructs a set of $n$-dimensional rectangles covering all contiguous
%groups of schemes in $U(a)$.
%
We explain the algorithm in two parts. First, we show how
\emph{contiguous} index schemes are coalesced into a smaller set of
index schemes while remaining in union normal form. Then, we explain
how this smaller unioned indexing schemes are converted to
specifications.

\subsubsection{Covering}
A covering of (possibly overlapping) intervals is calculated as a
value of \region{N} by coalescing \emph{contiguous} index schemes
until a fixed-point is reached.

\newcommand{\contig}[2]{\mathit{contig}(#1, #2)}
\begin{defn}[Contiguity]
  \label{def:contiguity}
  Two index schemes $S$ and $T$ are \emph{contiguous} written
  $\contig{S}{T}$, iff $\pi_i(S) = \pi_i(T)$ for all $1 \leq i \leq N$
  apart from some dimension $k$ such that $\pi_k(S) = [a, b]$ and
  $\pi_k(T) = [b+1, c]$.
\end{defn}
%
A coalesced set of index schemes is computed by:%
%
\begin{align*}
  \mathit{coalesce}(\mathcal{S}) = \; & \;\;
  \{\,S \cup T \mid S \in \mathcal{S}, T \in \mathcal{S} \, \wedge \,
                                        \contig{S}{T} \}
  \;\;\; \cup \;\;\;
  \{\,S \in \mathcal{S} \mid \neg \exists T \in \mathcal{S} . \; \contig{S}{T}\}
\end{align*}
%
Thus, $\mathit{coalesce}(\mathcal{S})$
comprises the union of the set of coalesced pairs of contiguous
schemes and the set of index schemes for which there is no other
contiguous scheme.
%
\begin{lemma}[Closure]
  For a set of index schemes $\mathcal{S}$, then
  $\mathit{coalesce}(\mathcal{S})$ is well-defined and is a set of
  index schemes.
\end{lemma}
\begin{proof}
  By \Cref{lem:vector-union}, the union of two index schemes is an
  index scheme if all components are equal except at most one
  component. Contiguity (\Cref{def:contiguity}) matches
  this precondition, thus the first part of the union in
  $\mathit{coalesce}$ produces a set of index schemes. The second part
  comprises a subset of the original index schemes.
\end{proof}
%
For our example, the fixed-pointed of $\mathit{coalesce}$ is reached within two
steps:
%
\begin{align*}
  \mathit{coalesce}(\mathcal{S}_0) & =
  \{[-1,0] \times [0,0],\,[0,0] \times [-1, 0],\,[0,0] \times
    [0,1],\,[0,1] \times [0,0]\} \\
  \mathit{coalesce}^2(\mathcal{S}_0) & =
  \{[-1, 1] \times [0, 0],\,[0, 0] \times [-1, 1]\} = \mathit{coalesce}^3(\mathcal{S}_0)
\end{align*}
%
\todo{Prove that this is in $Region_N$}

\subsubsection{Index schemes to syntax}

Now we convert coalesced indexing schemes into specifications. This happens in
three stages. First, we see if there is an all infinite indexing scheme. Second,
we determine if the indexing schemes are interval schemes so that they can be
represented exactly and if not how it can be altered into interval schemes to be
represented as approximations. Third, we split up the interval schemes so that
the region constants can express them. Finally, we convert the interval schemes
into a joins of meets of region constants.

If there is all infinite indexing scheme, then we cannot represent it
syntactically. This happens when use of constant indices in one or more array
terms causes the offsets to be subsumed. The simplest case is when there is an
array term with all of its indices constant.

In our example, there are no constant indices, so we proceed.

After handling the trivial case, correct approximation is chosen. Recall
\Cref{def:interval-scheme}, an indexing scheme is an interval scheme if each of
the sets in products are holed intervals. If that is the case, we trivially
convert the sets in each dimension of indexing schemes to holed interval
representation and proceed to the next stage and output an exact specification
at the end. If not, then the output will definitely have an upper bound and
potentially a lower bound.

Upper bound is established by \emph{elongating} index schemes in multiple
dimensions such that they become interval schemes. The elongation function is
given as
%
\begin{align*}
  \mathit{elongate}([a,b]) = \begin{cases}
    \interv{0}{b}{\textsf{false}} & a > 1 \\
    \interv{a}{0}{\textsf{false}} & b < -1
  \end{cases}
\end{align*}

Since index schemes of the previous stage are all contiguous and form closed
intervals, 0 lying between the lower and upper bounds of intervals is sufficient
to form an interval scheme. If any dimension of an index scheme is elongated,
then the whole index scheme is elongated. In the initial set of index schemes,
all schemes that were already interval schemes form a lower bound, while those
schemes and the new interval schemes generated by elongation form the upper
bound. The reason a lower bound may not be generated is all indexing schemes
with might need elongation.

The index schemes in the example are all interval schemes, so we proceed to the
next stage noting the output will be exact.

There is no direct translation from an interval scheme such as $\interv{-2}{1}{}
\times \interv{-1}{1}{}$ to region constants. The reason is region constants
\texttt{forward} and \texttt{backward} require upper and lower bounds to be 0
respectively and \texttt{centered} requires the upper bound to be the negation
of the lower bound. The solution is to use \Cref{lem:vector-union} to break up
interval schemes into two producing $\interv{-2}{0}{} \times \interv{-1}{1}{}$
and $\interv{0}{1}{} \times \interv{-1}{1}{}$.

The extended example does not have such representation problem, so the interval
schemes are left as they are.

The final stage of inference is simply mapping interval schemes into region
constants. Each interval scheme is connected with \texttt{+}. Individual schemes
are decomposed using \Cref{lem:vector-intersect} such that each of the resulting
index schemes have all but one dimensions set to \interv{-\infty}{\infty}{}.
Each of these interval schemes are mapped to region constants using
\Cref{subfig:region-model} and connected with \texttt{*}.

In the example, $\interv{-1}{1}{} \times \interv{0}{0}{}$ is decomposed into
$\interv{-1}{1}{} \times \interv{-\infty}{\infty}{}$ and
$\interv{-\infty}{\infty}{} \times \interv{0}{0}{}$. These are then mapped to
\texttt{\stencil{c}{1}{1}{} * \stencil{p}{2}{}{}}. Decomposition of
$\interv{0}{0}{} \times \interv{-1}{1}{}$ is similar. When combined we obtain the
following specification:
%
\begin{equation*}
  \texttt{stencil readOnce,} \;
  \stencil[s]{c}{1}{1}{} \texttt{*} \stencil[s]{p}{2}{}{} \; \texttt{+} \;
  \stencil[s]{p}{1}{}{}  \texttt{*} \stencil[s]{c}{2}{1}{}
\end{equation*}


\section{Evaluation}
\label{sec:evaluation}
\input{results}

To study the effectiveness of our approach, 
we built a corpus of over 1 million lines of Fortran code from a
range of scientific computing packages: The Unified Model (UM)~\cite{um},
E3ME~\cite{RePEc:aen:journl:2006se-a12}, BLAS~\cite{blas},
Hybrid4~\cite{GBC:GBC635}, GEOS-Chem~\cite{geos-chem}, Navier (based
on \cite{griebel1997numerical}), Computational Physics
ed. 1~\cite{giordano1997computational},
ARPACK-NG~\cite{arpackng}, MUDPACK~\cite{MUD}, Cliffs~\cite{Cliffs}, and
SPECFEM3D~\cite{specfem3d}.

We first examined how frequently stencil computations occur. We parsed
\num{\overalllinesParsed} lines of Fortran code and found that \overalltickAssignPercent\%
(\num{\overalltickAssign}) of statements have a left-hand side as an
array subscript on neighbourhood indices. This supports the idea that
stencil-like computations are common in scientific code.  We then used
the inference procedure of the previous section to generate
specifications for stencils in the corpus to assess the design of the
language.


\section{Related Work}
\label{sec:related-work}
\noindent
Various deductive verification tools can express array indexing in their
specifications, \eg{}, ACSL of \citet{baudin2008acsl} for C
(\eg{}~\citet[Example 3.4.1]{burghardt2010acsl}). A specification can be given
for a stencil computation but must use fine-grained indexing as in code and
therefore is similarly prone to indexing errors. Our approach is much more
abstract---it does not aim to reify indexing in the specification, but
provides simple spatial descriptions which capture a large number of common
patterns.

\citet{kamil2016verified} propose \emph{verified lifting} to extract a
functionally-complete mathematical representation of low-level, and potentially
optimised, stencils in Fortran code. This extracted predicate representation of
a stencil is used to generate code for the \textsc{Halide} high-performance
compiler~\citep{ragan2013halide}. Thus they must capture the full meaning of a
stencil computation which requires significant program analysis. For example,
they report that some degenerate stencil kernels take up to 17 hours to analyse
and others require programmer intervention for correct invariants to be
inferred. By contrast, it takes roughly 1.5 hours on commodity
hardware (3.2Ghz Intel Core i5, 16 Gb of RAM) to infer and generate
stencil specifications for our entire corpus.

% ACR-I think we could skip this in the interests of space
%This led them to develop \textsf{STNG}, a loop invariant and postcondition
%finder through syntax-guided synthesis. Using this approach they restrict the
%search space of postconditions to predicates of a certain form, one that is
%compatible with \textsf{Halide}. They then look at concrete loop iterations and
%form an hypothesis about the invariant, which is later confirmed or put back
%into the system using a SMT solver.

Our approach differs significantly. Rather than full representation of
array computations, we focus on specifying just the spatial behaviour
in a lightweight way.  Thus, it suffices for us to perform a
comparatively simple data-flow analysis which is efficient, scales
linearly with code size, and does not require any user intervention.
Whilst we do not perform deep semantic analysis of stencils, the
analysis part of our approach can be made arbitrarily more
sophisticated independent of the rest of the work.
% Hence, we do not require SMT solving or search of loop invariants. It
%suffices to do comparatively simple data-flow analysis.
%Since CamFort mostly does
%syntactic analysis, the inference procedure terminates quickly and
%never requires programmer intervention.
%The downside of our
%approach is that if optimisation causes the stencil computation to be heavily
%obfuscated, \textsf{STNG} would capture the access behaviour better.
%
Furthermore, Kamil \emph{et al.} do not provide a user-visible syntactic
representation of their specifications, and nor do they provide verification
from specifications \eg{}, to future-proof the code against later changes. Even
if they were to provide a syntactic representation, for complex stencils such as
Navier-Stokes from \Cref{sec:introduction}, it would be as verbose as the code
itself, making it difficult for programmer to understand the overall shape of
the indexing behaviour.

% GPGPU
Our work has similarities with efforts to verify kernels written for
General-Purpose GPU programming, such as in \citet{Blom:2014:SoCP}.
%Stencils are a form of
%kernel, and GPGPU programming can be viewed as a massively parallel
%method of transforming a large matrix.
However, their focus is mainly on the synchronisation of kernels and the
avoidance of data races., while we are interested in correctness
%embedded within a more typical general-purpose programming
%language. %Other work on GPGPU computation, such as
%\citet{Zhang:2012:CGO}, has focused primarily on generating
%optimised code based on relatively simple specifications: to
%provide performance while keeping the programmer's effort within
%reason.
%
% Sketching Stencils
\citet{Solar-Lezama:2007:PLDI} give specifications of stencils using unoptimised
``reference'' stencils, coupled with partial implementations which are completed
by a code generation tool. %These kinds of specifications are just simple
%implementations, so this tool is useful for hand-written, optimised
%stencils.
The primary purpose of this tool is optimisation rather than correctness, and
the language of specification is more elaborate than ours.

% Pochoir
\citet{Tang:2011:SPAA} define a specification language for writing stencils
embedded in C++ (with Cilk~\citep{blumofe1996cilk} extensions) that are then
compiled into parallel programs based on trapezoidal decompositions with
hyperspace cuts. Pochoir specifications are used for describing the kernel,
boundary conditions, and shape of the stencil. Pochoir is aimed at programmers
reluctant to implement the high-performance cache-oblivious ``hypertrapezoidal''
algorithms.  Like much of the related work, the goal is optimisation rather than
correctness.

By contrast, the work of \citet{Abe:2013:IPDPSW} studies correctness bringing a
form of model-checking to verify certain stencil computations in the context of
parallelism in partitioned global address
space languages. %These are scenarios where an array is divided into
%subarrays on multiple processors but each has the ability to access
%the others' memory. The authors avoid the state explosion dilemma of
%model-checking by relying upon the fact that accesses to the
%``boundary elements'' will require a different method by virtue of the
%non-locality of that memory. Since this access is performed in a
%different manner, those can be identified automatically by static
%analysis. To achieve this,
Abe \emph{et al.} provide a new language for writing stencil computations. Much
of the specification effort goes towards describing the distribution of the
computation over multiple processors. The code for the stencil kernel is
generated from a relatively high-level specification.  In contrast, we integrate
directly into existing, legacy codebases and established languages, bringing
the benefits of verification more easily to scientific computing.%such as
%older versions of Fortran.% We infer stencil specifications from
%Fortran code and check annotations on stencil computations within
%Fortran code.


\bibliography{references}

\appendix
% http://2017.splashcon.org/track/splash-2017-OOPSLA#Instructions-for-Authors
% "There is no page limit for bibliographic references and appendices,
% and, therefore, for the overall submission."
% http://2017.splashcon.org/track/splash-2017-OOPSLA#Instructions-for-Authors
% "There is no page limit for bibliographic references and appendices,
% and, therefore, for the overall submission."

\section{Details of the corpus data set}
\label{app:corpus}

\paragraph{Software Corpus}
Table~\ref{tab:corpus} shows summary statistics of the software
packages used in our evaluation, all of which are written in Fortran
90 or Fortran 77. In total we analysed \SI{\overallLoC} lines of code
from \numPackages{} packages, of which we successfully parsed
\SI{\overalllinesParsed} lines. The ``Number of files'' column shows
how many files in each corpus that we were able to analyse with
CamFort. The most common reasons for CamFort rejecting a file were
either use of a C-preprocessor, or illegal use of language features
from a modern Fortran variant.

\begin{enumerate}
\item \textbf{The Unified Model}~\cite{um} is a weather
  forecasting and climate modelling tool developed by the Met Office
  in the United Kingdom. It is used by research organisations and
  meteorological services around the world. We use the development
  branch (trunk) of the model. The code base in closed source but
  institutional licenses are available for research purposes. The Met
  Office runs a comprehensive code quality system incorporating
  dedicated committers (we counted 11) for particular parts of the
  model. We counted 120 additional contributors whose submissions are
  reviewed and tested before being accepted into the code base.

\item \textbf{E3MG} (An Energy-Environment-Economy (E3) Model at the Global
Level) is a macroeconomic model used for assessment of environmental
policy~\cite{RePEc:aen:journl:2006se-a12}. This was developed by
Cambridge Econometrics, an independent consultancy company.

\item \textbf{BLAS}~\cite{blas} (Basic Linear Algebra Subprograms) is
  a popular library providing efficient and portable routines for
  vector and matrix operations. These routines feature in many other
  libraries (including LAPACK). We used version 3.6.0. We chose to
  include this package for breadth, as it provides general numerical
  functions rather than a specialised scientific model.

\item \textbf{Hybrid4} is a vegetation and biomass model for
  simulating carbon, water and nitrogen flows~\cite{GBC:GBC635}.

\item \textbf{GEOS-Chem}~\cite{geos-chem} is a three-dimensional model
  of tropospheric chemistry developed at Harvard and used by $\sim$70
  universities and research institutions world-wide. We use v.10-01.

\item \textbf{Navier} is a small numerical simulation, giving a
  discrete approximation to the two-dimensional Navier-Stokes fluid
  equations, based on the book of~\citet{griebel1997numerical}.

\item \textbf{CP} consists of the example code from the second edition
  of the book ``Computational
  Physics''~\cite{nicholas2006computational} introducing numerical
  techniques and their application to modern physics problems such as
  fields, waves, statistical mechanics and quantum mechanics.

\item \textbf{ARPACK-NG}~\cite{arpackng} \todo{add a sentence of detail}

\item \textbf{SPECFEM3D}~\cite{specfem3d} \todo{add a little detail}
\end{enumerate}

\section{Proofs and extended definitions for the model}
\label{sec:proofs}

\newcommand{\off}{S}
\newcommand{\offP}{T}

\vectorIntersect*

\begin{proof}
  \begin{align*}
    \off \; \cap \; \offP &
    = \{x \mid \bigwedge_{1 \leq i \leq N } x_i \in \pi_i(\off) \}
      \;\cap\;
      \{x \mid \bigwedge_{1 \leq i \leq N } x_i \in \pi_i(\offP) \} \\
    & = \{x \mid \bigwedge_{1 \leq i \leq N }
      (x_i \in \pi_i(\off) \wedge x_i \in \pi_i(\offP)) \} \\
    & = \prod_{i = 1}^{N} \pi_i(\off) \cap \pi_i(\offP)
  \end{align*}
\end{proof}

\vectorUnion*

\begin{proof}
  \begin{align*}
    \off \; \cup \; \offP &
    = \{x \mid
      \bigwedge_{1 \leq i \leq N } x_i \in \pi_i(\off) \}
      \;\cup\;
      \{x \mid
          \bigwedge_{1 \leq i \leq N } x_i \in \pi_i(\offP) \} \\
    & = \{x \mid
          \bigwedge_{\substack{1 \leq i \leq N \\ i \neq k}}
            x_i \in \pi_i(\off) \wedge x_k \in \pi_k(\off) \vee
          \bigwedge_{\substack{1 \leq i \leq N \\ i \neq k}} x_i \in
            \pi_i(\offP) \wedge x_k \in \pi_k(\offP)
        \} \\
    & = \{x \mid
          \bigwedge_{\substack{1 \leq i \leq N \\ i \neq k}} x_i \in
            \pi_i(\off) \wedge
            x_k \in \pi_k(\off) \cup \pi_k(\offP)
        \} \\
        & = \pi_1(\off) \times \cdots \times
        (\pi_k(\off) \cup \pi_k(\offP)) \times \cdots \times
        \pi_N(\off)
  \end{align*}
\end{proof}

\intervalIdentities*

\begin{proof}
  We give the proof of the first identity and the second one is similar.
  \begin{align*}
    \interv{a}{b}{c} \cap \interv{d}{e}{f} = &
      \; \{ n \;|\; a \leq n \leq b \wedge (\neg c \implies n \neq 0) \}
      \;\cap \\
      & \; \{ n \;|\; d \leq n \leq e \wedge (\neg f \implies n \neq 0) \}
      \\
    = & \; \{ n \;|\; \max \{a,d\} \leq n \leq \min \{b,e\} \wedge (\neg c
      \vee \neg f \implies n \neq 0) \} \\
    = & \; \interv{\max{\{a,d\}}}{\min{\{b,e\}}}{c \wedge f}
  \end{align*}
\end{proof}

\regionLattice*

\begin{proof}
  Straightforward, the join and meet are mapped to $\cup$ and $\cap$, the set is
  inductively designed to be closed under these operations. Union and
  intersection are associative, commutative, and absroptive under closed sets
  and then they are also for \region{N}. This is enough to show that, it is a
  lattice.

  Further, we have $\bz{}^N \cap R = R$ and $\emptyset \cup R = R$ with
  $\bz{}^N$ and $\emptyset$ belonging to \region{N}. This makes the lattice a
  bounded one.

  Finally, the lattice is distributive since union distributes over intersection
  and vice versa when the set is closed under these operations.
\end{proof}



\end{document}
