%
\begin{dgroup*}
\begin{dmath*}
  \mathit{cons_n}(\textsf{once}(A_s),\textsf{once}(M_c)) =
    {\mathit{cons_n}(\textsf{mult}(A_s),\textsf{mult}(M_c)) }
\end{dmath*}
\begin{dmath*}
  \mathit{cons_n}(\textsf{mult}(\textsf{up}(M_s)),\textsf{mult}(M_c)) =
    {\forall u \in M_c, \exists v \in M_s .  u \preceq_n v }
\end{dmath*}
\begin{dmath*}
  \mathit{cons_}(\textsf{mult}(\textsf{low}(M_s)),\textsf{mult}(M_c)) =
    {\forall v \in M_s, \exists u \in M_c . u \preceq_n v }
\end{dmath*}
\begin{dmath*}
  \mathit{cons_n}(\textsf{mult}(\textsf{exact}(M_s)),N_c) =
    {\mathit{cons_n}(\textsf{mult}(\textsf{low}(M_s)),N_c) \wedge
     \mathit{cons_n}(\textsf{mult}(\textsf{up}(M_s)),N_c) }
\end{dmath*}
\begin{dmath*}
  \mathit{cons_n}(\textsf{mult}(\textsf{both}(M_1,M_2)),N_c) =
    {\mathit{cons_n}(\textsf{mult}(\textsf{low}(M_1)),N_c) \wedge
     \mathit{cons_n}(\textsf{mult}(\textsf{up}(M_2)),N_c) }
\end{dmath*}
\end{dgroup*}
