s\noindent
In the sciences, complex models are now almost always expressed as
computer programs. But how can a scientist have confidence that the
implementation of their model is as they intended? There is an
increasing awareness of the need for program verification in science
and the possibility of using (semi-)automated
tools~\cite{post2005computational,oberkampf2010verification,orchard2014computational}.
However, whilst program verification approaches are slowly maturing in
computer science they see little use in the natural and physical
sciences. This is partly due to a lack of training and awareness, but
also a lack of tools targeted at the needs of scientists. This
paper is part of a line of research on providing lightweight, easy-to-ue
verification tools targeted at common programming patterns in science,
motivated by analysis of real code.

We focus on one common concept: \emph{arrays}, the core data structure
in numerical modelling code, typically representing discrete
approximations of physical space or data sets. A common programming
pattern, sometimes referred to as the \emph{structured grid}
pattern~\cite{Asanovic2006}, traverses the index space of one or more
arrays via a loop, computing elements of another array or reducing the
elements to a single value. For example, the following Fortran code
computes the one-dimensional discrete Laplace transform approximating
a derivative:
%
\begin{minted}{fortran}
do i = 1, (n-1)
      b(i) = a(i-1) - 2*a(i) + a(i+1)
end do
\end{minted}
%
This is a \emph{stencil} computation, a programming idiom
where the elements of an array at each index $i$ from a
\emph{neighbourhood} of values around $i$ in some input array(s). This
pattern is common in scientific, graphical, and numerical
code, \eg{}, convolutions in image processing, approximations
to differential equations, cellular automata.

Such array computations are prone to error in their indexing
terms. For example, a logical off-by-one-error might manifest itself
as writing $\texttt{a(i)}$ instead of $\texttt{a(i-1)}$ (we revisit an
example we found of this in \Cref{sec:case-studies}).
Errors also arise by simple lexical mistakes when large amounts of
fine-grained indexing are involved in a single expression. For
example, the following snippet from a Navier-Stokes
fluid model~\citet{griebel1997numerical} has two arrays are
read with different data access patterns, across two dimensions, with
dense index-manipulation code:
%%
\begin{minted}[firstnumber=20,xleftmargin=2em]{fortran}
du2dx = ((u(i,j)+u(i+1,j))*(u(i,j)+u(i+1,j))+gamma*abs(u(i,j)+u(i+1,j))*(u(i,j)-u(i+1,j))- &
  (u(i-1,j)+u(i,j))*(u(i-1,j)+u(i,j))-gamma*abs(u(i-1,j)+u(i,j))*(u(i-1,j)-u(i,j))) /(4.0*delx)

duvdy = ((v(i,j)+v(i+1,j))*(u(i,j)+u(i,j+1))+gamma*abs(v(i,j)+v(i+1,j))*(u(i,j)-u(i,j+1))- &
  (v(i,j-1)+v(i+1,j-1))*(u(i,j-1)+u(i,j))-gamma*abs(v(i,j-1)+v(i+1,j-1))*(u(i,j-1)- u(i,j))) / (4.0*dely)

laplu = (u(i+1,j)-2.0*u(i,j)+u(i-1,j))/delx/delx+(u(i,j+1)-2.0*u(i,j)+u(i,j-1))/dely/dely
f(i,j) = u(i,j)+del_t*(laplu/Re-du2dx-duvdy)
\end{minted}
%
This miasma of indexing expressions is hard to read and
prone to simple textual input mistakes, \eg{}, swapping \texttt{-} and
\texttt{+}, missing an indexing term, or transforming the wrong
variable \eg{} \texttt{(i+1,j)} instead of \texttt{(i,j+1)}.

In practice, the typical development procedure for complex stencil
computations involves some ad hoc testing to ensure that no
mistakes have been made \eg{}, by visual inspections on data,
or comparison against manufactured or analytical
solutions~\cite{farrell2010automated}. Such testing is often
discarded once the code is shown correct. This is not the only
information that is discarded. The shape of the indexing pattern
is usually the result of choices made in the numerical-analysis
procedure to discretise some continuous equations. Rarely are these decisions captured in the source code,
yet the derived shape is usually uniform with a
clear and concise description \eg{},
\emph{centered space, of depth 1} referring to indexing
terms \texttt{a(i)}, \texttt{a(i-1)} and
\texttt{a(i+1)}~\cite{recktenwald2004finite}.

To support correct array programming, we propose a simple,
abstract specification language for the data access pattern of
array-loop computations. This provides a way to rule out bugs due to
indexing errors. The language design is informed by an initial
empirical study of array computations in a corpus real-world
scientific code base, totally 1.4 million lines of
code (\Cref{sec:study}). We confirm our initial hypotheses of the
ubiquity of looped array computations, but also that they have a
common form, reading from arrays in a fixed neighbourhood of
contiguous elements with a simple static pattern. From this we
designed a simple set of combinators to express common array patterns (\Cref{sec:lang}).
As an example, the Laplace program is specified in our language by:
%
\begin{minted}[linenos=false]{fortran}
!=  stencil centered(depth=1, dim=1) :: a
\end{minted}
%
That is, \texttt{a} is accessed with a symmetrical pattern in its
first dimension (``centered'') to a depth of one in each direction and
its result contributes to an array
write.  The Navier-Stokes example has two specifications:
%
\begin{minted}[linenos=false,xleftmargin=0em,xrightmargin=0.8em,breakindent=1.9em]{fortran}
!= stencil centered(depth=1,dim=1)*pointed(dim=2) + centered(depth=1,dim=2)*pointed(dim=1) :: u
!= stencil forward(depth=1,dim=1)*backward(depth=1,dim=2) :: v
\end{minted}
%
The specifications requires that \texttt{u}
is accessed with a centered pattern to depth of 1 in both dimensions
(this is known as the \emph{five-point stencil}) and \texttt{v} is
accessed in a neighbourhood bounded forwards to depth of $1$ in the
first dimension and backward to a depth of $1$ in the second
dimension. The specification is relatively small and abstract compared
with the code, with a small number of combinators that \emph{do not
  involve any indexing expressions}, \eg{} \texttt{a(i+1,j-1)}. This
contrasts with other specification approaches, \eg{}, deductive
verification tools such as ACSL~\citet{baudin2008acsl}, where
specifications about array computations must also be expressed in
terms of array-indexing expressions. Thus, any low-level mistakes that
could be made whilst programming complex indexing code could also be
made when writing its specification. Our specifications are
abstract to avoid the error-prone nature of index manipulation; our
specifications are more lightweight to aid its adoption in science.

We implemented a verification tool for our specification language as
an extension to CamFort~\cite{camfort}, an open-source program
analysis tool for Fortran. Specifications are associated to code via
comments, against which the tool checks the code for conformance. This
specify-and-check approach reduces testing effort and future-proofs
against bugs introduced during refactoring and maintenance. A
specification also concisely captures the array access pattern
and provides documentation. We also provide an inference procedure
which efficiently produces specifications with no programmer
intervention, automatically inserting specifications at appropriate
places in the source code. This aids adoption of our approach to
existing legacy code base.

The checking and inference algorithms (\Cref{sec:algorithms}) are
derived from a model of array access patterns, which is also used to
give a denotational model of our specification language
(\Cref{sec:semantics}).

Using the inference-mode of the tool, we applied our tool to our
original corpus (\Cref{sec:evaluation}). Our tool identifies and
infers specifications for roughly 60,000 array computations in the
corpus. Approximately 5,000 of the computations we found are non-trivial,
corresponding to code which is a possible source of errors. This
validates the design of our language in its capability to capture many
core patterns.

Our approach is not targeting a class of bugs that can be detected
automatically (\emph{push-button verification}). Instead, array
indexing bugs must be identified relative to a specification of the
intended access pattern.  Nevertheless, we report on instances of code
revisions from the commit histories of our corpus where a
correct specification would have either spotted an error (which was
latter correct) or would have assisted in an array refactoring task
(\Cref{sec:case-studies}).

\paragraph{Terminology and notation}

We fix some terminology for syntactic constructs in our target language.

\begin{defn}[Induction variables]
  An integer variable is a \emph{base induction variable} if it is the
  control variable of a ``\texttt{for}'' loop (\texttt{do} in
  Fortran), incremented by $1$ per iteration. A variable is
  interpreted as an induction variable only within the scope of the
  loop body. Throughout, $i, j, k$ range over induction variables.

  A \emph{derived induction variable} is an expression of the form
  $a \ast i + b$, where $a$ and $b$ are constant expressions, \ie{},
  an affine expression on an induction variable $i$.
\end{defn}

\begin{defn}[Array subscripts and indices]\label{def:array-subs}
  An \emph{array subscript}, denoted $a(\bar{e})$, is an expression
  which indicates the element of an $N$-dimensional array $a$ at the
  \emph{index} $\bar{e}$, specified by a comma-separated sequence of
  integer expressions, in expanded form as $(e_1, \ldots, e_N)$. An
  index $e_i$ is called \emph{relative} if the expression involves an
  induction variable. An \emph{absolute index} is a integer expression
  which is constant relative to the enclosing loop.
\end{defn}

\begin{defn}[Origin]\label{def:origin}
  An array subscript $a(\bar{e})$ has an \emph{origin index} if
  all $e \in \bar{e}$ are induction variables, \eg{}, \fortran{a(i,
    j)}.
\end{defn}

\begin{defn}[Neighbourhood index]\label{def:neighbour-ix}
  For an array subscript $a(\bar{e})$ an index $e \in \bar{e}$ is a
  \emph{neighbourhood index} if $e$ is of the form $e \equiv i$, $e \equiv i +
  c$, or $e \equiv i - c$, where $c$ is an integer constant. That is, a
  neighbourhood index is a constant translation of an induction variable. (The
  relation $\equiv$ identifies terms up-to commutativity of $+$ and the inverse
  relationship of $+$ and $-$ \eg{}, $(-b) + i \equiv i - b$).
\end{defn}


%%% Local Variables:
%%% mode: latex
%%% TeX-master: t
%%% End:
