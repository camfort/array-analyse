\documentclass[9pt]{sigplanconf}

\usepackage{amssymb}
\usepackage{amsmath}
\usepackage{amsthm}
\usepackage{stmaryrd}
\usepackage{color}
\usepackage{graphics}
\usepackage{fancyvrb}
\usepackage{subfigure}
\usepackage{amsthm}
\usepackage{tikz}
\usepackage{multirow}

\definecolor{darkblue}{rgb}{0.0,0.0,0.5}
\definecolor{darkgreen}{rgb}{0.0,0.4,0.0}
\definecolor{darkdarkgreen}{rgb}{0.0,0.35,0.0}

\usepackage[]{hyperref}
\hypersetup{
    unicode=false,          % non-Latin characters in Acrobat's bookmarks
    pdftoolbar=true,        % show Acrobat toolbar?
    pdfmenubar=true,        % show Acrobat menu?
    pdffitwindow=false,      % page fit to window when opened
    pdftitle={},    % title
    pdfauthor={}
    pdfsubject={},   % subject of the document
    pdfnewwindow=true,      % links in new window
    pdfkeywords={keywords}, % list of keywords
    colorlinks=true,       % false: boxed links; true: colored links
    linkcolor=darkblue,          % color of internal links
    citecolor=darkblue,        % color of links to bibliography
    filecolor=green,      % color of file links
    urlcolor=blue,          % color of external links
}


\CustomVerbatimEnvironment{SpecVerbatim}{Verbatim}{fontsize=\footnotesize,xleftmargin=0.5cm,
xrightmargin=0.2cm,commandchars=\\\{\},baselinestretch=0.98,numbersep=0.9em}
\CustomVerbatimEnvironment{ExmVerbatim}{Verbatim}{fontsize=\footnotesize,xleftmargin=0.5cm,
xrightmargin=0.2cm,baselinestretch=0.98,numbers=left,numbersep=0.9em,commandchars=\\\{\}}
\CustomVerbatimEnvironment{IVerbatim}{Verbatim}{fontsize=\relsize{-1},xleftmargin=0.5cm,
xrightmargin=0.2cm,commandchars=\\\{\},baselinestretch=0.98,numbersep=0.9em}


\definecolor{darkgreen}{rgb}{0.0,0.5,0.0}
\definecolor{darkpurple}{rgb}{0.6,0.0,0.6}
\definecolor{orange}{rgb}{0.8,0.4,0.0}
\definecolor{darkorange}{rgb}{0.5,0.2,0.0}
\definecolor{marco}{rgb}{0.0,0.3,0.5}
\definecolor{gray}{rgb}{0.2,0.2,0.2}

\newcommand{\bn}{\mathbb{N}}

\newcommand{\dnote}[1]{\textcolor{darkpurple}{Dom: #1}}
\newcommand{\mnote}[1]{\textcolor{darkgreen}{Mistral: #1}}
\newcommand{\anote}[1]{\textcolor{red}{Andy: #1}}

\newcounter{block}

\newtheorem{lemma}[block]{Lemma}
\newtheorem{proposition}[block]{Proposition}
%\newtheorem{definition}[block]{Definisssstion}

\theoremstyle{definition}

\newtheorem{theorem}[block]{Theorem}
\newtheorem{remark}[block]{Remark}
\newtheorem{example}[block]{Example}
\newtheorem{definition}[block]{Definition}

% Writing macros
\newcommand{\ie}{\emph{i.e.}}
\newcommand{\eg}{\emph{e.g.}}

\newcommand{\dimId}{\texttt{dim}}

% Semantics related
\newcommand{\interp}[1]{\llbracket{#1}\rrbracket}

% Syntax macros
\newcommand{\nonterm}[1]{\textit{#1}}
\newcommand{\term}[1]{\texttt{#1}}

\newcommand{\stenRefl}[1]{\term{reflexive} \, (\term{dim=}#1)}
\newcommand{\stenFwd}[3]{\term{forward} \, (\term{depth=}#1,
  \term{dim=}#2{#3})}
\newcommand{\stenBwd}[3]{\term{backward} \, (\term{depth=}#1,
  \term{dim=}#2{#3})}
\newcommand{\stenCen}[3]{\term{centered} \, (\term{depth=}#1,
  \term{dim=}#2{#3})}
\newcommand{\irrefl}{\texttt{irreflexive}}

\newcommand{\stenReflS}[1]{\term{refl} \, (\term{dim=}#1)}
\newcommand{\stenFwdS}[2]{\term{fwd} \, (\term{depth=}#1,
  \term{dim=}#2)}
\newcommand{\stenBwdS}[2]{\term{bwd} \, (\term{depth=}#1,
  \term{dim=}#2)}
\newcommand{\stenCenS}[2]{\term{cen} \, (\term{depth=}#1,
  \term{dim=}#2)}
\newcommand{\irreflS}{\texttt{irrefl}}

\newcommand{\stenFwdSR}[3]{\term{fwd} (\term{depth=}#1,
  \term{dim=}#2, #3)}
\newcommand{\stenBwdSR}[3]{\term{bwd} (\term{depth=}#1,
  \term{dim=}#2, #3)}
\newcommand{\stenCenSR}[3]{\term{cen} (\term{depth=}#1,
  \term{dim=}#2, #3)}
\newcommand{\stenReflSR}[1]{\term{refl} (\term{dim=}#1)}

% SYNTAX OPERATIONS AND PREDICATES
\newcommand{\neigh}{\textsf{neigh}}
\newcommand{\arrayTy}{\textsf{array}}
\newcommand{\rhsExp}{\textsf{rhsExp}}
\newcommand{\var}{\textsf{var}}

%% VECTOR NOTATIONS
\newcommand{\vect}[1]{\textbf{#1}}
\newcommand{\vtwoh}[2]{\setlength{\arraycolsep}{0em}
\left[\begin{array}{cc}#1 \; & \; #2\end{array}\right]}
\newcommand{\vthreeh}[3]{\setlength{\arraycolsep}{0em}
\left[\begin{array}{ccc}#1 \; & \; #2 \; & \; #3\end{array}\right]}
\newcommand{\vtwo}[2]{\setlength{\arraycolsep}{0em}
\left[\begin{array}{l}$#1$\\$#2$\end{array}\right]}
\newcommand{\vthree}[3]{\setlength{\arraycolsep}{0em}
\left[\begin{array}{l}$#1$\\$#2$\\$#3$\end{array}\right]}
\newcommand{\stwo}[4]
%{\vtwo{#1}{#2}\!\vtwo{#3}{#4}}
{\setlength{\arraycolsep}{0.1em}
\left[\begin{array}{rr}$#1$ & $#3$\\$#2$ & $#4$\end{array}\right]}

\newcommand{\singleEntry}[2]{\textbf{J}_{#2}^{#1}}
\newcommand{\zeroEntry}[2]{\textbf{K}_{#2}^{#1}}

%% OPERATIONS ON SPANS and VECTORS
\newcommand{\containedin}{\sqsubseteq}

%% MODEL
\newcommand{\effdims}[2]{\mathit{constr}(#1)_{#2}}

\newcommand{\consName}{\textbf{agree\textprime}}
\newcommand{\cons}[2]{#1 \,\, \consName{} \,\, #2}
\newcommand{\consAName}{\textbf{agree}}
\newcommand{\consA}[2]{#1 \,\, \consAName{} \,\, #2}
\newcommand{\consSub}[3]{#2 \,\, \consAName{}_{#1} \,\,#3}


\usepackage{xypic}

\begin{document}
\onecolumn
\begin{center}
{\LARGE{{Supplementary material for 
\textbf{ Verification of Stencil Computations through Spatial Specifications}}}}
\end{center}

\section{Correctness}


\begin{theorem}[Soundness]
\[
S \equiv T
\quad
\Rightarrow
\quad
\interp{S} \equiv \interp{T}
\]
\end{theorem}

\begin{proof}
By induction on the equational theory $\equiv$, with respect
to the model $\interp{-}_n$.

\begin{description}
  \item[\textsc{Case F \texttt{+} F}:]
    \begin{align*}
      & \; \interp{\stenFwdSR{a}{d}{r} \; \texttt{+} \; \stenFwdSR{b}{d}{s}}_{n} \\
      \equiv & \; \interp{\stenFwdSR{a}{d}{r}}_{n} \cup \interp{\stenFwdSR{b}{d}{s}}_{n} \\
      \equiv & \; \{i\singleEntry{d}{n} \mid i \in \{1, \ldots, a\} \} \cup
                  \{\zeroEntry{d}{n} \mid r \} \cup
                  \{i\singleEntry{d}{n} \mid i \in \{1, \ldots, b\} \} \cup
                  \{\zeroEntry{d}{n} \mid s \} \\
      \equiv & \; \{i\singleEntry{d}{n} \mid i \in \{1, \ldots, a\} \vee
                                             i \in \{1, \ldots, b\} \} \cup
                 \{\zeroEntry{d}{n} \mid r \} \cup \{\zeroEntry{d}{n} \mid s \} \\
      \equiv & \; \{i\singleEntry{d}{n} \mid i \in \{1, \ldots, a \max b\} \} \cup
                  \{\zeroEntry{d}{n} \mid r \vee s \}\\
      \equiv & \; \interp{\stenFwdSR{a \, \max \, b}{d}{r \vee s}}_n \qed
    \end{align*}
  \item[\textsc{Case C \texttt{+} C}:]
    \begin{align*}
      & \; \interp{\stenCenSR{a}{d}{r} \; \texttt{+} \; \stenCenSR{b}{d}{s}}_{n} \\
      \equiv & \; \interp{\stenCenSR{a}{d}{r}}_{n} \cup \interp{\stenCenSR{b}{d}{s}}_{n} \\
      \equiv & \; \{i\singleEntry{d}{n} \mid i \in \{-a, \ldots, a\} \setminus 0 \} \cup
                  \{\zeroEntry{d}{n} \mid r \} \cup
                  \{i\singleEntry{d}{n} \mid i \in \{-b, \ldots, b\} \setminus 0 \} \cup
                 \{\zeroEntry{d}{n} \mid s \} \\
      \equiv & \; \{i\singleEntry{d}{n} \mid i \in \{-a, \ldots, a\} \setminus 0 \vee
                                             i \in \{-b, \ldots, b\} \setminus 0 \} \cup
                 \{\zeroEntry{d}{n} \mid r \} \cup \{\zeroEntry{d}{n} \mid s \} \\
      \equiv & \; \{i\singleEntry{d}{n} \mid i \in \{-(a \max b), \ldots, a \max b\} \setminus 0 \} \cup
                  \{\zeroEntry{d}{n} \mid r \vee s \}\\
      \equiv & \; \interp{\stenCenSR{a \, \max \, b}{d}{r \vee s}}_n \qed
    \end{align*}
  \item[\textsc{Case B \texttt{+} B}:]
    \begin{align*}
      & \; \interp{\stenBwdSR{a}{d}{r} \; \texttt{+} \; \stenBwdSR{b}{d}{s}}_{n} \\
      \equiv & \; \interp{\stenBwdSR{a}{d}{r}}_{n} \cup \interp{\stenBwdSR{b}{d}{s}}_{n} \\
      \equiv & \; \{i\singleEntry{d}{n} \mid i \in \{-a, \ldots, -1\} \} \cup
                  \{\zeroEntry{d}{n} \mid r \} \cup
                  \{i\singleEntry{d}{n} \mid i \in \{-b, \ldots, -1\} \} \cup
                  \{\zeroEntry{d}{n} \mid s \} \\
      \equiv & \; \{i\singleEntry{d}{n} \mid i \in \{-a, \ldots, -1\} \vee
                                             i \in \{-b, \ldots, -1\} \} \cup
                  \{\zeroEntry{d}{n} \mid r \} \cup \{\zeroEntry{d}{n} \mid s \} \\
      \equiv & \; \{i\singleEntry{d}{n} \mid i \in \{-(a \max b), \ldots, -1 \} \} \cup
                  \{\zeroEntry{d}{n} \mid r \vee s \}\\
      \equiv & \; \interp{\stenBwdSR{a \, \max \, b}{d}{r \vee s}}_n \qed
    \end{align*}
  \item[\textsc{Case C \texttt{+} F}: with side condition  $\;\;(a \geq b)$]
    \begin{align*}
      & \; \interp{\stenCenSR{a}{d}{r} \; \texttt{+} \; \stenFwdSR{b}{d}{s}}_{n} \\
      \equiv & \; \interp{\stenCenSR{a}{d}{r}}_{n} \cup \interp{\stenFwdSR{b}{d}{s}}_{n} \\
      \equiv & \; \{i\singleEntry{d}{n} \mid i \in \{-a, \ldots, a\} \setminus 0 \} \cup
                  \{\zeroEntry{d}{n} \mid r \} \cup
                  \{i\singleEntry{d}{n} \mid i \in \{1, \ldots, b\} \} \cup
                  \{\zeroEntry{d}{n} \mid s \} \\
      \equiv & \; \{i\singleEntry{d}{n} \mid i \in \{-a, \ldots, a\} \setminus 0
                    \vee i \in \{1, \ldots, b\} \} \cup
                 \{\zeroEntry{d}{n} \mid r \} \cup \{\zeroEntry{d}{n} \mid s \} \\
      \equiv & \; \{i\singleEntry{d}{n} \mid i \in \{-a, \ldots, a\}
               \setminus 0 \} \cup \{\zeroEntry{d}{n} \mid r \vee s \} \quad \text{, using }a \geq b \\
      \equiv & \; \interp{\stenCenSR{a}{d}{r \vee s}}_n \qed
    \end{align*}
  \item[\textsc{Case C \texttt{+} B}: with side condition  $\;\;(a \geq b)$]
    \begin{align*}
      & \; \interp{\stenCenSR{a}{d}{r} \; \texttt{+} \; \stenBwdSR{b}{d}{s}}_{n} \\
      \equiv & \; \interp{\stenCenSR{a}{d}{r}}_{n} \cup \interp{\stenBwdSR{b}{d}{s}}_{n} \\
      \equiv & \; \{i\singleEntry{d}{n} \mid i \in \{-a, \ldots, a\} \setminus 0 \} \cup
                  \{\zeroEntry{d}{n} \mid r \} \cup
                  \{i\singleEntry{d}{n} \mid i \in \{-b, \ldots, -1\} \} \cup
                  \{\zeroEntry{d}{n} \mid s \} \\
      \equiv & \; \{i\singleEntry{d}{n} \mid i \in \{-a, \ldots, a\} \setminus 0
                    \vee i \in \{-b, \ldots, -1\} \} \cup
                 \{\zeroEntry{d}{n} \mid r \} \cup \{\zeroEntry{d}{n} \mid s \} \\
      \equiv & \; \{i\singleEntry{d}{n} \mid i \in \{-a, \ldots, a\}
               \setminus 0 \} \cup \{ \zeroEntry{d}{n} \mid r \vee s \} \quad \text{, using }a \geq b \\
      \equiv & \; \interp{\stenCenSR{a}{d}{r \vee s}}_n \qed
    \end{align*}
  \item[\textsc{Case B \texttt{+} F}:]
    \begin{align*}
      & \; \interp{\stenBwdSR{a}{d}{r} \; \texttt{+} \; \stenFwdSR{a}{d}{r}}_{n} \\
      \equiv & \; \interp{\stenBwdSR{a}{d}{r}}_{n} \cup \interp{\stenFwdSR{a}{d}{s}}_{n} \\
      \equiv & \; \{i\singleEntry{d}{n} \mid i \in \{-a, \ldots, -1\} \} \cup
                  \{\zeroEntry{d}{n} \mid r \} \cup
                  \{i\singleEntry{d}{n} \mid i \in \{1, \ldots, a\} \} \cup
                  \{\zeroEntry{d}{n} \mid s \} \\
      \equiv & \; \{i\singleEntry{d}{n} \mid i \in \{-a, \ldots, -1\}
                    \vee i \in \{1, \ldots, a\} \} \cup
                 \{\zeroEntry{d}{n} \mid r \} \cup \{\zeroEntry{d}{n} \mid s \} \\
      \equiv & \; \{i\singleEntry{d}{n} \mid i \in \{-a, \ldots, a\} \setminus 0 \} \cup
                              \{\zeroEntry{d}{n} \mid r \vee s \} \\
      \equiv & \; \interp{\stenCenSR{a}{d}{r \vee s}}_n \qed
    \end{align*}
%
%
  \item[\textsc{Case P \texttt{+} F}:]
    \begin{align*}
      & \; \interp{\stenReflS{d} \; \texttt{+} \; \stenFwdSR{a}{d}{r}}_{n} \\
      \equiv & \; \interp{\stenReflS{d}}_{n} \cup \interp{\stenFwdSR{a}{d}{r}}_{n} \\
      \equiv & \; \{\zeroEntry{d}{n}\} \cup
                  \{i\singleEntry{d}{n} \mid i \in \{1, \ldots, a\} \} \cup
                  \{\zeroEntry{d}{n} \mid r \} \\
      \equiv & \; \{\zeroEntry{d}{n}\} \cup
                  \{i\singleEntry{d}{n} \mid i \in \{1, \ldots, a\} \} \\
      \equiv & \; \interp{\stenFwdS{a}{d}}_n \qed
    \end{align*}
%
%
  \item[\textsc{Case P \texttt{+} B}:]
    \begin{align*}
      & \; \interp{\stenReflS{d} \; \texttt{+} \; \stenBwdSR{a}{d}{r}}_{n} \\
      \equiv & \; \interp{\stenReflS{d}}_{n} \cup \interp{\stenBwdSR{a}{d}{r}}_{n} \\
      \equiv & \; \{\zeroEntry{d}{n}\} \cup
                  \{i\singleEntry{d}{n} \mid i \in \{-a, \ldots, -1\} \} \cup
                  \{\zeroEntry{d}{n} \mid r \} \\
      \equiv & \; \{\zeroEntry{d}{n}\} \cup
                  \{i\singleEntry{d}{n} \mid i \in \{-a, \ldots, -1\} \} \\
      \equiv & \; \interp{\stenBwdS{a}{d}}_n \qed
    \end{align*}

%
  \item[\textsc{Case P \texttt{+} C}:]
    \begin{align*}
      & \; \interp{\stenReflS{d} \; \texttt{+} \; \stenCenSR{a}{d}{r}}_{n} \\
      \equiv & \; \interp{\stenReflS{d}}_{n} \cup \interp{\stenCenSR{a}{d}{r}}_{n} \\
      \equiv & \; \{\zeroEntry{d}{n}\} \cup
                  \{i\singleEntry{d}{n} \mid i \in \{-a, \ldots, a\}
               \setminus 0 \} \cup
                  \{\zeroEntry{d}{n} \mid r \} \\
      \equiv & \; \{\zeroEntry{d}{n}\} \cup
                  \{i\singleEntry{d}{n} \mid i \in \{-a, \ldots, a\}
               \setminus 0 \} \\
      \equiv & \; \interp{\stenCenS{a}{d}}_n \qed
    \end{align*}
  \item[\textsc{Case POINT}]
   let $R$ be a region constant of dimension $d$ and $S$ be a region
   constant of dimension $d'$. \\

   By the definition of $\interp{-}$ and the syntactic sugar,
    then (*) $\interp{R}^{\irreflS} = \interp{R} \setminus
   \{\zeroEntry{d}{n}\}$ and similarly $\interp{S}^{\irreflS} =
   \interp{S} \setminus \{\zeroEntry{d'}{n}\}$
   \begin{align*}
    & \; \interp{(R \term{*} S^{\irreflS}) \term{+} (R^{\irreflS}
      \term{*} S)}_n \\
 \equiv & \; \interp{(R \term{*} S^{\irreflS})}_n \cup
          \interp{(R^{\irreflS} \term{*} S)}_n \\
 \equiv & \; (\interp{R}_n \otimes (\interp{S}_n \setminus
         \{\zeroEntry{d'}{n}\}))
   \cup ((\interp{R}_n \setminus \{\zeroEntry{d}{n}\}) \otimes
          \interp{S}_n) \\
(**) \equiv & \; (\{x \mid i \in \{1, \ldots, 2^n\},
                   (u, v) \in \interp{R}_n \times (\interp{S}_n \setminus
         \{\zeroEntry{d'}{n}\}),
                   x = (u \bowtie v)_i,
                   j \in (\effdims{\interp{R}_n}{n} \cup \effdims{\interp{S}_n \setminus
         \{\zeroEntry{d'}{n}\}}{n}),
                   x_j \neq \infty
                  \})
           \\
 \cup \; & (\{x \mid i \in \{1, \ldots, 2^n\},
                   (u, v) \in (\interp{R}_n \setminus
         \{\zeroEntry{d}{n}\}) \times \interp{S}_n),
                   x = (u \bowtie v)_i,
                   j \in (\effdims{\interp{R}_n \setminus
         \{\zeroEntry{d'}{n}\}}{n} \cup \effdims{\interp{S}_n}{n}),
                   x_j \neq \infty
                  \})
   \end{align*}
   By the meaning of $^{\irreflS}$ and (*) we then have that:
   \begin{align*}
    \effdims{\interp{R^\irreflS}}{n} = & \bigcup_{\interp{R} \setminus
   \{\zeroEntry{d}{n}\}}
    \{x \mid x \in \{1, \ldots, n\}, u_x \neq \infty\} \\
     = & \bigcup_{\interp{R}}  \{x \mid x \in \{1, \ldots, n\}, u_x \neq
       \infty\}  \setminus \{x \mid x \in \{1, \ldots, n\},
       (\zeroEntry{d}{n})_x \neq \infty\} \\
     = & \bigcup_{\interp{R}}  \{x \mid x \in \{1, \ldots, n\}, u_x \neq
       \infty\}  \setminus \{d\} \\
     = & \effdims{\interp{R}}{n} \setminus \{d\}
   \end{align*}
   similarly $\effdims{\interp{S^\irreflS}}{n} = \effdims{\interp{S}}{n}
   \setminus \{d'\}$.
   Continuing from (**), fold the two set-builders together:
  %
\begin{align*}
\equiv & \; (\{x \mid i \in \{1, \ldots, 2^n\},
                   (u, v) \in ((\interp{R}_n \times (\interp{S}_n \setminus
         \{\zeroEntry{d'}{n}\})) \cup (\interp{R}_n \setminus
         \{\zeroEntry{d}{n}\}) \times \interp{S}_n),
                   x = (u \bowtie v)_i, \\
 & \qquad         j \in (\effdims{\interp{R}}{n} \cup
         (\effdims{\interp{S}}{n} \setminus \{d'\}) \cup (\effdims{\interp{R}}{n} \setminus
         \{d'\}) \cup \effdims{\interp{S}}{n}),
                   x_j \neq \infty
                  \}
\end{align*}
%
Since $X \cup (X \setminus Y) = X$ for all sets, which also lifts to
Cartesian product, \ie{}, $(A \times (B \setminus C)) \cup ((A \setminus
D) \times B) = A \times B$ then:
%
\begin{align*}
\equiv \; & \; (\{x \mid i \in \{1, \ldots, 2^n\},
                   (u, v) \in ((\interp{R}_n \times (\interp{S}_n \setminus
         \{\zeroEntry{d'}{n}\})) \cup (\interp{R}_n \setminus
         \{\zeroEntry{d}{n}\}) \times \interp{S}_n),
                   x = (u \bowtie v)_i, \\
 & \qquad         j \in (\effdims{\interp{R}}{n} \cup \effdims{\interp{S}}{n}),
                   x_j \neq \infty
                  \} \\
\equiv \; & \; (\{x \mid i \in \{1, \ldots, 2^n\},
                   (u, v) \in (\interp{R}_n \times \interp{S}_n),
                   x = (u \bowtie v)_i, j \in (\effdims{\interp{R}}{n} \cup \effdims{\interp{S}}{n}),
                   x_j \neq \infty
                  \} \\
\equiv \; & \interp{R}_n \otimes \interp{S}_n \\
\equiv \; & \interp{R \term{*} S} \qed
\end{align*}

\item[\textsc{Case $*$dim}]

Let $R$ and $S$ be region constants (i.e., dimensionality
of 1) of the same dimension $d$, denoted in this equation $R^d$ and $S^d$.

\textbf{Lemma}: If $n=1$, $[a] \bowtie_1 [b] = [a b]$

\begin{align*}
 & \interp{R^d \ast S^d} \\
\equiv \; & \interp{R} \otimes_1 \interp{S} \\
\equiv \; & \interp{R^d} \cup \interp{S^d} \;\; \textit{by the above lemma} \\
\equiv \; & \interp{R^d \term{+} S^d}
\end{align*}

  \item[\textsc{Case +IDEM, +COMM, +ASSOC}:]
    Follows from respective properties of union of sets.
  \item[\textsc{Case *COMM}:] $\interp{r_1 \; \texttt{$\ast$} \; r_2}_n =
    \interp{r_1}_n \otimes \interp{r_2}_n$. Hence, $\texttt{$\ast$}$ is
    commutative iff $\otimes$ is commutative. Now let $\interp{r_1}_n$ and
    $\interp{r_2}_n$ be $S$ and $T$ respectively. Then from the definition of
    $\otimes$, we have $S \otimes T = \{ x \;|\; (s,t) \in S \times T,
    h \in \effdims{S}{n} \cup \effdims{T}{n}, x = (s \bowtie t)_k,
    k \in \{ 1, \ldots, 2^n \}, x_h \neq \infty \}$. Then to show commutativity
    of $\otimes$, it is enough to show for any vectors $s$ and $t$, any row
    vector of $s \bowtie t$ is also a row vector in the matrix $t \bowtie s$
    possibly with a different row index, since $\otimes$ collects row vectors of
    pairwise permutations. This in effect means $s \bowtie t$ is a permutation
    of $t \bowtie s$'s rows.

    Now let $\neg$ be a unary operator over logical matricies that revert $1$s
    to $0$s and vice versa. Then we can define pairwise permutation as
    $(s \bowtie t)_k = s \odot b_k + t \odot \neg b_k$. Now since $k$ ranges
    over all bit strings of length $n$, for all $k$ there is a $l$ such that
    $b_l = \neg b_k$. Hence for all $k$,

    \begin{align*}
      (s \bowtie t)_k &= s \odot b_k + t \odot \neg b_k \\
                      &= s \odot \neg b_l + t \odot \neg (\neg b_l) \\
                      &= t \odot \neg b_l + s \odot b_l \\
                      &= (t \bowtie s)_l
    \end{align*}

  \item[\textsc{Case *ASSOC}:]
    \begin{lemma}\label{lem:bowinf}
      Pairwise mutation, $u \bowtie v$, would produce $\infty$ column $j$ vector
      if both $u$ and $v$ are $\infty$ in vector.
    \end{lemma}

    \begin{proof}
      Let $b$ be the logical matrix where each column represents the binary
      string from $1$ to $2^n$. Now if the entire column is $\infty$, then
      independent of the row number $i$, column $j$ must evaluate to $\infty$.
      \begin{align*}
        (u \bowtie v)_{i,j} =& u_{j} \times b_{i,j} + v_{j} \times \neg b_{i,j}
                             & \mbox{by definition of $\bowtie$} \\
                            =& \infty \times b_{i,j} + \infty \times \neg b_{i,j}
                             & \mbox{Using the assumption in the lemma} \\
                            =& \infty
                             & \mbox{The value of $b_{i,j}$ no longer matters}
      \end{align*}
    \end{proof}

    \begin{lemma}\label{lem:tensordims}
      Constrained dimensions of tensor product of two models is the union of the
      constrained dimensions of models in question separately.
      \begin{align*}
        \effdims{U \otimes V}{n} = \effdims{U}{n} \cup \effdims{V}{n} \\
      \end{align*}
    \end{lemma}

    \begin{proof}
      \begin{align}
        \effdims{U \otimes V}{n} = &
          \bigcup_{x \in U \otimes V} \{ i \;|\; x_i \neq \infty \} \\
        = & \effdims{U}{n} \cup \effdims{V}{n} \\
          & \parbox{7cm}{By definition of $\otimes$, constrained dimensions of
                         $U$ and $V$ can be non-$\infty$ and by
                         Lemma~\ref{lem:bowinf} all other dimensions are
                         $\infty$.} \\
      \end{align}
    \end{proof}
%
    We have $\interp{r_1 \; \texttt{$\ast$} \;
    (r_2 \; \texttt{$\ast$} \; r_3)}_n = \interp{r_1}_n \otimes (\interp{r_2}_n
    \otimes \interp{r_3}_n)$ by definition. So we need to show $\otimes$ is
    associative. Now let $\interp{r_1}_n$, $\interp{r_2}_n$, and
    $\interp{r_3}_n$ be $R$, $S$, and $T$ respectively.

    Expanding the definition of $\otimes$,
%
    \begin{align*}
        & \interp{r_1}_n \otimes (\interp{r_2}_n \otimes \interp{r_3}_n) \\
      = & \bigg\{ x \; \bigg| \;
            \parbox{10cm}{
              $(r,st) \in R \times (S \otimes T),
               h \in \effdims{R}{n} \cup \effdims{S \otimes T}{n},$ \\
              $j \in \{ 1, \ldots, 2^n \},
                 x = (r \bowtie st)_j, x_h \neq \infty$
            } \bigg\} \\
      = & \bigg\{ x \; \bigg| \;
            \parbox{10cm}{
              $(r,st) \in R \times (S \otimes T),
               h \in \effdims{R}{n} \cup \effdims{S}{n} \cup \effdims{T}{n},$ \\
              $j \in \{ 1, \ldots, 2^n \},
                 x = (r \bowtie st)_j, x_h \neq \infty$
            } \bigg\} \\
        & \mbox{using Lemma~\ref{lem:tensordims} and expanding} \\
      = & \bigg\{ x \; \bigg| \;
            \parbox{10cm}{
              $(r,s,t) \in R \times (S \times T),
               h \in \effdims{R}{n} \cup \effdims{S}{n} \cup \effdims{T}{n},$ \\
              $i,j \in \{ 1, \ldots, 2^n \},
                y = (s \bowtie t)_i, x = (r \bowtie y)_j,
                x_h \neq \infty, y_h \neq \infty$
            } \bigg\} \\
        & \mbox{expanding definition of $\otimes$} \\
    \end{align*}
%
    If $\ast$ is associative, then an equivalent definition is:
    \begin{equation*}
          \bigg\{ x \; \bigg| \;
            \parbox{10cm}{
              $(r,s,t) \in R \times (S \times T),
               h \in \effdims{R}{n} \cup \effdims{S}{n} \cup \effdims{T}{n},$ \\
              $k,l \in \{ 1, \ldots, 2^n \},
                y = (r \bowtie s)_k, x = (y \bowtie t)_l,
                x_h \neq \infty, y_h \neq \infty$
            } \bigg\}
    \end{equation*}d
%
    For these two sets to be equivalent, it is enough to show that for any $r$,
    $s$, and $t$, for each $(i,j)$ pair there is a $(k,l)$ pair. When definition
    of $\bowtie$ is expanded, we then get the following equality:

    $$r \odot b_j + s \odot b_i \odot \neg b_j + t \odot \neg b_i \odot \neg b_j
    = r \odot b_k \odot b_l + s \odot \neg b_k \odot b_l + t \odot \neg b_l$$

    Hence, we get the following equations for logical row vectors:

    \begin{align}
      b_j & = b_k \odot b_l \\
      b_i \odot \neg b_j & = \neg b_k \odot b_l \\
      \neg b_i \odot \neg b_j & = \neg b_l
    \end{align}

    Observe that pairwise product on logical vectors corresponds to logical
    conjunction operation. In the following derivation $\wedge$ is used in place
    of $\odot$ for consistency. $\neg$ denotes logical negation of each element
    of the vector. Vector disjunction can be defined as $a \vee b = a + b - a
    \odot b$. Standard boolean algebra generalises over vectors.

    To see there is always a solution it is sufficient to show that the system
    of equations is consistent.

    \begin{align}
      b_l &= b_i \vee b_j & \mbox{negating $(3)$ and applying De Morgan's law} \\
      b_j &= b_k \wedge (b_i \vee b_j) & \mbox{substituting $(4)$ into $(1)$} \\
      b_i \wedge \neg b_j &= \neg b_k \wedge (b_i \vee b_j) & \mbox{substituting $(4)$
        into $(2)$} \\
      b_j \vee b_i \wedge \neg b_j &= b_k \wedge (b_i \vee b_j) \vee \neg b_k \wedge (b_i \vee b_j) & \mbox{disjunction of $(5)$ and $(6)$} \\
      b_i \vee b_j &= b_i \vee b_j & \mbox{algebraically simplifying (7)} \nonumber
    \end{align}



  \item[\textsc{Case DIST}:]
    \begin{lemma}\label{lem:effdims}
      Constrained dimensions distribute over set union.
      $$\effdims{U \cup V}{n} = \effdims{U}{n} \cup \effdims{V}{n}$$
    \end{lemma}

    \begin{proof}
      \begin{align*}
        \effdims{U \cup V}{n} &= \bigcup_{x \in U \cup V} \{ i \; | \; i \in \{1, \ldots, n\} \wedge x_i \neq \infty \} \\
        &= \bigcup_{x \in U} \{ i \; | \; i \in \{1, \ldots, n\}, x_i \neq \infty \} \cup
           \bigcup_{x \in V} \{ i \; | \; i \in \{1, \ldots, n\}m x_i \neq \infty \} \\
        &= \effdims{U}{n} \cup \effdims{V}{n} \\
      \end{align*}
    \end{proof}

    \begin{align*}
             & \interp{r_1 \; \ast \; (r_2 \; \texttt{+} \; r_3)}_n \\
      \equiv & \interp{r_1}_n \otimes \interp{r_2 \; \texttt{+} \; r_3}_n \\
      \equiv & \interp{r_1}_n \otimes (\interp{r_2}_n \cup \interp{r_3}_n) \\
      \equiv &
  \bigg\{x \; \bigg| \;
    \parbox{11cm}{$(u, v) \in \interp{r_1}_n \times (\interp{r_2}_n \cup \interp{r_3}_n),
                   j \in \effdims{\interp{r_1}_n}{n} \cup
                         \effdims{\interp{r_2}_n \cup
                         \interp{r_3}_n}{n},$ \\[0.1em]
                  $i \in \{1, \ldots, 2^n\},
                   x = (u \bowtie v)_i, x_j \neq \infty$
                  } \bigg\} \\
      \equiv &
  \Bigg\{x \; \Bigg| \;
    \parbox{11cm}{$(u, v) \in \interp{r_1}_n \times \interp{r_2}_n \vee (u, v) \in \interp{r_1}_n \times \interp{r_3}_n), \\
                   j \in \effdims{\interp{r_1}_n}{n} \cup
                          \effdims{\interp{r_2}_n}{n} \vee
                    j \in \effdims{\interp{r_1}_n}{n} \cup
                          \effdims{\interp{r_3}_n}{n},$ \\ [0.1em]
                  $i \in \{1, \ldots, 2^n\},
                   x = (u \bowtie v)_i, x_j \neq \infty$
                  } \Bigg\} \\
             & \mbox{using Lemma~\ref{lem:effdims} and distributivity of cartesian product over union.} \\
      \equiv &
  \bigg\{x \; \bigg| \;
    \parbox{9cm}{$(u, v) \in \interp{r_1}_n \times \interp{r_2}_n,
                  j \in \effdims{\interp{r_1}_n}{n} \cup
                        \effdims{\interp{r_2}_n}{n},$ \\
                 $i \in \{1, \ldots, 2^n\},
                  x = (u \bowtie v)_i, x_j \neq \infty$
                  } \bigg\} \cup \\
  & \bigg\{x \; \bigg| \;
    \parbox{9cm}{$(u, v) \in \interp{r_1}_n \times \interp{r_3}_n,
                  j \in \effdims{\interp{r_1}_n}{n} \cup
                        \effdims{\interp{r_3}_n}{n},$ \\
                 $i \in \{1, \ldots, 2^n\},
                  x = (u \bowtie v)_i, x_j \neq \infty$
                  } \bigg\} \\
      \equiv & (\interp{r_1}_n \otimes \interp{r_2}_n)
        \cup (\interp{r_1}_n \otimes \interp{r_3}_n) \\
      \equiv & \interp{r_1 \ast r_2}_n \cup \interp{r_1 \ast r_3}_n \\
      \equiv & \interp{r_1 \ast r_2 + r_1 \ast r_3}_n \\
    \end{align*}
\end{description}
\end{proof}


\begin{theorem}[Soundness of approximation]
\begin{align*}
\forall S, T, M . \quad S <: T \; \wedge \; \cons{M}{\interp{S}_n}  \;
\Rightarrow \cons{M}{\interp{T}_n}
\end{align*}
\end{theorem}

\begin{proof}
By induction on the definition of $<:$
\begin{itemize}
\item 
\end{itemize}
\item \trule{shrink} $R <:^r S \; \Rightarrow \; R <: \term{atMost} \,
  S$



\item \trule{Eq} trivial by soundness of $\equiv$.
\item \trule{CongR} follows straightforwardly by the $(1 \leq *)$ 
cause for consistency
\itme \trule{rep} follows by $(*)$.
\end{proof}

\begin{lemma}
For a model, its indices never have an $\infty$ value in 
the constrained dimensions.
\begin{align*}
\forall M, n . \;\; j \in \textit{constr}(M)_n \Rightarrow
x \in M \, \wedge \, x_j \neq \infty
\end{align*}
\end{lemma}
\begin{proof}
By the definition of $\textit{constr}$.
\end{proof}

\begin{lemma}
For all pairs of $n$-dimensional specifications $R,S$, the model or
$R$ is a subset of the model of $R \term{*} S$:
\begin{align*}
\forall R, S, n . \;\; \interp{R} \subseteq (\interp{R} \otimes_n \interp{S})
\end{align*}
\end{lemma}

\begin{proof}
Given two $n$-vectors $u, v$ then 
$u = (u \bowtie v)_0$ and $v \in (u \bowtie v)_0$.

Forall $u_r \in \interp{R}$, then 

\begin{align*}
      & \interp{R}_n \\
\equiv \;\; & \{ x \mid (u, v) \in \interp{R}_n \times \interp{S}_n \;
              \wedge \; x = (u \bowtie v)_0 \}  \\
%
\subseteq  \;\; &
  \Bigg\{x \; \Bigg| \;
    \parbox{5.2cm}{$(u, v) \in \interp{R}_n \times \interp{S}_n $ \\[0.1em]
                 $\;\; \wedge \;\, j \in (\effdims{\interp{R}}{n} \cup
                   \effdims{\interp{S}}{n})$\\[0.1em]
                  $\;\; \wedge \;\, i \in \{1, \ldots, 2^n\} \, \wedge
                  \, x = (u \bowtie v)_i$ \\[0.1em]
                   $\;\; \wedge \;\, x_j \neq \infty$
                  } \Bigg\} \\
\equiv \;\; & \interp{R} \otimes_n \interp{S} 
\end{align*}

\end{proof}




\end{document}

