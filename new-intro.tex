\documentclass{article}

\usepackage{enumitem}
\usepackage{minted}
\newcommand{\ie}{\emph{i.e.}}
\newcommand{\eg}{\emph{e.g.}}
\usemintedstyle{vs}
\fvset{
  linenos=true,
  fontsize=\small,
  breaklines=true,
  breakafter=),
  xleftmargin=4em
}

\begin{document}

\emph{Stencils} are a ubiquitous programming pattern, common in
scientific and numerical computing applications. Informally, a stencil
computation computes an array whose value at each index $i$ is
calculated from a \emph{neighbourhood} of values around $i$ in some
input array(s), \eg{}, the Game of Life, convolutions in image
processing, approximations to differential equations. For example, the
following computes the one-dimensional discrete Laplace transform (an
approximation to a derivative) in Fortran:
%
\begin{minted}{fortran}
do i = 1, (n-1)
   b(i) = a(i-1) - 2*a(i) + a(i+1)
end do
\end{minted}
%
Values \texttt{b(i)} are calculated from a neighbourhood of elements
around \texttt{i} in the input array \texttt{a}.  In this example, the
access pattern is simple and easily understood. More complex
stencil computations are more prone to errors from simple lexical
mistakes. For example, the following is a three-line snippet from a
Navier-Stokes fluid simulator in which two arrays are read with
different data access patterns, across two dimensions.
%%
\begin{minted}[firstnumber=20,xleftmargin=4em]{fortran}
du2dx = ((u(i,j)+u(i+1,j))*(u(i,j)+u(i+1,j))+   &
  gamma*abs(u(i,j)+u(i+1,j))*(u(i,j)-u(i+1,j))- &
  (u(i-1,j)+u(i,j))*(u(i-1,j)+u(i,j))-          &
  gamma*abs(u(i-1,j)+u(i,j))*(u(i-1,j)-u(i,j))) &
  /(4.0*delx)

duvdy = ((v(i,j)+v(i+1,j))*(u(i,j)+u(i,j+1))+   &
  gamma*abs(v(i,j)+v(i+1,j))*(u(i,j)-u(i,j+1))- &
  (v(i,j-1)+v(i+1,j-1))*(u(i,j-1)+u(i,j))-      &
  gamma*abs(v(i,j-1)+v(i+1,j-1))*(u(i,j-1)-     &
  u(i,j))) / (4.0*dely)

laplu = (u(i+1,j)-2.0*u(i,j)+u(i-1,j))/delx/delx+ &
  (u(i,j+1)-2.0*u(i,j)+u(i,j-1))/dely/dely

f(i,j) = u(i,j)+del_t*(laplu/Re-du2dx-duvdy)
\end{minted}
%%
The access pattern is much harder to understand than the
one-dimensional Laplace.  The miasma of indexing expressions of the
form \texttt{var(i$\pm$a, j$\pm$b)} is not only hard to read, but is
prone to simple textual input mistakes, \eg{}, swapping \texttt{-} and
\texttt{+}, missing an indexing term, or transforming the wrong
variable \eg{} \texttt{(i+1,j)} instead of \texttt{(i,j+1)}.

In practice, we find that the usual development procedure for complex
stencil computations involves some ad hoc testing to ensure that no
transcription mistakes have been made. This may involve visual
inspection of the results, based on a rendered image. Typically such
testing is then discarded once the code is seen to be correct.

This is not the only information that is typically discarded. The
shape of the indexing pattern is often a result of choices made in the
numerical-analysis discretisation procedure used to convert a set of
continuous equations into a discrete approximation. Rarely are these
decisions captured in the source code, yet the shape of access is
usually uniform and is given a perspicuous and concise description in
numerical analysis literature \eg{}, ``\emph{centered in space, to a
  depth of 1}'' referring to indexing terms \texttt{a(i)},
\texttt{a(i-1)} and \texttt{a(i+1)}.

To support the development of correct stencil computations, we propose
a simple, abstract specification language for the data access pattern
of stencils. The shape of the stencil computation in the
Laplace example is specifed in our language as:
%
\begin{minted}[linenos=false]{fortran}
!=  stencil centered(depth=1, dim=1) :: a
\end{minted}
%
This describes that \texttt{a} is accessed with a symmetrical pattern
(``centered'') to a depth of one in each direction in its first
dimension.  The stencil computation shape in the Navier-Stokes example
is specified as:
%
\begin{minted}[linenos=false,xleftmargin=4em,breakindent=3em]{fortran}
!= stencil centered(depth=1,dim=1)*reflexive(dim=2) + centered(depth=1,dim=2)*reflexive(dim=1) :: u

!= stencil forward(depth=1,dim=1) * backward(depth=1,dim=2) :: v
\end{minted}
%
The specification requires that, over the whole fragment, \texttt{u}
is accessed with a centered pattern to depth of 1 in both dimensions
(this is known as the \emph{five-point stencil}) and \texttt{v} is
accessed in a neighbourhood bounded forwards to depth of $1$ in the
first dimension and backward to a depth of $1$ in the second
dimension.

We provide a verification tool for such specification which can check
the correctness of code against the specification, and can also infer
specifications, synthesising these as comments in the
source code. The main features and benefits of our approach are as follows:
%
\begin{itemize}
\item The main use case is that a programmers writes a specification
  first (perhaps informed by their numerical analysis), then writes
  the corresponding source code, and lastly has the tool check its
  conformance to the specifications. This specify-and-check approach
  removes the usually ad hoc testing effort, and provides the
  opportunity for easy re-verification in the event of any refactoring
  of software maintenance activities.
%
\item The specification provides a concise description which can be
  preserved in the source code to provide documentation to future
  developers, contrasting with ad hoc tests which are usually
  discarded or specialised to the development environment. Separate from the
  verification procedure, the infer-and-synthesise mode of the tool
  provides a way to automatically generate additional useful
  documentation for future users.
%
\item Specifications are integrated into the code as comments for easy
integration with existing tools, as well as
  documentation systems (\eg{}, we support Doxygen syntax).
%
\item Our specification format is deliberately abstract, with a small
  number of combinators that \emph{do not involve any indexing
    expressions}, \eg{} does not involve expressions like
  \texttt{a(i+1,j-1)}. This contrasts with other stencil specification
  approaches. For example, the work of Kamil et al.~\cite{kamil2016verified}
  provides full-functionality specifications of stencil computations from which
  equivalent DSL code can be generated.  If a user wishes to specify, say, the
  Naiver-Stokes stencil above, they must write a specification which is very
  similar, involving textually dense array-indexing expressions. Therefore any
  low-level mistakes that could be made programming complex indexing code could
  also be made when writing a specification. Our specifications are much more
  abstract, and also more lightweight with the aim of aiding adoption by
  scientists.
\end{itemize}

It should be noted that our verification tool does not deal in the
class of bugs that can be detected automatically (\emph{push-button
  verification}). Instead, stencil computation bugs must largely be
identified relative to a specification of the intended behaviour.

In this paper, we make the following contributions:
%
\begin{itemize}
\item we introduce our specification language for stencils
  that captures many common forms of data access pattern
  (Section~\ref{sec:lang});

\item we detail checking and inference algorithms for our
  specifications (Section~\ref{sec:analysis}), derived from a
  denotational model for the specificaiton language
  (Section~\ref{sec:semantics});

\item we provide an implementation of our approach as an extension to
  CamFort, an open-source program analysis tool for Fortran;

\item we report on a quantitative study of stencil computations on a
  corpus of numerical Fortran programs (ranging from small to large),
  totalling one million lines of code. Our tool identifies and infers
  specifications for 35,000 stencil computations in the
  corpus. Approximately 10\% of the stencils we found are non-trivial,
  corresponding to code which is a possible source of errors. This
  validates our hypothesis that the majority of stencil computations have a
  regular shape, and validates the design of our language in its
  capability to capture many core patterns.

\item We give a detailed verification case study of two
  particular stencil computation, simulating programming errors.
  Our approach is able to detect the all possible
  indexing errors.
\end{itemize}
%



\end{document}
