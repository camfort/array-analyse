\documentclass[acmlarge,review]{acmart}

% Essential packages
\usepackage{amsmath}
\usepackage{amssymb}
\usepackage{amsthm}
\usepackage{indentfirst}
\usepackage{cleveref}
\usepackage{stencilmacros}

% Essential declarations
\theoremstyle{definition}
\newtheorem{defn}{Definition}

\theoremstyle{plain}
\newtheorem{thm}{Theorem}
\newtheorem{lem}{Lemma}
\newtheorem{prop}{Proposition}
\newtheorem{case}{Case}

\theoremstyle{remark}
\newtheorem{remark}{Remark}

\begin{document}

\title{Lattice model}
\author{Mistral Contrastin}

\section{Giving specifications meaning}

We construct a lattice and use this lattice to give meanings to regions. We
then give full meaning to specifications by wrapping regions in variant types
for multiplicity and approximation. Model defined in this section will later be
used for two purposes. First, we show soundness of the equational theory of
specifications with respect to the model. Second, we define inference and
consistency checking procedures of specifications with respect to the
interpretations of the stencils in the code and the specifications.

\subsection{Constructing the lattice model of regions}

\begin{defn}
  We define the set \zinf{} as
  $\mathbb{Z}$ extended with $\infty$ and $-\infty$. For any $n$ in
  \zinf{}, we have $-\infty \leq n \leq \infty$. The resulting set is
  a total order with top and bottom elements.
\end{defn}

\begin{defn}
  We define an extended notion of closed interval on $\zinf$ which may
  contain a ``hole'' at the origin, written \interv{a}{b}{c} where $a$ and $b$
  are drawn from \zinf{} with $a \leq 0 \leq b$, $a$ is $-\infty$ \emph{iff} $b$
  is $\infty$, and $c$ is drawn from $\mathbb{B} = \{ \mathsf{true}, \mathsf{false} \}$. We
  define it as follows:
\marginpar{We might add the restriction to $[-\infty, \infty]$ later,
  but okay for now.}
%
  \begin{equation*}
    \interv{a}{b}{c} \triangleq
      \{ n \;|\; a \leq n \leq b \wedge (\neg c \implies n \neq 0) \}
  \end{equation*}

  Set of all such intervals (sets) is $\textit{Interval}$. If the superscript is
  leftout it is treated as $\mathsf{true}$.
\end{defn}

\begin{lem}{}\label{lem:zinf-identities}
  We have the following dual identities for \zinf{} intervals:
%
  \begin{align*}
    \interv{a}{b}{c} \cap \interv{d}{e}{f} & =
      \interv{\max \{a,d\}}{\min \{b,e\}}{c \wedge f} \\
    \interv{a}{b}{c} \cup \interv{d}{e}{f} & =
      \interv{\min \{a,d\}}{\max \{b,e\}}{c \vee f}
  \end{align*}
\end{lem}
%
\begin{proof}
  We give the proof of the first identity and the second one is similar.
  \begin{align*}
    \interv{a}{b}{c} \cap \interv{d}{e}{f} = &
      \; \{ n \;|\; a \leq n \leq b \wedge (\neg c \implies n \neq 0) \}
      \;\cap \\
      & \; \{ n \;|\; d \leq n \leq e \wedge (\neg f \implies n \neq 0) \}
      \\
    = & \; \{ n \;|\; \max \{a,d\} \leq n \leq \min \{b,e\} \wedge (\neg c
      \vee \neg f \implies n \neq 0) \} \\
    = & \; \interv{\max{\{a,d\}}}{\min{\{b,e\}}}{c \wedge f}
  \end{align*}
\end{proof}

\begin{defn}{(\emph{Offset subspace})}
  An $N$-dimenaional \emph{offset subspace} is a finite Cartesian product of $N$
  subsets of $\zinf{}$. The set of $N$-dimensional offset subspaces is written
  ${\mathcal{P}(\zinf{})}^N$.

  A offset spaces can be \emph{projected} in the $i^{\textit{th}}$ dimension by
  $\pi_i : {\mathcal{P}(\zinf{})}^N \to \mathcal{P}(\zinf{})$ where $i \in
  \mathsf{Fin}(N)$.
\end{defn}

\begin{defn}{(\emph{Subscript})}
  An $N$-dimensional offset subspace $S$ is called a \emph{subscript} if it
  satisfies:
%
  \begin{equation*}
    \forall i \in \textsf{Fin}(N).\
      \pi_i (S) = \{p\} \wedge p \in \mathbb{Z}
      \vee
      \pi_i (S) = \interv{-\infty}{\infty}{\mathsf{true}}
  \end{equation*}
  %
  That is, the $i^{th}$ component of the set is either a singleton
  in $\mathbb{Z}$ or the infinite interval.

  \todo{%
    We can move this explanation.

    Subscript spaces model array subscript expressions in our source langauge
    where each subscripted expression is either a constant translation of an
    induction variable within a loop or some other expression which is constant
    with respect to the loop e.g. \texttt{a(i+1, 0)}.
  }
\end{defn}

\begin{lem}{}\label{lem:vector-intersect}
  Intersection distributes over sets of points. If $\mathit{off}$ and
  $\mathit{off'}$ are sets of $N$-dimensional points, then
%
  \begin{equation*}
    \mathit{off} \cap \mathit{off'} =
      \prod_{i = 1}^{N} \pi_i(\mathit{off}) \cap \pi_i(\mathit{off'})
  \end{equation*}
\end{lem}
%
\begin{proof}
  \begin{align*}
    \mathit{off} \; \cap \; \mathit{off'} &
    = \{x \mid \bigwedge_{1 \leq i \leq N } x_i \in \pi_i(\mathit{off}) \}
      \;\cap\;
      \{x \mid \bigwedge_{1 \leq i \leq N } x_i \in \pi_i(\mathit{off'}) \} \\
    & = \{x \mid \bigwedge_{1 \leq i \leq N }
      (x_i \in \pi_i(\mathit{off}) \wedge x_i \in \pi_i(\mathit{off'})) \} \\
    & = \prod_{i = 1}^{N} \pi_i(\mathit{off}) \cap \pi_i(\mathit{off'})
  \end{align*}
\end{proof}

\begin{lem}{}\label{lem:vector-union}
  Let $\mathit{off}$ and $\mathit{off'}$ be two sets of
  points such that $\pi_i(\mathit{off})$ is equal to $\pi_i(\mathit{off'})$ for all $1
  \leq i \leq N$ except possibly at $k$, then we have the following:
%
  \begin{equation*}
    \mathit{off} \; \cup \; \mathit{off'}
    =
    \pi_1(\mathit{off}) \times \cdots \times
    (\pi_k(\mathit{off}) \cup \pi_k(\mathit{off'})) \times \cdots \times
    \pi_N(\mathit{off})
  \end{equation*}
\end{lem}
%
\begin{proof}
  \begin{align*}
    \mathit{off} \; \cup \; \mathit{off'} &
    = \{x \mid
      \bigwedge_{1 \leq i \leq N } x_i \in \pi_i(\mathit{off}) \}
      \;\cup\;
      \{x \mid
          \bigwedge_{1 \leq i \leq N } x_i \in \pi_i(\mathit{off'}) \} \\
    & = \{x \mid
          \bigwedge_{\substack{1 \leq i \leq N \\ i \neq k}}
            x_i \in \pi_i(\mathit{off}) \wedge x_k \in \pi_k(\mathit{off}) \vee
          \bigwedge_{\substack{1 \leq i \leq N \\ i \neq k}} x_i \in
            \pi_i(\mathit{off'}) \wedge x_k \in \pi_k(\mathit{off'})
        \} \\
    & = \{x \mid
          \bigwedge_{\substack{1 \leq i \leq N \\ i \neq k}} x_i \in
            \pi_i(\mathit{off}) \wedge
            x_k \in \pi_k(\mathit{off}) \cup \pi_k(\mathit{off'})
        \} \\
        & = \pi_1(\mathit{off}) \times \cdots \times
        (\pi_k(\mathit{off}) \cup \pi_k(\mathit{off'})) \times \cdots \times
        \pi_N(\mathit{off})
  \end{align*}
\end{proof}

\begin{defn}{(\emph{Interval subspace})}
  An interval subspace is a finite Cartesian product of intervals on \zinf{},
  denoted by the set $\textit{Interval}^N$ for a product of $N$ intervals. An
  interval subspace is an offset subspace.
\end{defn}

\begin{defn}{(\emph{Region})}
  A region \region{N} is a set of $N$-dimensional point intervals defined as
  the smallest set satisfying the following:
%
  \begin{enumerate}
    \item If $R$ is in $\textit{Interval}^N$, then $R$ is in \region{N}.
    \item If $R$ and $S$ are in \region{N}, then so are $R \cap S$ and
      $R \cup S$.
  \end{enumerate}
\end{defn}

\begin{prop}{}
  $(\region{N},\subseteq)$ is a distributive lattice.
\end{prop}
%
\begin{proof}
  Any sublattice of a distributive lattice is a distributive lattice. Power sets
  partially ordered with subset relation are distributive lattices. \region{N}
  is a subset of $\mathcal{P}{(\zinf{}^N)}$ since all interval subspaces are
  subsets of this power set and; union and intersection cannot add an element
  that is not already in the powerset. Inductive definition also ensures that
  \region{N} is closed under these operations. Therefore, \region{N} is a
  distributive sublattice of $(\mathcal{P}{(\zinf{}^N)}, \subseteq)$.
\end{proof}

\begin{defn}{}
  $\mathsf{Mult}$ and $\mathsf{Approx}$ are parametric labelled variant types
  with injections given by their definition:
%
  \begin{align*}
    \mathsf{Mult} \; a &
      \triangleq \mathsf{mult} \; a \;\mid\; \mathsf{only} \; a \\
    \mathsf{Approx} \; a &
      \triangleq \mathsf{exact} \; a \;\mid\; \mathsf{lower} \; a \;\mid\;
        \mathsf{upper} \; a \;\mid\; \mathsf{both} \; a \; a
  \end{align*}
%
  \eg{}, $\mathsf{lower}$ is an injection $\mathsf{lower} : a \to \mathsf{Approx}
  \; a$ etc.

  These will be used in the following subsection to give meaning to the
  specification modifiers for approximation and multiplicity.
\end{defn}

\subsection{Denotational semantics for specifications}

We give some basic interpretations: $\interp{\texttt{n}} = n$ where $n$ is in
$\mathbb{Z}^+$, $\interp{\texttt{pointed}} = \llbracket \epsilon \rrbracket =
\mathsf{true}$, $\interp{\texttt{nonpointed}} = \mathsf{false}$.
Typographically, \textcap{bold monotype font} is used within specifications
to indicate a partial textual capture. For convenience, we use \textit{italic
serif font} inside specifications rather than interpreting its textual capture.
This coalesces syntactic and semantic representation but avoids clutter in the
presentation.

Let $\textit{vec-gen}_N : \mathbb{N}^+ \times \textit{Interval} \to
\textit{Interval}^N$ be the function generating an $N$ dimensional vector
interval such that \vecgen{N}{i}{\interv{a}{b}{c}} has
$\interv{-\infty}{\infty}{\mathsf{true}}$ in all dimensions except the $i^{th}$
one. There, it has \interv{a}{b}{c}.

First, we give meaning to region constants (introduced in \Cref{}).
%
\begin{align*}
  \interp{\stencil{p}{$i$}{}{}}_N & =
    \vecgen{N}{i}{\interv{0}{0}{\mathsf{true}}}\\
%
  \interp{\stencil{c}{$i$}{$k$}{\textcap{p}}}_N & =
    \vecgen{N}{i}{\interv{-k}{k}{\interp{\textcap{p}}}} \\
%
  \interp{\stencil{f}{$i$}{$k$}{\textcap{p}}}_N & =
    \vecgen{N}{i}{\interv{0}{k}{\interp{\textcap{p}}}} \\
%
  \interp{\stencil{b}{$i$}{$k$}{\textcap{p}}}_N & =
  \vecgen{N}{i}{\interv{-k}{0}{\interp{\textcap{p}}}}
\end{align*}

Second, we provide the semantics for the permissive region operator, \texttt{+},
and the restrictive region operator, \texttt{*}, by associating them with join
and meet respectively. Let \textcap{R} and \textcap{S} be regions.
%
\begin{align*}
  \interp{\texttt{\textcap{R} + \textcap{S}}}_N & =
    \interp{\textcap{R}}_N \vee \interp{\textcap{S}}_N
&
  \interp{\texttt{\textcap{R} * \textcap{S}}}_N & =
    \interp{\textcap{R}}_N \wedge \interp{\textcap{S}}_N
\end{align*}

Third, we need to mark presence of modifiers such as \texttt{readOnce} and
\texttt{atMost} as introduced in \Cref{}. In the context of
multiplicity modifiers we have the following semantics:
%
\begin{align*}
  \interp{\texttt{readOnce}} & = \mathsf{mult} &
  \llbracket\epsilon\rrbracket & = \mathsf{once}
\end{align*}

In the context of approximation modifiers, we have
%
\begin{align*}
  \interp{\texttt{atLeast}} & = \mathsf{lower} &
  \interp{\texttt{atMost}} & = \mathsf{upper} &
  \llbracket \epsilon \rrbracket = \mathsf{exact}
\end{align*}

Finally, we can give full semantics of a stencil specification:
%
\begin{equation*}
  \interp{\texttt{stencil \textcap{Mult}, \textcap{Appr}, \textcap{Region}}}_N =
    \interp{\textcap{Mult}}
           {(\interp{\textcap{Appr}}
                    {(\interp{\textcap{Region}}_N)})}
\end{equation*}

\section{Equational \& approximation theories}

We know introduce the equational and approximation theory of stencil
specifications based on $\equiv$ and $\preceq$ in the following two subsections.

\subsection{Equivalences}

We define an equivalence relation, $\equiv$, on regions as follows:

\begin{description}
  \item[Basic] \texttt{*} and \texttt{+} are both idempotent, commutative, and
    associative;
%
  \item[Subsumption] If $S$ and $R$ are regions with $S \preceq R$, then
    $S \texttt{+} R \equiv R$ and $S \texttt{*} R \equiv S$.
%
  \item[Dist] \texttt{*} distributed over \texttt{+} and dually
    \texttt{+} distributes over \texttt{*}, meaning if \textcap{R}, \textcap{S},
    and \textcap{T} are regions, then we have the following dual equivalences:
%
    \begin{align*}
      \texttt{\textcap{R}*(\textcap{S}+\textcap{T})} & \equiv
        \texttt{(\textcap{R}*\textcap{S})+(\textcap{R}*\textcap{T})} &
      \texttt{\textcap{R}+(\textcap{S}*\textcap{T})} & \equiv
        \texttt{(\textcap{R}+\textcap{S})*(\textcap{R}+\textcap{T})}
    \end{align*}
%
  \item[Overlapping pointed] If \textcap{R} is one of \texttt{forward},
    \texttt{backward}, or \texttt{centered}, then we have
%
    \begin{equation*}
      \stencil{\textcap{R}}{$n$}{$k$}{\texttt{nonpointed}} \;\texttt{+}\;
      \stencil{p}{$n$}{}{} \equiv
      \stencil{\textcap{R}}{$n$}{$k$}{}
    \end{equation*}
%
  \item[Centered] The region constants \texttt{forward} and \texttt{backward}
    are two halves of \texttt{centered} specifications:
%
    \begin{align*}
      \stencil[s]{c}{$n$}{$k$}{\textcap{p1}} \equiv
        \stencil[s]{f}{$n$}{$k$}{\textcap{p2}} \texttt{+}
        \stencil[s]{b}{$n$}{$k$}{\textcap{p3}}
    \end{align*}
%
    Here \textcap{p1} is \texttt{nonpointed} if both \textcap{p2} and
    \textcap{p3} are \texttt{nonpointed} and \texttt{pointed} otherwise.
\end{description}

\begin{thm}{(Equational soundness)}
  Equational theory is sound with respect to the lattice model. Let $R$
  and $S$ be $N$ dimensional regions, then we have
%
  \begin{equation*}
    R \equiv S \implies \interp{R}_N = \interp{S}_N
  \end{equation*}
\end{thm}

\begin{proof}
  Both $\equiv$ and $=$ are equivalence relations. We only need to show that
  each property associated with specifications equivalence holds as an equality
  in the model.
%
  \begin{description}
    \item[Basic \& Subsumption] These follow from basic properties of lattices.
%
    \item[Dist] This is immediate from the definition of distributive
      lattice which the model is one.
%
    \item[Overlapping pointed] Counterparts of individual regions for some
      Boolean $p$ are given below:
%
      \begin{align*}
        \interp{\stencil{\textcap{R}}{$n$}{$k$}{\texttt{nonpointed}}}_N
        & = \vecgen{N}{n}{\interv{-k}{k}{\mathsf{false}}} \\
%
        \interp{\stencil{p}{$n$}{}{}}_N
        &= \vecgen{N}{n}{\interv{0}{0}{\mathsf{true}}} \\
%
        \interp{\stencil{\textcap{R}}{$n$}{$k$}{}}_N
        &= \vecgen{N}{n}{\interv{-k}{k}{\mathsf{true}}}
      \end{align*}

      Using \Cref{lem:vector-union}, it is sufficient to show
      $ \interv{-k}{k}{p} \;\cup\; \interv{0}{0}{\mathsf{true}} =
        \interv{-k}{k}{\mathsf{true}} $ holds. This immediately follows from
      \Cref{lem:zinf-identities}.
%
    \item[Centered] Let $\interp{\textcap{p1}}$, $\interp{\textcap{p2}}$, and
      $\interp{\textcap{p3}}$ be $p1$, $p2$, and $p3$ respectively. We
      know that $p1$ is $p2 \vee p3$ from (\textbf{Centered}) detinition. Then
      we get the following interpretation for region constants involved in the
      equivalence:
%
      \begin{align*}
        \interp{\stencil{c}{$k$}{$n$}{$p1$}}_N
          &= \vecgen{N}{n}{\interv{-k}{k}{p2 \vee p3}} \\
%
        \interp{\stencil{f}{$k$}{$n$}{$p2$}}_N
          &= \vecgen{N}{n}{\interv{0}{k}{p2} }\\
%
        \interp{\stencil{b}{$k$}{$n$}{$p3$}}_N
          &= \vecgen{N}{n}{\interv{0}{k}{p3}}
      \end{align*}
%
      Using \Cref{lem:vector-union}, it is sufficient to show $\interv{-k}{k}{p2
      \vee p3} = \interv{0}{k}{p2} \cup \interv{-k}{0}{p3}$ holds. This
      immediately follows from \Cref{lem:zinf-identities}.
  \end{description}
\end{proof}

\subsection{Approximations}
We define a partial of order of approximations, $\preceq$. This relation
provides us with the following:

\begin{description}
  \item[Equivalence] If $S$ and $R$ are regions and $S \equiv R$, then we have
    $S \preceq R$.
%
  \item[Combined] If $S$ and $R$ are regions, then we have
    $S \preceq S \texttt{+} R$ and $S \texttt{*} R \preceq S$.
%
  \item[Depth] Let $k$ and $l$ be in positive integers and $k \leq l$, $n$ some
    fixed dimension, and \textcap{p} either \texttt{pointed} or
    \texttt{nonpointed}. Further, let \textcap{R} be one of \texttt{centered},
    \texttt{forward}, and \texttt{backward}. We then have
%
    \begin{equation*}
      \stencil{R}{$n$}{$k$}{\textcap{p}} \preceq \stencil{R}{$n$}{$l$}{\textcap{p}}
    \end{equation*}
%
\end{description}

\begin{thm}{(Approximation soundness)}
  Theory of approximation is sound with respect to the lattice model. Let $R$
  and $S$ be $N$ dimensional regions, then we have
%
  \begin{equation*}
    R \preceq S \implies \interp{R}_N \subseteq \interp{S}_N
  \end{equation*}
\end{thm}
%
\begin{proof}
  Both $\preceq$ and $\subseteq$ are partial orders on sets. We only need to
  show that each inequality holds in the model.
%
  \begin{description}
    \item[Equivalence] Region equavilance is modeled with set equality and the
      partial order on regions is modeled with subset relation. Partial orders
      are reflexive and the model of eqivalence agrees with that of the partial
      order, so this rule holds.
%
    \item[Combined] \texttt{*} and \texttt{+} maps to meet and join operations
      in the lattice. All meets and joins in a lattice satisfy these
      inequalities.

    \item[Depth] When $m$ is $k$ and $l$ in
      $\stencil{\textcap{R}}{$n$}{$m$}{\textcap{p}}$, we obtain the
      interpretations $\vecgen{N}{n}{\textit{int}_k}$ and
      $\vecgen{N}{n}{\textit{int}_l}$ for some $N$ as interpretations of
      regions. Call these \textit{ks} and \textit{ls} respectively. One way to
      show $\textit{ks} \subseteq \textit{ls}$ is to show that the intersection
      of \textit{ks} and \textit{ls} is \textit{ks}.

      To show that the intersection is \textit{ks}, we use
      \Cref{lem:vector-intersect} which reduces the problem to showing that the
      intersection of $\textit{int}_k$ and $\textit{int}_l$ is the former since
      at every point except $n$ the intervals agree, hence the intersection
      remains the same at these dimensions. Then we use
      \Cref{lem:zinf-identities} to calculate the intersection. By assumption we
      have $k \leq l$, so the maximum of the lower bounds is $-k$ and the
      minimum of lower bounds is $k$ this is same as interval $\textit{int}_k$.
      Note that pointedness is the same for both intervals and depths are
      positive integers, so the pointedness is preserved under intersection.
      This shows the intersection of \textit{ls} and \textit{ks} is indeed
      \textit{ks} and concludes the proof.
  \end{description}
\end{proof}

We present few inequalities that can be derived from the axioms and are useful
when writing specifications without inference:

\begin{prop}{(Centered approximation)}
  For any dimension $n$, depth $k$, and pointed attribute $p$,
  we have
%
  \begin{align*}
    \stencil{f}{$n$}{$k$}{\textcap{p}} & \preceq
      \stencil{c}{$n$}{$k$}{\textcap{p}} \\
%
    \stencil{b}{$n$}{$k$}{\textcap{p}} & \preceq
      \stencil{c}{$n$}{$k$}{\textcap{p}}
  \end{align*}
\end{prop}
%
\begin{proof}
  \shortv{%
    Both inequalities are proved by combining the missing half of centered by
    using the inequality (\textbf{Combined}) and then merging them into a single
    region constant using (\textbf{Centered}).
  }

  \longv{%
    Proof of the first inequality:
    \begin{align*}
      \stencil[s]{f}{$n$}{$k$}{\textcap{p}}
        & \preceq
          \stencil[s]{f}{$n$}{$k$}{\textcap{p}}
          \;\texttt{+}\;
          \stencil[s]{b}{$n$}{$k$}{\textcap{p}}
        & (\textbf{Combined}) \\
      & \equiv \stencil[s]{c}{$n$}{$k$}{\textcap{p}} & (\textbf{Centered})
    \end{align*}

    Proof of the second inequality:
    \begin{align*}
      \stencil[s]{b}{$n$}{$k$}{\textcap{p}}
        & \preceq
          \stencil[s]{f}{$n$}{$k$}{\textcap{p}}
          \;\texttt{+}\;
          \stencil[s]{b}{$n$}{$k$}{\textcap{p}}
        & (\textbf{Combined}) \\
      & \equiv \stencil[s]{c}{$n$}{$k$}{\textcap{p}} & (\textbf{Centered})
    \end{align*}
  }
\end{proof}

\begin{prop}{(Point approximation)}
  Let \textcap{R} be one of \texttt{forward}, \texttt{backward}, and
  \texttt{centered}, $n$ a fixed dimension, and $k$ a fixed depth, then we have
%
  \begin{align*}
    \stencil{p}{$n$}{}{} & \preceq \stencil{R}{$n$}{$k$}{} \\
%
    \stencil{R}{$n$}{$k$}{\texttt{nonpointed}} & \preceq
      \stencil{R}{$n$}{$k$}{}
  \end{align*}
\end{prop}
%
\begin{proof}
  \shortv{%
    Both inequalities are proved by breaking \textcap{R} using
    (\textbf{Overlapping pointed}) followed by the inequality
    (\textbf{Combined}).
  }

  \longv{%
    Proof of the first inequality:
    \begin{align*}
      \stencil[s]{p}{$n$}{}{}
        & \preceq \stencil{R}{$n$}{$k$}{nonpointed}
                  \;\texttt{+}\;
                  \stencil[s]{p}{$n$}{}{}
        & (\textbf{Combined}) \\
      & \equiv \stencil{R}{$n$}{$k$}{} & (\textbf{Overlapping pointed})
    \end{align*}

    Proof of the second inequality:
    \begin{align*}
      \stencil{R}{$n$}{$k$}{nonpointed}
        & \preceq \stencil{R}{$n$}{$k$}{nonpointed}
                  \;\texttt{+}\;
                  \stencil[s]{p}{$n$}{}{}
        & (\textbf{Combined}) \\
      & \equiv \stencil{R}{$n$}{$k$}{} & (\textbf{Overlapping pointed})
    \end{align*}
  }
\end{proof}

\section{Analysis, Consistency, Inference}

\subsection{Analysis}

\subsection{Consistency}

Once the specification is synthesised into objects in the model and stencil
under examination is analysed, consistency checking itself is remarkably simple.
Two stage checking is outlined below.

\begin{enumerate}
  \item Multiplicity of the model is checked against the stencil. The only way
    this fails is if the original specification involved \texttt{readOnce}, but
    some indexing occur multiple times in the stencil.  Note that absence of
    \texttt{readOnce} is interpreted liberally, and if the stencil indices
    happen to occur uniquely, this does not constitute a failure.
  \item Definition of vector intervals is completely unwinded into sets of
    offset tuples and depending the approximation modifier (or lack thereof)
    we compare these tuples to offsets of the stencil computation. Lower and
    upper bounds are checked with subset and superset relations respectively,
    while absence of \texttt{atMost} and \texttt{atLeast} results in exact
    checking with set equality.
\end{enumerate}

Here is the mathematical definition of consistency checking:
\begin{align*}
  \mathit{consistent}(\mathit{ix}, \mathit{model}) & = \begin{cases}
    \mathsf{false} & \mathit{model} = \mathsf{once}(x) \wedge ix = \mathsf{mult}(y) \\
    \mathit{consistent'}(\mathit{peel}(ix), \mathit{peel}(model)) & \textit{otherwise}
  \end{cases} \\
%
  \mathit{consistent'}(\mathit{ix}, \mathit{model}) & = \begin{cases}
    m = \mathit{ix} & \mathit{model} = \mathsf{exact}(m) \\
    m \supseteq \mathit{ix} & \mathit{model} = \mathsf{upper}(m) \\
    m \subseteq \mathit{ix} & \mathit{model} = \mathsf{lower}(m)
  \end{cases}
\end{align*}

Here helper function $\mathit{peel} : \mathsf{Approx} \; a \to a$ removes the
approximation label.

Due to use of infinite intervals, the model and index sets are potentially
infinite meaning simple procedures used to compute equality, subset, and
superset relations for finite sets are no avail. The subset and superset
relations are handled by computing the intersection of $m$ and $\mathit{ix}$
using \Cref{lem:vector-intersect}, then checking equality with $m$ and
$\mathit{ix}$ respectively.

\todo{Here we simply show \textit{consistency'} as set equality, but since sets
are infinite we might need to elaborate based on representation.}

\subsection{Inference}

\todo{%
  Definition of origin is not needed.

  \begin{defn}{(Origin)}
    In $N$-dimensions, the \emph{origin} $\mathbf{0}_N$ is the singleton set
    $\{0^N\}$, or equivalently the vector of unit intervals $\interv{0}{0}{N}$.
  \end{defn}
}

\begin{defn}{(Extended origin distance)}
  We measure the \emph{extended origin distance} of a point from the extended
  origin as the minimum number of translations by $\pm 1$ in one dimension at a
  time, to reach a point in the extended origin.

  This notion extends to subscripts. Since there may be an infinite number of
  points, we measure the distance from the closest one.
\end{defn}
%
\begin{prop}{}
  The closed form of extend origin distance for an $N$-dimensional subscript $S$
  is denoted with $\mathit{dist}(S)$ and is defined as:
%
  \begin{align*}
    \mathit{dist}(S) & = \sum_{i=1}^{N} \mathit{dist}_i(S)
    &
    \mathit{dist}_i(S) & = \begin{cases}
      0 & \pi_i(S) = \interv{-\infty}{\infty}{} \\
      |\mathsf{a}| - 1 & \pi_i(S) = \{\mathsf{a}\} \\
    \end{cases}
  \end{align*}
\end{prop}
%
\begin{proof}
  Now we prove that function $\mathit{dist}$ indeed gives the lowest number of
  unit translations to reach the extended origin.
\end{proof}

\begin{defn}{(Adjacent)}
  A subscript $S$ is said to be \emph{adjacent at index $j$} of another
  subscript $S'$ if for every index $i$ except $j$, $\pi_i(S) = \pi_i(S')$ but
  at $j$ then $\pi_j(S) \neq \pi_j(S')$ such that $|S| = |S'| + 1$.
\end{defn}
%
\begin{prop}
  The unique adjacent at index $i$ of a subscript, if it exists, is given by the
  following partial function:
%
  \begin{align*}
    \mathit{adjacent}(S,i) & = \prod^{N}_{j=1} \mathit{shift}(S,i,j) &
%
    \mathit{shift}(S,i,j) & =
      \begin{cases}
        \{a - 1\} & i = j \wedge \pi_i(S) = \{\mathsf{a}\} \wedge \mathsf{a} > 0\\
        \{a + 1\} & i = j \wedge \pi_i(S) = \{\mathsf{a}\} \wedge \mathsf{a} < 0\\
        \pi_j(S)  & otherwise
      \end{cases}
  \end{align*}
\end{prop}
%
\begin{proof}
\todo{Hm- I guess uniqueness follows from this being a function rather
  than a relation? Seems fine no problem to prove- }
\end{proof}

\todo{On the following, add some intutition first- what is this and
  what is it for?}
\begin{defn}{(Exact set)}
  The \emph{exact set} at origin distance $d$, $\mathit{ES}_{P,d}$, of a group
  of $N$ dimensional points $P$ is defined below:
%
  \begin{align*}
    \mathit{ES}_{P,0} & = \{ p \in P \mid \forall \i.\ \pi_i(p) \in \interv{-1}{1}{} \} \\
    \mathit{ES}_{P,d} & =
      \{ p \in P \mid \forall i.\
      p_i \notin [-1,1] \implies
         \exists p' \in \mathit{ES}_{P,d-1}.\ p' =
         \mathit{adjacent}(p,i) \}
  \end{align*}
%
  The overall \emph{exact set}, $\mathit{ES}_P$ is then
  $\bigcup_{1 \leq d \leq D} \mathit{ES}_{P,d}$, where $D$ is the maximum origin
  distance in $P$.
\end{defn}

\begin{lem}
  The following three statements are equivalent for a set of points $P \subseteq
  \mathbb{Z}^N$:
%
  \begin{enumerate}
    \item $P$ defines a superposition of hyper-rectangles each of which crosses
      the extended origin;
    \item $P$ is the same as its exact set $\mathit{ES}_P$;
    \item $P$ when regarded as offsets from induction variables forms a
      neighbourhood stencil;
    \item $P$ is a subset of \region{N}.
  \end{enumerate}
\end{lem}

\begin{thm}{(Soundness of exact shape inference)}
  For a set of point offsets vectors $P$, the inference procedure for
  neighbourhood stencils is sound if the following holds:
%
  \begin{equation*}
    \mathit{consistent}(\mathsf{mult}(P),\mathsf{mult}(\mathit{infer}(P)))
  \end{equation*}
\end{thm}

\begin{thm}{(Completeness of inference)}\label{thm:inf-completeness}
  Inference procedure can infer a specification for every stencil computation.
\end{thm}
%
\begin{proof}
  Would be nice to have.
\end{proof}

\begin{thm}{(Completeness of exact inference)}
  Inference procedure can infer a specification for every stencil computation
  with array indexing behaviour forms a superposition of hyperrectangles removed
  at most one unit from the origin in any of its dimension.
\end{thm}
%
\begin{proof}
  Would be nice to have.
\end{proof}

\begin{thm}{(Optimality of inference)}\label{thm:inf-optimality}
  For exact inference, inference procedure always infers specifications with
  fewest number of region constants.
\end{thm}
%
\begin{proof}
  Would be nice to have.
\end{proof}

\begin{remark}{}
  Optimality of general inference is trivial. We only need a single region
  constant covering accesses in any one of the stencil's dimensions and that
  would provide an upper bound for the stencil.
\end{remark}

\end{document}
