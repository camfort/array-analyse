\newcommand{\off}{S}
\newcommand{\offP}{T}

\vectorIntersect*

\begin{proof}
  \begin{align*}
    \off \; \cap \; \offP &
    = \{x \mid \bigwedge_{1 \leq i \leq N } x_i \in \pi_i(\off) \}
      \;\cap\;
      \{x \mid \bigwedge_{1 \leq i \leq N } x_i \in \pi_i(\offP) \} \\
    & = \{x \mid \bigwedge_{1 \leq i \leq N }
      (x_i \in \pi_i(\off) \wedge x_i \in \pi_i(\offP)) \} \\
    & = \prod_{i = 1}^{N} \pi_i(\off) \cap \pi_i(\offP)
  \end{align*}
\end{proof}

\vectorUnion*

\begin{proof}
  \begin{align*}
    \off \; \cup \; \offP &
    = \{x \mid
      \bigwedge_{1 \leq i \leq N } x_i \in \pi_i(\off) \}
      \;\cup\;
      \{x \mid
          \bigwedge_{1 \leq i \leq N } x_i \in \pi_i(\offP) \} \\
    & = \{x \mid
          \bigwedge_{\substack{1 \leq i \leq N \\ i \neq k}}
            x_i \in \pi_i(\off) \wedge x_k \in \pi_k(\off) \vee
          \bigwedge_{\substack{1 \leq i \leq N \\ i \neq k}} x_i \in
            \pi_i(\offP) \wedge x_k \in \pi_k(\offP)
        \} \\
    & = \{x \mid
          \bigwedge_{\substack{1 \leq i \leq N \\ i \neq k}} x_i \in
            \pi_i(\off) \wedge
            x_k \in \pi_k(\off) \cup \pi_k(\offP)
        \} \\
        & = \pi_1(\off) \times \cdots \times
        (\pi_k(\off) \cup \pi_k(\offP)) \times \cdots \times
        \pi_N(\off)
  \end{align*}
\end{proof}

\intervalIdentities*

\begin{proof}
  We give the proof of the first identity and the second one is similar.
  \begin{align*}
    \interv{a}{b}{c} \cap \interv{d}{e}{f} = &
      \; \{ n \;|\; a \leq n \leq b \wedge (\neg c \implies n \neq 0) \}
      \;\cap \\
      & \; \{ n \;|\; d \leq n \leq e \wedge (\neg f \implies n \neq 0) \}
      \\
    = & \; \{ n \;|\; \max \{a,d\} \leq n \leq \min \{b,e\} \wedge (\neg c
      \vee \neg f \implies n \neq 0) \} \\
    = & \; \interv{\max{\{a,d\}}}{\min{\{b,e\}}}{c \wedge f}
  \end{align*}
\end{proof}

\regionLattice*

\begin{proof}
  Straightforward, the join and meet are mapped to $\cup$ and $\cap$, the set is
  inductively designed to be closed under these operations. Union and
  intersection are associative, commutative, and absroptive under closed sets
  and then they are also for \region{N}. This is enough to show that, it is a
  lattice.

  Further, we have $\bz{}^N \cap R = R$ and $\emptyset \cup R = R$ with
  $\bz{}^N$ and $\emptyset$ belonging to \region{N}. This makes the lattice a
  bounded one.

  Finally, the lattice is distributive since union distributes over intersection
  and vice versa when the set is closed under these operations.
\end{proof}
