\documentclass{article}

\usepackage{amsmath}
\usepackage{amssymb}
\usepackage{amsthm}
\usepackage{stmaryrd}
\usepackage{indentfirst}
\usepackage{cleveref}

\theoremstyle{definition}
\newtheorem{defn}{Definition}

\theoremstyle{plain}
\newtheorem{lem}{Lemma}
\newtheorem{prop}{Proposition}

\newcommand{\eg}{\emph{e.g.}}

\newcommand{\zinf}{\textnormal{$\mathbb{Z}^\infty$}}
\newcommand{\interp}[1]{\llbracket\texttt{#1}\rrbracket}
\newcommand{\interv}[3]{\textnormal{$\lbrack#1,#2\rbrack^{#3}$}}
\newcommand{\vecgen}[3]{\textnormal{$\textit{vec-gen}_{#1}(#2,#3)$}}

\begin{document}

\section{Giving specifications meaning}

We construct a lattice and use this lattice to give meanings to regions. We
then give full meaning to specifications by wrapping regions in variant types
for multiplicity and approximation. Model defined in this section will later be
used for two purposes. First, we show soundness of the equational theory of
specifications with respect to the model. Second, we define inference and
consistency checking procedures of specifications with respect to the
interpretations of the stencils in the code and the specifications.

\subsection{Constructing the lattice model of regions}

\begin{defn}{(\zinf)}
  We define the set \zinf{} as $\mathbb{Z}$ extended with $\infty$ and
  $-\infty$. For any $n$ in \zinf{}, we have $-\infty \leq n \leq \infty$. The
  resulting set is a total order with top and bottom elements.
\end{defn}

\begin{defn}{(\emph{Offset space})}
  \emph{Offset space} is a subset of \zinf{} members. $\mathcal{P}(\zinf{})$
  couple with $\supseteq$ is the power set distributive lattice.
\end{defn}

\begin{defn}{(\emph{Offsets vector})}
  An \emph{offsets vector} is a $N$ dimensional vector of offset spaces denoted
  by $(\textit{offs}_1, \dots, \textit{offs}_N)$.

  The set of $N$ dimensional offsets vectors is $\mathcal{P}{(\zinf{})}^N$. We
  define the partial order $\leq$ on $N$ dimensional offsets vectors
  \textit{voffs} and \textit{voffs'} as follows:
%
  \begin{equation*}
    \textit{voffs} \leq \textit{voffs'} \triangleq
      \forall n. 1
        \leq i \leq N \implies \textit{voffs}_i \supseteq \textit{voffs'}_i
  \end{equation*}
\end{defn}

\begin{prop}{}
  $(\mathcal{P}{(\zinf{})}^N, \leq)$ is a distributive lattice.
\end{prop}
%
\begin{proof}
  Product of two distributive lattices is itself a distributive lattice when the
  partial order is defined as componentwise comparison in the underlying
  lattice.
\end{proof}

\begin{defn}{(\emph{\zinf{} interval})}
  A \emph{\zinf{} interval} is of the form \interv{a}{b}{c}, where $a$ and $b$
  are drawn from \zinf{} with $a \leq 0 \leq b$ and $c$ is drawn from $\{
  \textsf{p}, \textsf{np} \}$. We define it as follows:
%
  \begin{equation*}
    \interv{a}{b}{c} \triangleq
      \{ n \;|\; a \leq n \leq b \wedge (c = \textsf{np} \implies n \neq 0) \}
  \end{equation*}

  Set of all such intervals (sets) is $\textit{Interval}$.
\end{defn}

\begin{prop}{}
  $(\textit{Interval}, \supseteq)$ is a sublattice of
  $(\mathcal{P}(\zinf{}),\supseteq)$ with \interv{-\infty}{\infty}{\textsf{p}}
  and \interv{0}{0}{\textsf{np}} as its top and bottom elements.
\end{prop}
%
\begin{proof}
  We know $\textit{Interval} \subseteq \mathcal{P}(\zinf{})$ since intervals
  represents sets drawn from {\zinf{}}

  Let \interv{a}{b}{c} and \interv{d}{e}{f} be two arbitrary \zinf{} intervals.
  We need to show that their meet and join both lie in \textit{Interval}.
%
  \begin{align*}
    \interv{a}{b}{c} \wedge \interv{d}{e}{f} = &
      \{ n \;|\; a \leq n \leq b \wedge (c = \textsf{np} \implies n \neq 0) \}
      \;\cup \\
      & \{ n \;|\; d \leq n \leq e \wedge (f = \textsf{np} \implies n \neq 0) \}
      \\
    = & \{ n \;|\; \min{a,d} \leq n \leq \max{b,e} \wedge ((c =
      \textsf{np} \wedge f = \textsf{np}) \implies n \neq 0) \} \\
    = & \interv{\min{\{a,d\}}}{\max{\{b,e\}}}{k}
  \end{align*}
%
  where $k$ is \textsf{np} if both $c$ and $d$ are \textsf{np} and \textsf{p}
  otherwise. Minimum of lower bound is still negative or zero and maximum of
  upper bounds is still positive or zero, hence the meet lies in
  \textit{Interval}.
%
  \begin{align*}
    \interv{a}{b}{c} \vee \interv{d}{e}{f} = &
      \{ n \;|\; a \leq n \leq b \wedge (c = \textsf{np} \implies n \neq 0) \}
      \;\cap \\
      & \{ n \;|\; d \leq n \leq e \wedge (f = \textsf{np} \implies n \neq 0) \}
      \\
    = & \{ n \;|\; \max{a,d} \leq n \leq \min{b,e} \wedge ((c =
      \textsf{np} \vee f = \textsf{np}) \implies n \neq 0) \} \\
    = & \interv{\max{\{a,d\}}}{\min{\{b,e\}}}{k}
  \end{align*}
%
  where $k$ is \textsf{np} if either $c$ or $d$ are \textsf{np} and \textsf{p}
  otherwise. Maximum of lower bound is still negative or zero and minimum of
  upper bounds is still positive or zero, hence the join lies in
  \textit{Interval}.

  \interv{-\infty}{\infty}{\textsf{p}} is then \zinf{} and
  \interv{0}{0}{\textsf{np}} is $\emptyset$. Since these are in the sublattice
  and they are the bottom and top elements in the parent lattice, they are
  necessarily the bottom and top elements in the sublattice.
\end{proof}

\begin{prop}{}
  $(\textit{Interval},\supseteq)$ is a distributive lattice.
\end{prop}
%
\begin{proof}
  It is enough to show for arbitrary $\interv{a}{b}{c}$, $\interv{d}{e}{f}$,
  $\interv{g}{h}{i}$ in \textit{Interval}, $\interv{a}{b}{c} \vee
  (\interv{d}{e}{f} \wedge \interv{g}{h}{i}) = (\interv{a}{b}{c} \vee
  \interv{d}{e}{f}) \wedge (\interv{a}{b}{c} \vee \interv{g}{h}{i})$. When this
  holds the dual law also holds. The property immediately follows from
  definition due to distributivity of set intersection over union.
\end{proof}
%
\begin{defn}{(\emph{Vector interval})}
  A vector interval is an $N$ dimensional object which has \zinf{} intervals at
  each dimension. It is denoted with $(\textit{int}_1, \dots, \textit{int}_n)$,
  where $\textit{int}_i$ is a \zinf{} interval. If \textit{vint} is a vector
  interval, $\textit{vint}_i$ denotes the \zinf{} interval at the $i^{th}$
  dimension.

  The set of $N$ dimensional vector intervals is $\textit{Interval}^N$. We
  define a partial order $\leq$ on $N$ dimensional vector intervals
  \textit{vint} and \textit{vint'} as follows:
%
  \begin{equation*}
    \textit{vint} \leq \textit{vint}' \triangleq
      \forall n. 1
        \leq i \leq N \implies \textit{vint}_i \supseteq \textit{vint'}_i
  \end{equation*}
\end{defn}
%
\begin{prop}{}
  $(\textit{Interval}^N, \leq)$ is a distributive lattice.
\end{prop}
%
\begin{proof}
  It is a standard result that product of two distributive lattices is a
  distributive lattice itself when the partial order on the resulting lattice is
  defined as the componentwise comparison. This generalises to $N$ dimensions by
  induction.
\end{proof}

\begin{defn}{}
  \textsf{Mult} and \textsf{Approx} are parametric labelled variant types with
  injections given by their definition:
%
  \begin{align*}
    \textsf{Mult} \; a &
      \triangleq \textsf{mult} \; a \;\mid\; \textsf{only} \; a \\
    \textsf{Approx} \; a &
      \triangleq \textsf{exact} \; a \;\mid\; \textsf{lower} \; a \;\mid\;
        \textsf{upper} \; a \;\mid\; \textsf{both} \; a \; a
  \end{align*}
%
  \eg{}, \textsf{lower} is an injection $\mathsf{lower} : a \to \mathsf{Approx}
  \; a$ etc.

  These will be used in the following subsection to give meaning to the
  specification modifiers for approximation and multiplicity.
\end{defn}

\subsection{Denotational semantics for specifications}

We give some basic interpretations: $\interp{\texttt{n}} = n$ where $n$ is in
$\mathbb{Z}$, $\interp{pointed} = \llbracket \epsilon \rrbracket = \textsf{p}$,
$\interp{\texttt{nonpointed}} = \textsf{np}$. Typographically, cursive font is
used within specifications to indicate a partial textual capture.

Let $\textit{vec-gen}_N : \mathbb{N} \times \textit{Interval} \to
\textit{Interval}^N$ be the function generating an $N$ dimensional vector
interval such that \vecgen{N}{i}{\interv{a}{b}{c}} has
$\interv{-\infty}{\infty}{\textsf{p}}$ in all dimensions except the $i^{th}$
one. There, it has \interv{a}{b}{c}.

First, we give meaning to region constants (introduced in \Cref{}).
%
\begin{align*}
  \interp{\texttt{pointed(dim=\textit{i})}}_N & =
    \vecgen{N}{\interp{\textit{i}}}{\interv{0}{0}{\textsf{p}}}\\
%
  \interp{\texttt{centered(dim=\textit{i}, depth=\textit{k}, \textit{p})}}_N & =
    \vecgen{N}
           {\interp{\textit{i}}}
           {\interv{-\interp{\textit{k}}}
                   {\interp{\textit{k}}}
                   {\interp{\textit{p}}}
           } \\
%
  \interp{\texttt{forward(dim=\textit{i}, depth=\textit{k}, \textit{p})}}_N & =
    \vecgen{N}
           {\interp{\textit{i}}}
           {\interv{0}
                   {\interp{\textit{k}}}
                   {\interp{\textit{p}}}
           } \\
%
  \interp{\texttt{backward(dim=\textit{i}, depth=\textit{k}, \textit{p})}}_N & =
    \vecgen{N}
           {\interp{\textit{i}}}
           {\interv{-\interp{\textit{k}}}
                   {0}
                   {\interp{\textit{p}}}
           }
\end{align*}

Second, we provide the semantics for the permissive region operator, \texttt{+},
and the restrictive region operator, \texttt{*}, by associating them with meet
and join respectively. Let \texttt{\textit{R}} and \texttt{\textit{S}} be
regions.
%
\begin{align*}
  \interp{\texttt{\textit{R} + \textit{S}}}_N & =
    \interp{\texttt{\textit{R}}}_N \wedge \interp{\texttt{\textit{S}}}_N
&
  \interp{\texttt{\textit{R} * \textit{S}}}_N & =
    \interp{\texttt{\textit{R}}}_N \vee \interp{\texttt{\textit{S}}}_N
\end{align*}

Third, we need to mark presence of modifiers such as \texttt{readOnce} and
\texttt{atMost} as introduced in \Cref{}. In the context of
multiplicity modifiers we have the following semantics:
%
\begin{align*}
  \interp{readOnce} & = \textsf{mult} &
  \llbracket\epsilon\rrbracket & = \textsf{once}
\end{align*}

In the context of approximation modifiers, we have
%
\begin{align*}
  \interp{atLeast} & = \textsf{lower} &
  \interp{atMost} & = \textsf{upper} &
  \llbracket \epsilon \rrbracket = \textsf{exact}
\end{align*}

Finally, we can give full semantics of a stencil specification:
%
\begin{equation*}
  \interp{stencil \textit{Mult}, \textit{Appr}, \textit{Region}} =
    \interp{\textit{Mult}}{(\interp{\textit{Appr}}(\interp{\textit{Region}}))}
\end{equation*}

\section{Equational theory}

\section{Analysis, Consistency, Inference}

\subsection{Analysis}

\subsection{Consisteny}

\subsection{Inference}

\end{document}
