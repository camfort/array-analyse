\begin{align*}
\setlength{\arraycolsep}{0.7em}
& \interp{\textit{approx}, \textit{mult}, \textit{region}}_n
= \interp{\textit{mult}}^m \; (\interp{\textit{approx}}^a \,
\interp{\textit{region}}_n)
\end{align*}
with interpretation on modifiers:
\begin{align*}
& \begin{array}{lll}
\interp{\varepsilon}^{a}   = \textsf{exact} &
\interp{\term{atMost}}^{a} \hspace{1.3em} = \textsf{upper} &
\interp{\term{atLeast}}^{a} = \textsf{lower} \\[0.3em]
\interp{\varepsilon}^{m} = \textsf{once}
& \interp{\term{readOnce}}^{m} = \textsf{mult} &
\end{array}\\[-1.5em]
\end{align*}
%

%\interp{approx\texttt{,} \; mult\texttt{,} \; region}_n = \\
%   \begin{cases}
%     \textsf{once}(\textsf{up}(\interp{\mathit{region}}_n)) &
%       \mathit{approx} = \texttt{atMost} \wedge \mathit{mult} = \texttt{readOnce} \\
%     \textsf{mult}(\textsf{up}(\interp{\mathit{region}}_n)) &
%       \mathit{approx} = \texttt{atMost} \wedge \mathit{mult} = \epsilon\\
%     \textsf{once}(\textsf{low}(\interp{\mathit{region}}_n)) &
%       \mathit{approx} = \texttt{atLeast} \wedge \mathit{mult} = \texttt{readOnce}\\
%     \textsf{mult}(\textsf{low}(\interp{\mathit{region}}_n)) &
%       \mathit{approx} = \texttt{atLeast} \wedge \mathit{mult} = \epsilon \\
%     \textsf{once}(\textsf{exact}(\interp{\mathit{region}}_n)) &
%       \mathit{approx} = \epsilon \wedge \mathit{mult} = \texttt{readOnce}\\
%     \textsf{mult}(\textsf{exact}(\interp{\mathit{region}}_n)) &
%       \mathit{approx} = \epsilon \wedge \mathit{mult} = \epsilon
%   \end{cases}