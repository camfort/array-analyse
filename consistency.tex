\begin{definition}
$unique(a,v)$ is the predicate indicating wheter indices vector of
array $v$ in assignment $a$ contain each element only once.
\end{definition}
%
\begin{dgroup*}
\begin{dmath*}
  cons(\textsf{once}(m),a,v) =
    {unique(a,v) \wedge cons(\textsf{mult}(m),a,v) }
\end{dmath*}
\begin{dmath*}
  cons(\textsf{mult}(\textsf{low}(m)),a,v) =
    {\forall u \in \textsf{analyse}(a)(v) \; \exists v \in m \cdot u \sim v }
\end{dmath*}
\begin{dmath*}
  cons(\textsf{mult}(\textsf{up}(m)),a,v) \quad =
    {\forall v \in m \; \exists u \in \textsf{analyse}(a)(v) \cdot u \sim v }
\end{dmath*}
\begin{dmath*}
  cons(\textsf{mult}(\textsf{exact}(m)),a,v) =
    {cons(\textsf{mult}(\textsf{up}(m)),a,v) \wedge
      cons(\textsf{mult}(\textsf{low}(m)),a,v) }
\end{dmath*}
\end{dgroup*}
