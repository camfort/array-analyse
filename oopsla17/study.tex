At the beginning of this project, we had the following hypotheses:
\begin{enumerate}
\item Loops over arrays mainly read from arrays with a static pattern
based on constant offsets from (base or dervied) induction variables;

\item Most loop-array computations of the previous form read
from arrays with a \emph{contiguous} pattern, \eg{}:
%
\begin{minted}{fortran}
do i = 1, n
  b(n) = (a(i) + a(i+1)) / 2.0
\end{minted}
%
\item Most loop-array computations of the previous form read
from arrays with a pattern that includes the immediate
neighbours (offsets of 1 from the induction variables);

\item Many array computations are \emph{stencil computations},
with a static access pattern as described in Hypothesis 2, writing
to an array at an index based on a (possibly constant offset from induction
variables, \eg{}, the classic five-point stencil;

\item Many array computations read from each particular index just once.
\end{enumerate}
%
From these hypotheses, we conjectured that the common programming
patterns captured above could be specified declaratively with a small
set of combinators capture the shape of data access patterns based on
overlapping (hyper)rectangles. We performed a large scale source-code
analysis to validate these hypotheses and guide the design of our
language. We believe these results are also of wider value.
\dnote{Explain why: what else do they tell us?}

\subsection{Methodology}
%
\noindent
We constructed a corpus of \numPackages{} scientific computing
packages written in Fortran ranging in size and scope: The Unified
Model (UM)~\cite{um}, E3MG~\cite{RePEc:aen:journl:2006se-a12},
BLAS~\cite{blas}, Hybrid4~\cite{GBC:GBC635},
GEOS-Chem~\cite{geos-chem}, Navier (based on
\cite{griebel1997numerical}), Computational Physics 2 (CP2)
~\cite{nicholas2006computational}, ARPACK-NG~\cite{arpackng}, and
SPECFEM3D~\cite{specfem3d}. \dnote{Include the rest} This covers
approximately 1.4 million lines of physical Fortran code (2.4 million
including comments and white space).  Appendix~\ref{app:corpus}
provides further detail on these packages and their sizes, with UM and
GEOS-Chem the largest at $\approx$630kloc and $\approx$450kloc respectively. We used
Wheeler's \emph{SLOCCount} to get a count of the physical source lines
(excluding comments and blank lines)~\cite{wheeler2001sloccount}.

Half of the packages come from active research teams who are our
project partners (\ie{}, they were not selected carefully to skew
results). BLAS was selected as it is a common numerical library;
similarly CP2 was selected as it gives standard numerical
analysis algorithms from a popular computational physics textbook.
The smallest project Navier ($\approx$500loc) is based on a standard textbook on
computational fluid dynamics.

We built a code analysis tool based on CamFort, an open-source analysis
tool for Fortran programs~\cite{camfort}. Fortran source files are
parsed to an AST and various standard control-flow and data-flow
analyses are computed, including of key importance for us: induction
variable identification, and reaching definitions. The resulting AST
is traversed top-down and assignment statements inside loops are
analysed and classified.

\paragraph{Classifications}

We focus on assignment statements where
the right-hand side reads from one or more arrays. Left-hand
sides and right-hand sides are classified in the following way:

\begin{tabular}{l|l|l}
  Shorthand & Classification (of subscripts) & Example \\ \hline
 (\textsf{vars}) & Variables (just LHS) & \fortran{x = ...}
  \\
 (\textsf{IVs}) & Induction variables (just LHS) & \fortran{a(i, j)} \\
 (\textsf{const}) & Constants & \fortran{a(0, 1)} \\
 (\textsf{neigh}) & Neighbourhood offsets (of the form
                                                  $i \pm c$) & \fortran{a(i, j-1)} \\
 (\textsf{neigh+c}) & Neighbourhood offsets and constants &
                                                            \fortran{b(i, 0, j+1)} \\
 (\textsf{aff}) & Affine offsets (of the form $a * i \pm b$) &
                                     \fortran{a(2*i+1,j)} \\
 (\textsf{aff+c}) & Affine offsets and constants & \fortran{a(i+1, 0,
                                                   3*j+2)} \\
 (\textsf{other}) & Subscript expression not included above &
\fortran{x(f(i))}, \fortran{a(i `mod` 2)}
\end{tabular} \\[1em]

\noindent
Note that the \textsf{vars} and \textsf{IVs} are only give to
left-hand sides.
The classification is computed from all information that flows into an
assignment, for example, the following decomposes a one-dimensional
three-point pattern across multiple intermediate assignments:
%%
\begin{minted}{fortran}
do i = 1, n
  x = a(i)
  y = a(i+1)
  b(i) = (a(i-1) + x + y)/3.0
end do
\end{minted}
%%x
Our analysis recognises this as one array computation (rather than three), starting at
line 4, and reading from subscripts \fortran{a(i)}, \fortran{a(i+1)},
\fortran{a(i-1)}. We thus traverse the body of a loop in reverse,
using reaching-definitions to calculate of the array reads that flow
to a particular assignment. Since \fortran{x} and \fortran{y} flow to
the line 4, line 2 and line 3 are not classified, though multiple
assignments can flow to multiple classified statements.

We further sub-classify sets of array subscripts
on the right-hand side based on their spatial properties: \\

\begin{tabular}{l||l|p{0.28\linewidth}|l}
  Property   & Shorthand & Classifications (of RHS pattern) & Example \\ \hline
%%%%%%
\multirow{2}{*}{Contiguity} & (\textsf{contig}) & Contiguous  &
                                                                \fortran{a(i) + a(i+1) + a(i+2)} \\
  & (\textsf{disjoint}) & Non-contiguous & \fortran{a(i) + a(i+2)} \\ \hline
%%%%%%
\multirow{2}{*}{Shape} & (\textsf{rect}) & (Hyper)rectangle &
 \fortran{a(i,j) + a(i+1,j) + a(i,j+1) + a(i+1,j+1)} \\
           & (\textsf{sumRect}) & Composed (hyper)rectangles &
  \fortran{a(i,j) + a(i,j-1) + a(i-1,j) + a(i+1,j) + a(i,j+1)} \\ \hline
%%%%%%
  \multirow{2}{*}{Reuse} & (\textsf{readonce}) & Unique subscripts
                                         & \eg{} \fortran{b(i) = a(i) + a(i+1)} \\
  & (\textsf{mult}) & Repeated subscripts &
\eg{}  \fortran{b(i) = a(i)  + a(i)}
\end{tabular} \\[1em]

\noindent
To assess the hypothesis that 
arrays are mostly read in a pattern that includes the immediate neighbours
to the ``origin'', we categorised the position of subscript patterns for each
right-hand side, in each dimension: \\

\begin{tabular}{l||l|p{0.33\linewidth}|l}
  Property   & Shorthand & Classification (indices per
                           dimension) & Example \\ \hline
%%%%%%
  \multirow{2}{*}{Positioning} & (\textsf{origin}) & Includes origin
& \fortran{a(i)} or \fortran{a(i+1, j)} (in $2^{\textit{nd}}$
  dimension) \\
             & (\textsf{straddle}) & Within distance 1 of origin &
\fortran{a(i+1), a(i-1)} \\
             & (\textsf{away}) & Away from the origin
                                          & \fortran{a(i+2), a(i+3)}
\end{tabular} \\[1em]

\noindent
Finally, we categorised the relationship between the left-hand side
and right-hand side, in terms of the use of induction variables.
The two sides are \emph{consistent} if the same induction variables
appear in each side, used for the same dimension. This is
weakened to a \emph{permutation} if the roles of the induction
variables changes. This is weakened further if the left-hand side
induction variables are either
a subset or superset of the induction variables on the right-hand
side. Otherwise, the two sides are seen as inconsistent:

\begin{tabular}{l|l|l}
  (\textsf{const}) & Consistent & \fortran{a(i, j) = b(i, j) +
                                  b(i+1,j+1)} \\
  (\textsf{perm} & Permutation & \fortran{a(i, j) = c(j, i)} or \fortran{a(i,
                                                               0) =
                                 b(0, i)} \\
  (\textsf{LHSsub} & LHS subset & \fortran{a(i) = b(i, j) + b(i, j-1)}
  \\
  (\textsf{LHSsup} & LHS superset & \fortran{a(i, j) = b(i)} \\
  (\textsf{inconst} & Inconsistent & \fortran{a(i) = b(j)}
\end{tabular} \\[1em]
%


%\begin{itemize}
% (Vars, ...)
%\item An assignment to a variable, with an RHS comprising some array
%  computation, tends to correspond to a reduction (e.g., calculating
%the max value in an array).
%\end{itemize}

\subsection{Results}

We first consider the initial hypotheses in turn again the collected data.
%
\begin{enumerate}
\item Loops over arrays mainly read from arrays with a static pattern
based on constant offsets from (base or dervied) induction variables;
\end{enumerate}
%
We identified $108,773$ instances of assignments within loops in
which an array subscript flows to the right-hand side. We refer
to each one of these as an \emph{array computation}.
We performed the classification described above
and grouped the data by the right-hand side classification only:
%%
\begin{center}
\begin{tabular}{lS[table-format=4]S[table-format=2.2]cc}
\textbf{RHS classification} & \textbf{Number} & \% &
\textit{grouping 1} & \textit{grouping 2} \\ \hline
Affine                          & 34        & 0.03\%  &
                                                            \multirow{4}{*}{92.61\%}
  \\ \cline{1-3}\cline{5-5}
All constant                    & 12160     & 11.18\%  \\ \cline{1-3}\cline{5-5}
Neighbourhood                   & 65334     & 60.06\% &
                                                          & \multirow{2}{*}{81.4\%}
                                                           \\ \cline{1-3}
Neighbourhood + constants       & 23209     & 21.34\%  \\ \hline
Other                           & 8036      & 7.39\%  \\ \hline \hline
Total                           & 108773    &  \\
\end{tabular}
\end{center}
%
\noindent
Since, affine, consant, and neighbour are all static patterns, we
group them (\textit{grouping 1}) to see that $92.61\%$ of the array
computations (\numexpr(100,737))
 have a static pattern that are either constants or
constant offsets of base/derived induction variables, confirming
\emph{hypothesis 1}. Note that affine (derived induction variables)
are rare. Overwhelmingly, the main category is neighbourhood
offsets (\emph{grouping 2}) (possibly mixed with constants in some
dimensions), at $81.4\%$ in total. The ``affine + constants'' class is
not represented at all.
%
\begin{enumerate}[resume]
\item Most loop-array computations of the previous form read
from arrays with a \emph{contiguous} pattern, \eg{}:
%
\begin{minted}{fortran}
do i = 1, n
  b(n) = (a(i) + a(i+1)) / 2.0
\end{minted}
%
\item Most loop-array computations of the previous form read
from arrays with a pattern that includes the immediate
neighbours (offsets of 1 from the induction variables);
\end{enumerate}
%
Focussing on grouping1 
\end{enumerate}