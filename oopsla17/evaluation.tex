\input{results}

To study the effectiveness of our approach, 
we built a corpus of over 1 million lines of Fortran code from a
range of scientific computing packages: The Unified Model (UM)~\cite{um},
E3ME~\cite{RePEc:aen:journl:2006se-a12}, BLAS~\cite{blas},
Hybrid4~\cite{GBC:GBC635}, GEOS-Chem~\cite{geos-chem}, Navier (based
on \cite{griebel1997numerical}), Computational Physics
ed. 1~\cite{giordano1997computational},
ARPACK-NG~\cite{arpackng}, MUDPACK~\cite{MUD}, Cliffs~\cite{Cliffs}, and
SPECFEM3D~\cite{specfem3d}.

We first examined how frequently stencil computations occur. We parsed
\num{\overalllinesParsed} lines of Fortran code and found that \overalltickAssignPercent\%
(\num{\overalltickAssign}) of statements have a left-hand side as an
array subscript on neighbourhood indices. This supports the idea that
stencil-like computations are common in scientific code.  We then used
the inference procedure of the previous section to generate
specifications for stencils in the corpus to assess the design of the
language.

We would not expect to infer a stencil for each array statement we
found because our analysis restricts the array-subscript-statements
that we classify as a stencil. In fact, we were able to infer a
stencil from \overalltickAssignSuccessPercentOfTickAssign\% of these
statements. A single statement can involve multiple arrays and we
ended up with \num{\overallnumStencilSpecs} specifications. This shows
that we can express a large number of stencil shapes within our
high-level abstraction and validates our initial hypothesis that many
stencil computations have a regular shape.
