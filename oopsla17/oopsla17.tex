\documentclass[acmlarge,review,anonymous]{acmart}\settopmatter{printfolios=true}

%% Some recommended packages.
\usepackage{booktabs}   %% For formal tables:
                        %% http://ctan.org/pkg/booktabs
\usepackage{subcaption} %% For complex figures with subfigures/subcaptions
                        %% http://ctan.org/pkg/subcaption

\usepackage{minted}
\usepackage{hyperref}
\usepackage{natbib}
\usepackage{amsmath}
\usepackage{amssymb}
\usepackage{syntax}
\usepackage{indentfirst}
\usepackage{cleveref}
\usepackage{xcolor}
\usepackage{siunitx} % For pretty-printing numeric values and SI units
                     % of measure. e.g., the tabular column type S is
                     % used to print nice-looking tables of numbers.
\sisetup{ % defaults
  group-separator={,},
  group-minimum-digits={3},
  output-decimal-marker={.},
  table-format = 6
}

%% Bibliography style
\bibliographystyle{ACM-Reference-Format}
%% Citation style
%% Note: author/year citations are required for papers published as an
%% issue of PACMPL.
\citestyle{acmauthoryear}   %% For author/year citations

\fvset{fontsize=\scriptsize}

\setlength{\grammarindent}{3em} % increase separation between LHS/RHS

% Writing macros
\newcommand{\eg}{\emph{e.g.}}
\newcommand{\ie}{\emph{i.e.}}

\newcommand{\inbar}{\ \ \textbar\ \ }

% Meta macros
\newcommand{\todo}[1]{\textcolor{blue}{#1}}

% For evluation and empirical study
% Defines the commands (headers)
\newcommand{\numPackages}{11}
% renews the commands with actual data
%\input{results}

\begin{document}
\title{Understanding and Verifying the Shape of Stencil Computations}

\author{Mistral Contrastin}
\affiliation{
  \department{Computer Laboratory}
  \institution{University of Cambridge}
}
\email{Mistral.Contrastin@cl.cam.ac.uk}

\author{Andrew Rice}
\affiliation{
  \department{Computer Laboratory}
  \institution{University of Cambridge}
}
\email{Andrew.Rice@cl.cam.ac.uk}

\maketitle

\section{Introduction}

\paragraph{Context of this research}

There is an increasing awareness of the need for program verification
techniques in
science~\cite{post2005computational,oberkampf2010verification,orchard2014computational}.
Whilst there are various kinds of automated and semi-automated
verification tools available, they see little use in science. This is
partly due to a lack of training and a lack of awareness. But its also
due to a lack of targetting. This paper is part of a line of research
to provide lightweight verification tools targetting
programming patterns common in computational science code, 
that are easy to use by scientists, and that integrate with common
practises and workflows.




\section{An Empirical Study of Stencils in Scientific Fortran code}



\begin{enumerate}
\item Stencils are ubiquitious;
\item Most stencils read their data from source arrays with a static pattern
base on constant offsets from the mid-point;
\item Most stencils read their data from contiguous \emph{regions} in an array;
\item Most stencils read each index at most once.
\end{enumerate}

\subsection{Study}
%
\noindent
We put together a corpus of around 1 million lines of Fortran code from a
\numPackages{} scientific computing packages ranging in size and scope: The Unified Model (UM)~\cite{um},
E3MG~\cite{RePEc:aen:journl:2006se-a12}, BLAS~\cite{blas},
Hybrid4~\cite{GBC:GBC635}, GEOS-Chem~\cite{geos-chem}, Navier (based
on \cite{griebel1997numerical}), Computational Physics
2~\cite{nicholas2006computational},
ARPACK-NG~\cite{arpackng}, and
SPECFEM3D~\cite{specfem3d}. Appendix~\ref{app:corpus} provides
further detail on these packages.

\section{A Specification Language for Shape}

\section{Analysis, Checking, and Inference}

\section{Evaluation}

\bibliography{references}

\appendix
% http://2017.splashcon.org/track/splash-2017-OOPSLA#Instructions-for-Authors
% "There is no page limit for bibliographic references and appendices,
% and, therefore, for the overall submission."
% http://2017.splashcon.org/track/splash-2017-OOPSLA#Instructions-for-Authors
% "There is no page limit for bibliographic references and appendices,
% and, therefore, for the overall submission."

\section{Details of the corpus data set}
\label{app:corpus}

\paragraph{Software Corpus}
Table~\ref{tab:corpus} shows summary statistics of the software
packages used in our evaluation, all of which are written in Fortran
90 or Fortran 77. In total we analysed \SI{\overallLoC} lines of code
from \numPackages{} packages, of which we successfully parsed
\SI{\overalllinesParsed} lines. The ``Number of files'' column shows
how many files in each corpus that we were able to analyse with
CamFort. The most common reasons for CamFort rejecting a file were
either use of a C-preprocessor, or illegal use of language features
from a modern Fortran variant.

\begin{enumerate}
\item \textbf{The Unified Model}~\cite{um} is a weather
  forecasting and climate modelling tool developed by the Met Office
  in the United Kingdom. It is used by research organisations and
  meteorological services around the world. We use the development
  branch (trunk) of the model. The code base in closed source but
  institutional licenses are available for research purposes. The Met
  Office runs a comprehensive code quality system incorporating
  dedicated committers (we counted 11) for particular parts of the
  model. We counted 120 additional contributors whose submissions are
  reviewed and tested before being accepted into the code base.

\item \textbf{E3MG} (An Energy-Environment-Economy (E3) Model at the Global
Level) is a macroeconomic model used for assessment of environmental
policy~\cite{RePEc:aen:journl:2006se-a12}. This was developed by
Cambridge Econometrics, an independent consultancy company.

\item \textbf{BLAS}~\cite{blas} (Basic Linear Algebra Subprograms) is
  a popular library providing efficient and portable routines for
  vector and matrix operations. These routines feature in many other
  libraries (including LAPACK). We used version 3.6.0. We chose to
  include this package for breadth, as it provides general numerical
  functions rather than a specialised scientific model.

\item \textbf{Hybrid4} is a vegetation and biomass model for
  simulating carbon, water and nitrogen flows~\cite{GBC:GBC635}.

\item \textbf{GEOS-Chem}~\cite{geos-chem} is a three-dimensional model
  of tropospheric chemistry developed at Harvard and used by $\sim$70
  universities and research institutions world-wide. We use v.10-01.

\item \textbf{Navier} is a small numerical simulation, giving a
  discrete approximation to the two-dimensional Navier-Stokes fluid
  equations, based on the book of~\citet{griebel1997numerical}.

\item \textbf{CP} consists of the example code from the second edition
  of the book ``Computational
  Physics''~\cite{nicholas2006computational} introducing numerical
  techniques and their application to modern physics problems such as
  fields, waves, statistical mechanics and quantum mechanics.

\item \textbf{ARPACK-NG}~\cite{arpackng} \todo{add a sentence of detail}

\item \textbf{SPECFEM3D}~\cite{specfem3d} \todo{add a little detail}
\end{enumerate}


\end{document}
