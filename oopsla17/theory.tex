\noindent
Our region specifications are subject to an equational theory
$\equiv$, which explains which region specifications are equivalent,
and a mutually-defined approximation theory $\preceq$ for
over- and under- approximation on regions.

\subsection{Equivalences}

\noindent
We define an equivalence relation, $\equiv$. The purpose of this
relation is to allow programmers to write specifications with greater
flexibility. It allows specifications to be written in various levels
of compactness allowing for space optimisation, or greater clarity of
documentation. The relation is defined on regions as follows:

\begin{description}
  \item[Basic] \texttt{*} and \texttt{+} are both idempotent, commutative, and
    associative;
%
  \item[Subsumption] If $S$ and $R$ are regions with $S \preceq R$, then
    $S \texttt{+} R \equiv R$ and $S \texttt{*} R \equiv S$.
%
  \item[Distribution] \texttt{*} distributes over \texttt{+} and dually
    \texttt{+} distributes over \texttt{*}, meaning if \textcap{R}, \textcap{S},
    and \textcap{T} are regions, then we have the following dual equivalences:
%
    \begin{align*}
      \texttt{\textcap{R}*(\textcap{S}+\textcap{T})} & \equiv
        \texttt{(\textcap{R}*\textcap{S})+(\textcap{R}*\textcap{T})} &
      \texttt{\textcap{R}+(\textcap{S}*\textcap{T})} & \equiv
        \texttt{(\textcap{R}+\textcap{S})*(\textcap{R}+\textcap{T})}
    \end{align*}
%
  \item[Overlapping pointed] If \textcap{R} is one of \texttt{forward},
    \texttt{backward}, or \texttt{centered}, then we have the following:
%
    \begin{equation*}
      \stencil{\textcap{R}}{$n$}{$k$}{\texttt{nonpointed}} \;\texttt{+}\;
      \stencil{p}{$n$}{}{} \equiv
      \stencil{\textcap{R}}{$n$}{$k$}{}
    \end{equation*}
%
  \item[Centered] The region constants \texttt{forward} and \texttt{backward}
    are two halves of \texttt{centered} specifications:
%
    \begin{align*}
      \stencil[s]{c}{$n$}{$k$}{\textcap{p1}} \equiv
        \stencil[s]{f}{$n$}{$k$}{\textcap{p2}} \texttt{+}
        \stencil[s]{b}{$n$}{$k$}{\textcap{p3}}
    \end{align*}
%
    Here \textcap{p1} is \texttt{nonpointed} if both \textcap{p2} and
    \textcap{p3} are \texttt{nonpointed}, otherwise \textcap{p1} is
    \texttt{pointed}.
\end{description}

\subsection{Approximations}

\noindent
We define a partial of order of approximations, $\preceq$. This relation is used
in the equational theory and provides a means of writing more compact lower and upper
bound specifications. The relation is defined as follows:

\begin{description}
  \item[Equivalence] If $S$ and $R$ are regions and $S \equiv R$, then we have
    $S \preceq R$.
%
  \item[Combined] If $S$ and $R$ are regions, then we have
    $S \preceq S \texttt{+} R$ and $S \texttt{*} R \preceq S$.
%
  \item[Depth] Let $k$ and $l$ be in positive integers and $k \leq l$, $n$ some
    fixed dimension, and \textcap{p} either \texttt{pointed} or
    \texttt{nonpointed}. Further, let \textcap{R} be one of \texttt{centered},
    \texttt{forward}, and \texttt{backward}. We then have
%
    \begin{equation*}
      \stencil{R}{$n$}{$k$}{\textcap{p}} \preceq \stencil{R}{$n$}{$l$}{\textcap{p}}
    \end{equation*}
%
\end{description}
%
We present some derivable inequalities that are useful
when writing specifications:

\begin{restatable}[Centered approximation]{prop}{centeredApprox}
  For any dimension $n$, depth $k$, and pointed attribute $p$,
  we have
%
  \begin{align*}
    \stencil{f}{$n$}{$k$}{\textcap{p}} & \preceq
      \stencil{c}{$n$}{$k$}{\textcap{p}} \\
%
    \stencil{b}{$n$}{$k$}{\textcap{p}} & \preceq
      \stencil{c}{$n$}{$k$}{\textcap{p}}
  \end{align*}
\end{restatable}

\begin{restatable}[Point approximation]{prop}{pointApprox}
  Let \textcap{R} be one of \texttt{forward}, \texttt{backward}, and
  \texttt{centered}, $n$ a fixed dimension, and $k$ a fixed depth, then we have
%
  \begin{align*}
    \stencil{p}{$n$}{}{} & \preceq \stencil{R}{$n$}{$k$}{} \\
%
    \stencil{R}{$n$}{$k$}{\texttt{nonpointed}} & \preceq
      \stencil{R}{$n$}{$k$}{}
  \end{align*}
\end{restatable}
%
If an array computation conforms to a specification $R$ and
$R \preceq S$ then it conforms also to $\texttt{atMost} \, S$.