
\noindent
We define a lattice model of array access patterns which serves as
a semantic model of our specification language and a model of array
access patterns in source code. This model (1) is used to explain the
meaning of our specifications; (2) provides a basis for the inference and
checking algorithms (\Cref{sec:algorithms}); (3) justifies an equational
theory for specifications in the next section; (4) is used to optimise
specifications using lattice identities; and (5) can be used to guide
correct implementations.

The model is defined over vectors of sets of integers which we call
\emph{index schemes}.  As an initial informal example, consider the
 stencil kernel \fortran{y(i) = x(i,0) + x(i+1,0)}.
The access pattern on array \term{x}, relative
to induction variable \term{i}, is captured
by a vector containing two integer sets: $\langle{\{0, 1\},
  \mathbb{Z}\}}\rangle$. This describes that, in the first dimension,
the array is read at offsets of $0$ and $1$ from an induction
variable. In the second dimension, the index is unconstrained
as it is a constant.

Index schemes form a lattice which provides a rich set of equations
and properties, which we exploit. We first set up the 
domain of the model (\S\ref{sec:domain}), using it to define a semantics
for our specification language (\S\ref{sec:semantics}) and then as
the target for an interpretation on imperative code (\S\ref{sec:fromcode}).
We state various results in this section, for which the proofs are
provided in \Cref{sec:proofs}.

\subsection{Lattice model of regions}
\label{sec:domain} 

%\begin{defn}[Extended integers]
%  We define the set \zinf{} as
%  $\mathbb{Z}$ extended with $\infty$ and $-\infty$. For any $a$ in
%  \zinf{}, we have $-\infty \leq a \leq \infty$. The resulting set is
%  a total order with top and bottom elements $-\infty$ and
%  $\infty$ respectively.
%\end{defn}

\begin{defn}[Index scheme]
  An $N$-dimensional \emph{index scheme} is a vector containing $N$
  integer sets, \ie{}, a member of $\mathcal{P}(\bz)^{N}$.
  An equivalent view of these vectors is as an $N$-times Cartesian
  product on subsets of $\bz$.
  We use $S, T, U$ to denote index schemes. Henceforth, we assume index schemes
  are all $N$-dimensional\footnote{The particular
    dimensionality is derived from array types when the model is
    used.} for some $N$.

  Index schemes can be \emph{projected} in the $i^{\textit{th}}$ dimension by
  $\pi_i : {\mathcal{P}(\bz{})}^N \to \mathcal{P}(\bz{})$. For an index scheme
  $I$, we refer to $\pi_i(S)$ as the \emph{$i^{th}$ component} of $S$. We assume
  that $i$, when used for projection, always lies between $1$ and $N$.
\end{defn}


\begin{restatable}{lem}{vectorIntersect}
\label{lem:vector-intersect}
  Intersection distributes over index schemes. That is, for index schemes $S, T
  \in \mathcal{P}(\bz)^N$
%
  \begin{equation*}
    S \cap T = \prod_{i = 1}^{N} \pi_i(S) \cap \pi_i(T)
  \end{equation*}
\end{restatable}

Union does not distribute over index schemes, however, a more restricted
property holds.

\begin{restatable}{lem}{vectorUnion}
\label{lem:vector-union} If
$S$ and $T$ are index schemes where $\pi_i(S) = \pi_i(T)$ for all $1 \leq i
\leq N$ apart from some dimension $k$, then:
%
  \begin{equation*}
    S \cup  T
    =
    \pi_1(S) \times \cdots \times
    (\pi_k(S) \cup \pi_k(T)) \times \cdots \times
    \pi_N(S)
  \end{equation*}
\end{restatable}
%
\begin{defn}[Intervals with an optional hole]
  We define an extended notion of closed interval on $\bz{}$ which may contain a
  \emph{hole} at the origin, written \interv{a}{b}{c} where $a$ and $b$ are
  drawn from \bz{} with $a \leq 0 \leq b$ and $c$ is drawn from $\mathbb{B} = \{
  \mathsf{true}, \mathsf{false} \}$. Intervals are interpreted as sets as
  follows:
%
  \begin{equation*}
    \interv{a}{b}{c} \triangleq
      \{ n \mid a \leq n \leq b \wedge (\neg c \implies n \neq 0) \}
  \end{equation*}
  %
  If the
  superscript to the interval is omitted it is treated as $\mathsf{true}$ (no hole).
  We also add the distinguished interval $\interv{-\infty}{\infty}{}$, which is simply an
  alias for \bz{}, but this notation avoids separate handling of the infinite interval
  in the following definitions, lemmas, and proofs. Here, $-\infty$
  and $\infty$ behave as top and bottom elements to \bz{} respectively.

  We denote the set of all such intervals as $\textit{Interval}$. 
\end{defn}
%
\begin{restatable}{lem}{intervalIdentities}
 \label{lem:interval-identities}
  We have the following dual identities for \bz{} intervals:
%
  \begin{align*}
    \interv{a}{b}{c} \cap \interv{d}{e}{f} & =
      \interv{\max \{a,d\}}{\min \{b,e\}}{c \wedge f} \\
    \interv{a}{b}{c} \cup \interv{d}{e}{f} & =
      \interv{\min \{a,d\}}{\max \{b,e\}}{c \vee f}
  \end{align*}
\end{restatable}
%

We define two specialisations of index schemes:
the \emph{subscript scheme} and the \emph{interval scheme}.

\begin{defn}[Interval scheme]\label{def:interval-scheme}
  An interval scheme is an $N$-length vector ($N$-ways Cartesian
  product) of intervals on \bz{}, denoted by the set
  $\textit{Interval}^N$ for a product of $N$ intervals.
\end{defn}

\begin{defn}[Subscript scheme]
  A \emph{subscript scheme} is an index scheme scheme where:
  %
  \begin{equation*}
    \forall i.\ 1 \leq i \leq N \implies
      \pi_i (S) = \{ p \}
      \; \vee \;
      \pi_i (S) = \interv{-\infty}{\infty}{}
    \end{equation*}
%
  That is, the $i^{th}$ component of the set is either a singleton in \bz{} or
  the infinite interval.
\end{defn}
%
\begin{defn}[Region]
  A region is an index scheme and \region{N} is the set of all regions (\ie{},
  $\region{N} \subseteq \bz^N)$. The set of all regions is  defined
  as the smallest set satisfying the following:
%
  \begin{enumerate}
    \item If $R$ is in $\textit{Interval}^N$, then $R$ is in \region{N}.
    \item If $R$ and $S$ are in \region{N}, then so are $R \cap S$ and
      $R \cup S$.
  \end{enumerate}
\end{defn}
%
\begin{restatable}{prop}{regionLattice}
  \label{prop:region-lattice}
  $(\region{N},\cup,\cap,\subseteq)$ is a bounded distributive lattice with top
  $\top = \bz{}^N$ and bottom $\bot = \emptyset$.
\end{restatable}
%
The set of regions $\region{N}$ is the target of our specification
language model, and of the code analysis. We further wrap this in a
labelled variant which provides information on multiplicity and approximation.

\begin{defn}
  \label{defn:modifiers}
  $\mathsf{Mult}$ and $\mathsf{Approx}$ are parametric labelled variant types
  with injections given by their definition:
%
  \begin{align*}
    \mathsf{Mult} \;\; a \;\,
      \triangleq \;\, \mathsf{mult} \; a \;\mid\; \mathsf{only} \; a \qquad\quad
    \mathsf{Approx} \;\; a \;\,
      \triangleq \;\, \mathsf{exact} \; a \;\mid\; \mathsf{lower} \; a \;\mid\;
        \mathsf{upper} \; a
  \end{align*}
%
  \eg{}, $\mathsf{lower}$ is an injection $\mathsf{lower} : a \to \mathsf{Approx}
  \; a$ etc.
%  These will be used in the following subsection to give meaning to the
%  specification modifiers for approximation and multiplicity.
\end{defn}

\subsection{Denotational semantics for specifications}
\label{sec:semantics}

\noindent
An interpretation function $\interp{-}_N$ maps closed\footnote{That
  is, we assume there are no occurrences of \textit{rvar} in a
  specification being modelled.  Any \emph{open} specification
  containing region variables can be made closed by straightforward
  syntactic substitution with a (closed) \textit{region}.}
specifications to sets of $N$-dimensional index schemes with modifier
information, \ie{} specifications are mapped to
$\textsf{Mult} (\textsf{Approx} (\region{N}))$.

The interpretation is overloaded on \emph{regions} in
\Cref{subfig:region-model}. Various intermediate notions are used.

\begin{defn}
  Let $\textit{promote}_N : \mathbb{N}^+ \times \textit{Interval} \to
  \textit{Interval}^N$ be a function generating an interval scheme such that if
  $v$ is $\vecgen{N}{i}{\interv{a}{b}{c}}$, then $\pi_i(v) = \interv{a}{b}{c}$
  and $\pi_j(v) = \bz{}$ in all other dimensions $j$.
\end{defn}

%
\begin{figure}[!t]
 \label{fig:semantics}
\begin{subfigure}[t]{0.4\textwidth}
\begin{align*}
  \interp{-}_N & : \textit{region} \rightarrow \region{N} \\
  %
   \interp{\texttt{\textcap{r} + \textcap{s}}}_N & =
    \interp{\textcap{r}}_N \vee \interp{\textcap{s}}_N
\\
  \interp{\texttt{\textcap{r} * \textcap{s}}}_N & =
   \interp{\textcap{r}}_N \wedge \interp{\textcap{s}}_N \\
%%
  \interp{\texttt{\textcap{rconst}}}_N & =
    \vecgen{N}{i}{\interp{\texttt{\textcap{rconst}}}^{\mathsf{c}}}
  \\[1em]
  %%
  \interp{-}^{\mathsf{c}} & : \textit{rconst} \rightarrow
                            \textit{Interval} \\
  \interp{\stencil{p}{$i$}{}{}}^{\mathsf{c}} & =
    {\interv{0}{0}{\mathsf{true}}}\\
%
  \interp{\stencil{c}{$i$}{$k$}{\textcap{p}}}^{\mathsf{c}} & =
    {\interv{-k}{k}{\interp{\textcap{p}}}} \\
%
  \interp{\stencil{f}{$i$}{$k$}{\textcap{p}}}^{\mathsf{c}} & =
    {\interv{0}{k}{\interp{\textcap{p}}}} \\
%
  \interp{\stencil{b}{$i$}{$k$}{\textcap{p}}}^{\mathsf{c}} & =
    {\interv{-k}{0}{\interp{\textcap{p}}}}
\end{align*}
\caption{Interpretation of regions}
\label{subfig:region-model}
\end{subfigure}
\hspace{1em}
\begin{subfigure}[t]{0.3\textwidth}
\begin{align*}
\interpApprox{-} & : \textit{approx} \rightarrow (A \rightarrow
  \textsf{Approx} \, A) \\
\interpApprox{\texttt{atLeast}} & = \mathsf{lower} \\
  \interpApprox{\texttt{atMost}} & = \mathsf{upper} \\
  \interpApprox{\epsilon} & = \mathsf{exact} \\ \\
  \interpMult{-} & : \textit{mult} \rightarrow (A \rightarrow
  \textsf{Mult} \, A) \\
  \interpMult{\texttt{readOnce}} & = \mathsf{once} \\
  \interpMult{\epsilon} & = \mathsf{mult}
\end{align*}
\caption{Interpretation of modifiers}
\label{subfig:modifier-model}
\end{subfigure}
\caption{Semantic model of specifications}
\end{figure}

The intermediate interpretation $\interp{-}^{\mathsf{c}}$ of
\Cref{subfig:region-model} models region contants. This is lifted by
$\textit{promote}_N$ to interpret region constants in the last
equation of $\interp{-}_N$. The \term{+} and \term{*} operators are
modelled in terms of the join (union) and meet (intersection) of
interval schemes. Thus the syntax of regions is modelled as members of
$\region{N}$.

We mark in our model the presence of modifiers such as
\texttt{readOnce} and \texttt{atMost}.  Approximation modifiers are interpreted as
injections into the $\mathsf{Approx}$ variant by $\interpApprox{-}$ in
\Cref{subfig:modifier-model}.  The $\textsf{Approx}$ type corresponds
to the presence or absence of the spatial approximation modifier, with
\textsf{exact} when there is no such modifier and \textsf{lower} and
\textsf{upper} for \term{atLeast} and \term{atMost} respectively. In a similar way,
multiplicity modifiers are interpreted as injections in the
$\mathsf{Mult}$ variant by $\interpMult{-}$, corresponding to
the presence or absence of the \term{readOnce} modifier as shown in
\Cref{subfig:modifier-model}.

\begin{defn}[Semantics of specifications]
The intermediate interpretations of \Cref{fig:semantics}
are composed to give a model for the top-level specification
syntax as:
%
\begin{equation*}
  \interp{\texttt{stencil \textcap{mult}, \textcap{approx}, \textcap{region}}}_N =
    \interpMult{\textcap{mult}} \;
           {(\interpApprox{\textcap{approx}} \;
                    {\interp{\textcap{region}}_N)}}
\end{equation*}
\end{defn}

\begin{restatable}[Equational soundness]{thm}{equationalSoundness}
  The lattice model is sound with respect to the equational theory. Let $R$ and $S$
  be $N$-dimensional region terms, then we have
%
  \begin{equation*}
    \forall R, S, N. \; R \equiv S \implies \interp{R}_N = \interp{S}_N
  \end{equation*}
\end{restatable}

\begin{restatable}[Approximation soundness]{thm}{approxSoundness}
  The lattice model is sound with respect to the theory of approximation. Let $R$
  and $S$ be $N$ dimensional regions, then we have
%
  \begin{equation*}
    \forall R, S, N. \; R \preceq S \implies \interp{R}_N \subseteq \interp{S}_N
  \end{equation*}
\end{restatable}
%
Note that the model is not complete with respect to equations or
approximations since the specification language has no model of the
bottom element of the lattice.

\subsection{Interpreting array subscripts}
\label{sec:fromcode}

\begin{defn}
  \label{defn:subscript}
  Recall array subscript terms of the form $a(\bar{e})$ from
  \Cref{def:array-subs}. We interpret these terms with the partial
  interpretation $\interp{-}^{\mathit{aterm}} : \textit{array-term}
  \prightarrow{} \mathcal{P}(\bz{})^N$. The interpretation is defined when
  all indices are either constant or neighbourhood indices as defined in
  \Cref{def:neighbour-ix}.
%
  \begin{align*}
    \interp{a(\bar{e})}^{\mathit{aterm}} =
      \prod_{1 \leq i \leq N} \mathit{subscript}(\bar{e}_i) & &
  %
    \textit{subscript}(e) = \begin{cases}
      \{ c \} & e \equiv i \pm c \\
      \bz & e \; \mbox{is absolute}
    \end{cases}
  \end{align*}
\end{defn}
%
Note that this produces subscript schemes (a form of index scheme), but
these are not members of \textit{Interval} or $\textit{Region}_N$. In
the next section, we use this interpretation on array subscripts in
the static analysis producer preceding checking and inference.

\subsection{Union Normal Form}\label{subsec:union-normal-form}

\noindent
We lastly address a problem of \emph{representation} for index schemes
in the model. Index schemes may incorporate the infinite set \bz{} in
their components, \eg{} when a dimension is unconstrained or when an
absolute index of array term is interpreted. In order to provide a
finite checking procedure, we require a finite and compact
representation for indices which accounts for this.

Recall that \Cref{lem:vector-intersect} states that the
intersection of index schemes is an index scheme. When
combined with \Cref{lem:interval-identities} (union and intersection
on holed intervals), we can extend this closure to \emph{interval} schemes.
Subscripts schemes do not enjoy the same closure propery as the intersection of
two singleton sets may be empty, disqualifying the result from 
being a subscript index. In the following sections, we do not take
intersection of subscript indices.

Union is not as flexible as
intersection. \Cref{lem:vector-union} does not form a closure for
indexing schemes in general. For this reason, we represent sets of
indexing schemes as unevaluated union terms, or \emph{union normal
  form}.  Since the model forms a distributive lattice as established
in \Cref{prop:region-lattice}, the intersections can be pushed inwards
using the distributive law leaving only unions at the outer-level. We
further exploit associativity of union to put unions into a cons-tree
making the unions effectively a non-empty list.  We do not, however,
attempt to canonicalise the representation. Later on
\Cref{lem:vector-union} and the index scheme subset are used to reduce the
size of union normal form.

%%% Local Variables:
%%% mode: latex
%%% TeX-master: t
%%% End:
